\section{多边形}

多边形的存储记录一般采用的是顺次记录其顶点的方法,如凸包的表示方法。方向一般为逆时针。

\subsection{多边形的面积}

任取 $n$ 边形的一个点 $P_0$ 为顶点,将该点与其它点连接,得到 $n-2$ 个三角形。利用叉积性质计算面积,求和累加后除 2 即可。因为叉积计算的是有向面积,所以两条边叉积得到的三角形,在多边形外面的那部分最后会抵消掉。

\begin{minted}{c++}
double polygonArea(Point p[], int n) {
    double area = 0.0;
    for (int i = 1; i < n-1; i++)
        area += cross(p[i] - p[0], p[i+1] - p[0]);
    return area / 2.0;
}
\end{minted}

\subsection{判断点在多边形内外}
\noindent 多边形的顶点必须以逆时针或顺时针顺序给出。

\begin{minted}{c++}
bool isPointInPolygon(Point p[], int n, Point a) {
    int wn = 0;
    for (int i = 0; i < n; i++) {
        if (checkIfPointOnSegment(a, p[i], p[(i+1) % n]))
            return true;
        if (a == p[i])
            return true;
        int k = dcmp(cross(p[(i+1)%n] - p[i], a - p[i]));
        int d1 = dcmp(p[i].y - a.y),
            d2 = dcmp(p[(i+1)%n].y - a.y);
        if (k > 0 && d1 <= 0 && d2 > 0)
            wn++;
        if (k < 0 && d2 <= 0 && d1 > 0)
            wn--;
    }
    return wn != 0;
}
\end{minted}

\subsection{三角形的重心}
三角形的重心是三条中线的交点。设 $AM, BN$ 分别是三角形 $BC$ 边和 $AC$ 边上的中线,则三角形的重心 $G$ 即为 $AM$ 和 $BN$ 的交点,满足公式 $G=\frac{A+B+C}{3}$,其中 $A+B+C$ 表示坐标分量相加。

性质:重心是中线的一个三等分点。
\begin{minted}{c++}
Point triangleBaycenter(Point A, Point B, Point C) {
    return Point((A.x + B.x + C.x) / 3.0, (A.y + B.y + C.y) / 3.0);
}
\end{minted}

\subsection{求质点组的重心}
$n$ 个质点,位置分别为 $A_1, A_2, ..., A_n$;质量为 $m_1, m_2, ..., m_n$;由重心公式有 $G = \frac{\Sigma_{k=1}^{n}{m_kA_k}}{\Sigma_{k=1}^{n}{m_k}}$.

\begin{minted}{c++}
Point getPollygonBaycenter(Point p[], double mass[], int n) { // not tested
    Point t = Point(0.0, 0.0);
    double sumMass = 0.0;
    for (int i = 0; i < n; i++) {
        t = t + p[i] * mass[i];
        sumMass += mass[i];
    }
    if (dcmp(sumMass) == 0)     // judge
        return Point(0, 0);
    return t / sumMass;
}
\end{minted}
