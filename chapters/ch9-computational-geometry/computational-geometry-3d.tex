\section{三维计算几何}

\begin{itemize}
    \item 基本定义与二维下的点、向量的定义相同。注意三维下点+点没有定义,但是点+向量=点、向量+向量=向量仍然成立。
    \item 点积、模长、两向量夹角的计算方法仍然与二维下的相同,只不过需要引入一个新的坐标 z 参与运算。
\end{itemize}

\subsection{基本定义}

\begin{minted}{c++}
inline int dcmp(double x) {
    if (fabs(x) < eps)
        return 0;
    return x < 0 ? -1 : 1;
}

inline double sq(double x) {
    return x * x;
}

struct Point3 {
    double x, y, z;
    Point3 (double _x = 0, double _y = 0, double _z = 0): x(_x), y(_y), z(_z) {}
    bool operator == (const Point3 &b) const {
        return dcmp(x - b.x) == 0 && dcmp(y - b.y) == 0 && dcmp(z - b.z) == 0;
    }
    bool operator < (const Point3 &b) const {
        return dcmp(x - b.x) == 0 ? dcmp(y - b.y) == 0 ? z < b.z : y < b.y : x < b.x; 
    }
    double len() {
        return sqrt(sq(x) + sq(y) + sq(z));
    }
    double len2() {
        return sq(x) + sq(y) + sq(z);
    }
    double dist(const Point3 &b) const {
        return sqrt(sq(x - b.x) + sq(y - b.y) + sq(z - b.z));
    }
    Point3 operator + (const Point3 &b) const {
        return Point3(x + b.x, y + b.y, z + b.z);
    }
    Point3 operator - (const Point3 &b) const {
        return Point3(x - b.x, y - b.y, z - b.z);
    }
    Point3 operator * (const double &k) const {
        return Point3(x * k, y * k, z * k);
    }
    Point3 operator / (const double &k) const {
        return Point3(x / k, y / k, z / k);
    }
    double operator * (const Point3 &b) const {
        return x * b.x + y * b.y + z * b.z;
    }
    Point3 operator ^ (const Point3 &b) const {
        return Point3(y * b.z - b.y * z, z * b.x - x * b.z, x * b.y - y * b.x);
    }
    double rad(Point3 a, Point3 b) {
        Point3 p = (*this);
        return acos((a - p) * (b - p) / (a.dist(p) * b.dist(p)));
    }
    Point3 trunc(double r) {
        double l = len();
        if (!dcmp(l))
            return *this;
        r /= l;
        return Point3(x * r, y * r, z * r);
    }
} a;

typedef Point3 Vector3;
\end{minted}

\subsection{求两向量夹角}
\begin{minted}{c++}
double angle(Vector3 a, Vector3 b) {
    return acos(a * b / a.len() / b.len());
}
\end{minted}

\subsection{点、线、平面}
\begin{minted}{c++}
// 点 P 到平面 P0-n 的距离,P0 为平面内一点,n 为平面的单位法向量
double distanceToPlane(Point3 p, Point3 p0, Vector3 n) {
    return fabs((p - p0) * n);
}
// 点 P 在平面 P0-n 上的投影,同上
Point3 projection(Point3 p, Point3 p0, Vector3 n) {
    return p - n * ((p - p0) * n);
}
// 直线与平面的交点
Point3 linePlaneIntersection(Point3 p1, Point3 p2, Point3 p0, Vector3 n) {
    Vector3 v = p2 - p1;
    double t = (n * (p0 - p1)) / (n * (p2 - p1));   // 记得判断分母 $\neq$ 0
    return p1 + v * t;              // 如果是线段,判断 $t \in [0, 1]$
}
\end{minted}

\subsection{叉积应用}
\begin{minted}{c++}
// 过不共线的三点的平面,法向量为 $(p_2 - p_0) \times (p_1 - p_0)$.

// 三角形的有向面积
double area2(Point3 a Point3 b, Point3 c) {
    return ((b - a) ^ (c - a)).len();
}

// 判断点是否在三角形内
bool pointInTriangle(Point3 p, Point3 p0, Point3 p1, Point3 p2) {
    double s1 = area2(p, p0, p1),
        s2 = area2(p, p1, p2),
        s3 = area2(p, p2, p0);
    return dcmp(s1 + s2 + s3 - area2(p0, p1, p2)) == 0;
}
\end{minted}

\subsection{向量旋转}
\begin{minted}{c++}
// 向量 v 绕向量 u 转角 theta
Vector3 doRotate(Vector3 v, Vector3 u, double theta) {
    Vector3 res = v * cosl(theta) +  u * ((u * v) * (1.0 - cosl(theta))) + (u ^ v) * sinl(theta);
    return res;
}
\end{minted}

\subsection{点沿直线移动}
\begin{minted}{c++}
// 点 p 沿方向向量 dir 移动距离 d
Point3 forward(Point3 p, Vector3 dir, double dist) {
    double len = dir.len();
    double x0 = p.x + dir.x / len * dist,
        y0 = p.y + dir.y / len * dist,
        z0 = p.z + dir.z / len * dist;
    return Point3(x0, y0, z0);
}
\end{minted}

\subsection{三维空间点到线段的距离}
\begin{minted}{c++}
// 点到线段的距离
double distToSegment(Point3 p, Point3 a, Point3 b) {
    if (a == b)
        return (p-a).len();
    Vector3 v1 = b - a, v2 = p - a, v3 = p - b;
    if (dcmp(v1 * v2) < 0)
        return v2.len();
    else if (dcmp(v1 * v3) > 0)
        return v3.len();
    else return ((v1 ^ v2) / v1.len()).len();
}    

// 点到直线的距离
double distToLine(Point3 p, Point3 a, Point3 b) {
    Vector3 v1 = b - a, v2 = p - a;
    return (v1 ^ v2).len() / v1.len();
}
\end{minted}

\subsection{计算四面体的体积}

$V = \frac{1}{3}S\cdot h = \frac{1}{6}((\vec{AB} \times \vec{AC}) \cdot \vec{AD})$

\begin{minted}{c++}
double volume(Point3 a, Point3 b, Point3 c, Point3 d){
    return 1.0 / 6 * (d - a) * ((b - a) ^ (c - a));
}
\end{minted}