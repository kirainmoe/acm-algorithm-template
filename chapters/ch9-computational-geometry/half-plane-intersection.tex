\section{半平面交}
\paragraph{半平面} 半平面指一条直线的一侧的点构成的点集,一般包含直线上的点。数学表示即 $Ax+By+C \ge(\le) 0$ 或 $y \ge kx + b$.

\subsection{半平面的表示}
使用有向的方向向量和直线上一点的参数表示法表示直线,规定方向向量的左侧即为半平面。

\subsection{半平面交}
求形如 $n$ 条 $y \ge kx + b$ 的半平面的并,类似线性规划。答案应该是形如凸包的下凸壳。如果是 $t \le kx +b$ 的半平面,则求的就是上凸壳。

\noindent 算法1:如果半平面的交不封闭,只需要将 $y=kx+b$ 对偶成点 $(k, -b)$,然后水平排序求这些点的下凸壳。

\noindent 算法2:增量法。如果半平面的交封闭,则维护一个双端队列,将边按照极角排序,用求凸包的方法来求。步骤如下:

\begin{itemize}
    \item 对目标向量进行极角排序(如果排序时遇到共线向量,则取靠近可行域的一个)
    \item 维护一个凸壳,使用双端单调队列维护。后来加入的向量只会影响最开始加入的或最后加入的边。
    \item 维护一个交点数组。当单调队列中的元素超过 2 个的时候,就会产生一个交点。对于当前向量,如果上一个交点在这条向量表示的半平面交的异侧,则上一条边就没有意义。
    \item 后加入的边也可能会影响队首的边。
    \item 如果半平面交是一个凸 $n$ 边形,则最后在交点数组里会得到 $n$ 个点。注意判断半平面交不存在或面积为 0 的情况。
\end{itemize}

\begin{minted}{c++}
/* 半平面交:封闭 */
bool checkPointOnLeft(Line l, Point P) {
    return dcmp(cross(l.s, P - l.p)) > 0;
}
// l[]: 多个半平面的边集,poly[]: 结果数组 
int halfPlaneIntersection(Line l[], Point poly[], int n) {
    // step 1: sort lines with angle 
    std::sort(l, l + n);
    // step 2: create a deque
    Point *p = new Point[n];    // p[i] is the intersection of q[i] and q[i+1]
    Line *q = new Line[n];
    int first, last;
    // at the beginning there is only a half plane l[0]
    q[first = last = 0] = l[0];
    // step 3: for each vector
    // check the position of intersection of current vector & last/first vector in the hell
    for (int i = 1; i < n; i++) {
        while (first < last && !checkPointOnLeft(l[i], p[last - 1]))
            last--;
        while (first < last && !checkPointOnLeft(l[i], p[first]))
            first++;
        q[++last] = l[i];       // add current vector
        // step 4: if vectors are parallel, preserve the inner one
        if (fabs(cross(q[last].s, q[last - 1].s)) < eps) {
            last--;
            if (checkPointOnLeft(q[last], l[i].p))
                q[last] = l[i];
        }
        // step 5: get the intersection of last 2 vectors(if exists)
        if (first < last)
            p[last - 1] = getLineIntersection(q[last - 1], q[last]);
    }
    // step 6: delete the useless plane
    while (first < last && !checkPointOnLeft(q[first], p[last - 1]))
        last--;
    // hale plane intersection not exists
    if (last - first <= 1)
        return 0;
    // the intersection of first and last intersection
    p[last] = getLineIntersection(q[last], q[first]);
    // copy to dist array
    int m = 0;
    for (int i = first; i <= last; i++)
        poly[m++] = p[i];
    return m;
}
\end{minted}
