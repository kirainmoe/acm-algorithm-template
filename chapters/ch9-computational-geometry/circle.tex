\section{圆}
\subsection{定义}
\begin{minted}{c++}
struct Circle {
    Point c;    // center
    double r;   // radius
    Circle(Point c, double r): c(c), r(r) {}
    // 通过圆心角 rad 求圆上一点的坐标 
    Point pt(double rad) {
        return Point(c.x + cos(rad) * r, c.y + sin(rad) * r);
    }
};
\end{minted}

\subsection{直线 AB 与圆 C 的交点}

\begin{itemize}
    \item 方法1:解方程组:设交点 $P=A+t(B-A)$,代入圆方程得到 $(at+b)^2+(ct+d)^2=r^2$,整理得到二次方程 $et^2+ft+g=0$。根据判别式区分相离、相切、相交的情况并得到交点。
    \item 方法2:先求出圆心 $C$ 在 $AB$ 上的投影点 $P$,再求 $\vec{AB}$ 对应的单位向量 $\vec{v}$,两个交点分别为 $p-L\vec{v}$ 和 $p+L\vec{v}$,$L$ 为 $P$ 到交点的距离,可以由勾股定理算出。
\end{itemize}

\noindent 注:第二种方法可以先求圆心到直线距离判断交点个数,再用勾股定理 $L^2=r^2-d^2$ 算出 $L$.

\begin{minted}{c++}
int getLineCircleIntersection(Line l, Circle C, Point ans[]) {
    double a = l.s.x, b = l.p.x - C.c.x, c = l.s.y, d = l.p.y - C.c.y;
    double e = a * a + c * c, f = 2 * (a * b + c * d), g = b * b - d * d - C.r * C.r;
    double delta = f * f - 4 * e * g;
    if (dcmp(delta) < 0)        // 0
        return 0;
    if (dcmp(delta) == 0) {     // 1
        double t1 = - f / (2 * e);
        ans[0] = C.pt(t1);
        return 1;
    }
    double t1 = (-f - sqrt(delta)) / (2 * e),
        t2 = (-f + sqrt(delta)) / (2 * e);      // 2
    ans[0] = C.pt(t1), ans[1] = C.pt(t2);
    return 2;
}
\end{minted}

\begin{minted}{c++}
int getLineCircleIntersection2(Line l, Circle C, Point ans[]) {
    Point proj = l.p + l.s * (dot(l.s, C.c - l.p) / dot(l.s, l.s));
    Vector normal = normalVector(l.s);
    double dist = square(proj.x - C.c.x) + square(proj.y - C.c.y);
    if (dist > C.r)
        return 0;
    if (dist == C.r) {
        ans[0] = proj;
        return 1;
    }
    double len = sqrt(square(C.r) - dist);
    ans[0] = proj + normal * len, ans[1] = proj - normal * len;
    return 2;
}
\end{minted}

\subsection{两圆相交}
\noindent 若两圆有交点,则设其中一个交点为 $P_1$,圆心距为 $d$,根据余弦定理算出 $C_1C_2$ 到 $C_1P_1$ 的角 $da$ 和 $C_1C_2$ 的极角 $a$,$a \pm da$ 角对应的点即为两个交点(或重合为一个交点)。
\begin{minted}{c++}
int getCirclesIntersection(Circle C1, Circle C2, Point ans[]) {
    double dist = length(C1.c - C2.c);
    if (C1.c == C2.c && dcmp(C1.r - C2.r) == 0) 
        return -1;      // 两圆重合   
    if (dcmp(dist) == 0 || (dcmp(C1.r + C2.r - dist < 0)) || (dcmp(fabs(C1.r - C2.r) - dist) > 0))
        return 0;       // 内离,外离
    Vector d = C2.c - C1.c;
    double ang = atan2(d.y, d.x),
        da = acos((C1.r * C1.r + dist * dist - C2.r * C2.r) / (2 * C1.r * dist));
    ans[0] = C1.pt(ang - da), ans[1] = C1.pt(ang +da);
    if (ans[0] == ans[1])       // 重合 
        return 1;
    else
        return 2;
}
\end{minted}

\subsection{过定点到圆的切线}
\noindent 首先计算点到圆心的距离判断是否存在切线和条数。如果点在圆上只有一条切线,切线即为点与圆心连线向量旋转 90 度。否则用反正弦函数可以求出切角 $ang$,将点与圆心连线分别顺时针和逆时针旋转 $ang$ 即可得到两条切线。
\begin{minted}{c++}
int getTangents(Point p, Circle C, Vector v[]) {
    Vector u = C.c - p; 
    double dist = length(u);
    if (dist < C.r)
        return 0;       // 点在圆内,没有切线
    else if (dcmp(dist - C.r) == 0) {
        v[0] = rotateVector(u, acos(-1) / 2.0);
        return 1;
    } else {
        double ang = asin(C.r / dist);
        v[0] = rotateVector(u, ang), v[1] = rotateVector(u, -ang);
        return 2;
    }
}
\end{minted}

\subsection{两圆的公切线}
\noindent 两圆的公切线有六种情况:
\begin{itemize}
    \item 两圆完全重合,有无数条公切线。
    \item 两圆内含,没有公切线。
    \item 两圆内切,有一条公切线。
    \item 两圆相交,有两条公切线。
    \item 两圆外切,有三条公切线。
    \item 两圆相离,有四条公切线。
\end{itemize}

\begin{minted}{c++}
int getCirclesTangents(Circle A, Circle B, Point *a, Point *b)  {
    int cnt = 0;
    if (A.r < B.r) {
        std::swap(a, b);
        std::swap(A, B);
    }
    double d2 = lengthSquare(A.c - B.c),
        rdiff = A.r - B.r, rsum = A.r + B.r;
    if (d2 < rdiff * rdiff) {
        return 0;       // 内含 
    }
    double base = atan2(B.c.y - A.c.y, B.c.x - A.c.x);
    if (d2 == 0 && dcmp(A.r - B.r) == 0)
        return -1;      // 重合
    if (d2 == rdiff * rdiff)  {
        a[cnt] = A.pt(base), b[cnt++] = B.pt(base);
        return cnt;
    }
    // 相交的基本情况 
    double ang = acos((A.r - B.r) / sqrt(d2));
    a[cnt] = A.pt(base + ang), b[cnt++] = B.pt(base + ang);
    a[cnt] = A.pt(base - ang), b[cnt++] = B.pt(base - ang);
    if (d2 == rsum * rsum) {    // 外切 
        a[cnt] = A.pt(base), b[cnt++] = B.pt(base);
    }
    else if (d2 > rsum * rsum) {    // 外离 
        ang = acos((A.r + B.r) / sqrt(d2));
        a[cnt] = A.pt(base + ang), b[cnt++] = B.pt(base + ang);
        a[cnt] = A.pt(base - ang), b[cnt++] = B.pt(base - ang);
    }
    return cnt;
}
\end{minted}

\subsection{两点/三点定圆}
\begin{minted}{c++}
Circle getCircle(Point a, Point b) {
    Point c = Point((a.x + b.x) / 2.0, (a.y + b.y) / 2.0);
    double r = length(Vector(b - a)) / 2.0;
    return (Circle) { c, r };
}

Circle threePointCircle(Point a, Point b, Point c) {
    Point p = (a + b) / 2.0,
        q = (a + c) / 2.0;
    Vector v = rotate(b - a, PI / 2.0),
        w = rotate(c - a, PI / 2.0);
    Point o;
    double r;
    if (dcmp(v ^ w) == 0) {        // parallel
        if (dcmp(length(a - b) + length(b - c) -length(a - c)) == 0)
            o = (a + c) / 2.0, r = length(c - o);
        if (dcmp(length(a - b) + length(a - c) - length(b - c)) == 0)
            o = (b + c) / 2.0, r = length(b - o);
        if (dcmp(length(a - c) + length(b - c) - length(a - b)) == 0)
            o = (a + b) / 2.0, r = length(a - o);
        return (Circle){ o, r };
    }
    o = lineIntersection(p, v, q, w);
    r = length(a - o);
    return (Circle) { o, r };
}
\end{minted}

\subsection{最小覆盖圆}
\begin{minted}{c++}
Circle MinCoveredCircle(Point p[], int n) {
    random_shuffle(p, p + n);
    Circle c = (Circle) {p[0], 0.0};

    for (rint i = 1; i < n; i++) {
        if (checkPointInCircle(p[i], c))
            continue;
        c.c = p[i], c.r = 0.0;
        for (rint j = 0; j < i; j++) {
            if (checkPointInCircle(p[j], c))
                continue;
            c = getCircle(p[i], p[j]);
            for (rint k = 0; k < j; k++) {
                if (checkPointInCircle(p[k], c))
                    continue;
                c = threePointCircle(p[i], p[j], p[k]);
            }
        }
    }
    return c;
}
\end{minted}


\subsection{最小覆盖球的模拟退火算法}
\begin{minted}{c++}
double solve() {
    const double start = 1000;
    double step = start, ans = inf, mt;
    Point z;
    z.x = z.y = z.z = 0;
    int s = 0;
    while (step > eps) {
        for (int i = 0; i < n; i++)
            if (dist(z, pts[s]) < dist(z, pts[i]))
                s = i;
        mt = dist(z, pts[s]);
        ans = min(ans, mt);
        z.x += (pts[s].x - z.x) / start * step;
        z.y += (pts[s].y - z.y) / start * step;
        z.z += (pts[s].z - z.z) / start * step;
        step *= 0.97; 
    }
    return ans;
}
\end{minted}