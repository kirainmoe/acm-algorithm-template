\section{旋转卡壳}
\paragraph{对踵点对} 对于点 $P_i, P_j$,如果存在两条分别穿过这两个点的平行直线,把凸包夹在中间(即 $P_i, P_j$ 与凸包相切),则这样的点对成为对踵点对。对踵点对最多只能有四对。

\noindent 用两条平行且与已知凸包相切的直线在凸包上进行扫描,用于解决一些问题,如:凸多边形直径(凸多边形上或平面距离最远的两点);凸多边形之间的最大、最小距离;最小面积/周长外接矩形。旋转卡壳具体思想如下:
\begin{itemize}
    \item 使用两条有向直线将凸包夹在中间,上面的水平向左,下面的水平向右;
    \item 初始的时候找到 $y$ 坐标最小和最大的两个点 $P_i$ 和 $P_j$.
    \item 逆时针旋转两条直线,假设穿过 $P_i, P_j$ 的有向直线分别需要逆时针旋转 $\theta_i, \theta_j$角度才能贴边 $P_iP_{i+1}, P_jP_{j+1}$.
    \item 如果 $\theta_i < \theta_j$,则当旋转角为 $\theta_i$时,对踵点对中 $P_i$ 变为 $P_{i+1}$;同理$\theta_j > \theta_i$,旋转角 $\theta_j$,$P_j$ 变成 $P_{j+1}$.
    \item 如果 $\theta_i = \theta_j$,两条直线同时贴住新的边,$P_i P_j, P_{i+1} P_{j+1}$ 分别为对踵点对.
    \item 重复上述过程,直到穿过 $P_i$ 的直线倾斜角大于 $\pi$ 时终止。
\end{itemize}

\noindent 实际上,判断最远点对的时候如何得到距离每条对应边的的最远点呢?可以把点到直线的距离化解为三角形的面积,再运用叉积进行比较。同时,凸包上的点依次与对应边产生的距离成单峰函数,面积上升到最高点后,又会下降。利用单峰函数这一性质,我们可以根据凸包上点的顺序,枚举对踵点,直到下一个点的距离小于当前点就可以停止了,而且随着对应边的旋转,最远点也只会顺着这个方向旋转,我们可以从上一次的对踵点开始继续寻找这一次的。

\noindent 但是下面的模板不具有普遍适用性。

\begin{minted}{c++}
double rotatingCalipers(Point p[], int n) {
    double res = 0.0;
    if (n == 2)
        return lengthSquare(p[1] - p[0]);

    p[n++] = p[0];      // put first point to the end        
    int m = 1;
    // foreach every point
    for (int i = 0; i < n; i++) {
        Vector v1 = p[i+1] - p[i], v2 = p[m] - p[i], v3 = p[m + 1] - p[i];
        // compare area
        while (dcmp(cross(v1, v2) - cross(v1, v3)) < 0) {
            m = (m + 1) % n;
            v2 = p[m] - p[i], v3 = p[m + 1] - p[i];
        }
        res = std::max(res, std::max(lengthSquare(p[i] - p[m]), lengthSquare(p[i + 1] - p[m])));
    }
    return res;
}    
\end{minted}

可以改进上面的代码,原理如下:
\begin{minted}{c++}
for (int i = 0, m = 1; i < n; i++) {
    // when area(p[i], p[i+1], p[m+1]) <= area(p[i], p[i+1], p[m]), stop rotate
    // ===> cross(p[i+1]-p[i], p[m+1]-p[i]) - cross(p[i+1]-p[i], p[m]-p[i]) <= 0
    // because cross(a, b) - cross(a, c) = cross(a, b-c)
    // thus cross(p[i+1]-p[i], p[m+1]-p[m]) <= 0
    Vector v1 = p[i+1] - p[i], v2 = p[m+1] - p[m];
    // compare area
    while (dcmp(cross(v1, v2)) > 0) {
        m = (m + 1) % n;
        v2 = p[m+1] - p[m];
    }
    res = std::max(res, std::max(lengthSquare(p[i] - p[m]), lengthSquare(p[i + 1] - p[m])));
}
\end{minted}
