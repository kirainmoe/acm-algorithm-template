\section{杜教筛和积性函数}

\subsection{积性函数}

\paragraph{定义} 对于数论函数 $f(n)$,如果对于$\forall x, y$ 满足 $(x, y) = 1$, 有 $f(xy) = f(x)f(y)$,则称 $f(n)$ 是一个积性函数。
\noindent \par 特别地,如果不需要满足 $(x, y) = 1$ 就有 $f(xy) = f(x)f(y)$ ,则 $f(n)$ 是一个完全积性函数。

\paragraph{常见积性函数}
\begin{itemize}
    \item 因数个数 $d(x) = \sum_{i|n} 1$
    \item 因数和 $\sigma(x) = \sum_{i|n} i$
    \item 欧拉函数 $\phi(x) = \sum_{i=1}^x [gcd(x, i) = 1]$
    \item 莫比乌斯函数 $\mu(x)$
    \item 单位函数 $e(n) = [n = 1]$
    \item 常数函数 $I(n( = 1)$
    \item 恒等函数 $id(n) = n$
\end{itemize}

\paragraph{性质} 若 $f(x), g(x)$ 为积性函数,则以下函数均为积性函数:
\begin{itemize}
    \item $h(x) = f(x^p)$
    \item $h(x) = f^p(x)$
    \item $h(x) = f(x)g(x)$
    \item $h(x) = \sum_{d|x} f(d)g(\frac{x}{d})$
\end{itemize}

\subsection{狄利克雷卷积}

\paragraph{定义} 两个数论函数 $f, g$ 的狄利克雷卷积定义为:

$$(f \star g)(n) = \sum_{d|n} f(d)g(\frac{n}{d})$$

\paragraph{狄利克雷卷积的性质}
\begin{itemize}
    \item 交换律:$(f \star g) = (g \star f)$
    \item 结合律:$(f \star g) \star h = f \star (g \star h)$
    \item 分配率:$f \star (g + h) = f \star g + f \star h$
    \item $f \star \epsilon = f$, $\epsilon$ 为 Dirichlet 卷积单位元
\end{itemize}

\paragraph{常见积性函数的狄利克雷卷积}
\begin{itemize}
    \item $\epsilon = \mu \star I \longleftrightarrow \epsilon(n) = \sum_{d|n}\mu(d)$
    \item $d = I \star I \longleftrightarrow d(n) = \sum_{d|n}1$
    \item $\sigma = id \star I \longleftrightarrow \sigma (n) = \sum_{d|n}d$
    \item $\phi = \mu \star id \longleftrightarrow \phi(n) = \sum_{d|n} d \cdot \mu(\frac{n}{d})$
\end{itemize}

\subsection{杜教筛}

\paragraph{原理} 对于一个积性函数 $f(x)$,求 $S(n) = \sum_{i=1}^{n} f(i)$,其中 $n$ 范围很大。考虑找到一个合适的函数 $g$ 使得 $f, g$ 的狄利克雷卷积的前缀和 $f \star g$ 容易计算,可以得到一般使用杜教筛的式子:

$$g(1)S(n) = \sum_{i=1}^{n} (f \star g)(i) - \sum_{i=2}^{n}g(i)S(\lfloor \frac{n}{i} \rfloor)$$

如果对于 $(f \star g)$ 的前缀和有 $O(1)$ 求法,那么预处理 $f$ 函数的前 $n^{\frac{2}{3}}$,后面的部分数论分块+递归求解。
\begin{minted}{c++}
// μ * I = $\epsilon$
// S(n) = 1 - $\sum_{d=2}^nS(\lfloor \frac{n}{d} \rfloor)$
// φ * I = id
// S(n) = $\sum_{i=1}^n i - \sum_{d=2}^n S(\lfloor \frac{n}{d} \rfloor)$ 

unordered_map<int, pair<int, ll> > ans;
pair<int, ll> solve(ll x) {
    if (x <= lim - 1)
        return make_pair(mu[x], phi[x]);
    if (ans.find(x) != ans.end())
        return ans[x];
        
    int mures = 1;
    ll phires = x * (x + 1) / 2;
    for (register int l = 2, r; l >= 0 && l <= x; l = r + 1) {
        r = x / (x / l);
        pair<int, ll> tmp = solve(x / l);
        mures -= (r - l + 1) * tmp.first;
        phires -= (r - l + 1) * tmp.second;
    }
    ans[x] = make_pair(mures, phires);
    return ans[x];
}
\end{minted}
