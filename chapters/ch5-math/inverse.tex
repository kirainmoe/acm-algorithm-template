\section{乘法逆元}
\subsection{exgcd 求逆元}
原理:$ax \equiv 1 (\mod{b})$, 也即 $ax - by' = 1$, 令 $y = -y'$ 得到 $ax + by = 1$,使用 exgcd 求解得到 $x$ 后对 $x$ 做模 $b$ 意义下的处理令 $x \in [0, b-1], x \in N$:$x=((x \mod{p}) + p) \mod{p}$ 即为所求。
\begin{minted}{c++}
int x, y;
void exgcd(int a, int b, int &x, int &y) {
    if (b == 0) {
        x = 1, y = 0;
        return;
    }
    exgcd(b, a % b, x, y);
    int tmp = x;
    x = y;
    y = tmp - (a / b) * y;
}
int rev(int a, int p) {
    exgcd(a, p, x, y);
    return ((x % p) + p) % p;
}
\end{minted}

\subsection{费马小定理求逆元}

\paragraph{费马小定理} 如果 $a$ 不是 $p$ 的倍数,则 $a^{p-1} \equiv 1 (mod p)$。

\begin{minted}{c++}
long long mod_pow(long long a, long long mod) {
    long long ans = 1, b = a;
    while (b) {
        if (b & 1)
            ans = ans * a % mod;
        a = a * a % mod;
        b >>= 1;
    }
    return ans;
}
long long inv(long long num, long long mod) {
    return mod_pow(num, mod - 2, mod);
}
\end{minted}

\subsection{线性预处理逆元}

\begin{minted}{c++}
void pretreat_inverse(int n) {
    inv[1] = 1;
    for(int i = 2; i < n; i++)
      inv[i] = (mod -  mod / i) * inv[mod % i] % mod;
}
\end{minted}