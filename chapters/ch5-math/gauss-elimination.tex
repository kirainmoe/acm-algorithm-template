\section{高斯消元}
\subsection{高斯消元求解方程}
\begin{minted}{c++}
double del, cc[MAXN][MAXN];    // coefficient
double ans[MAXN];
bool gauss(int n) {
    for (rint i = 1; i <= n; i++) {
        int r = i;
        for (rint j = i + 1; j <= n; j++)
            if (fabs(cc[r][i]) < fabs(cc[j][i]))
                r = j;
        if (fabs(cc[r][i]) < eps)
            return false;
        if (i != r)
            swap(cc[i], cc[r]);
        double tmp = cc[i][i];
        for (rint j = i; j <= n + 1; j++)
            cc[i][j] /= tmp;
        for (rint j = i + 1; j <= n; j++) {
            tmp = cc[j][i];
            for (rint k = i; k <= n + 1; k++)
                cc[j][k] -= cc[i][k] * tmp;
        }
    }
    ans[n] = cc[n][n + 1];
    for (rint i = n - 1; i >= 1; i--) {
        ans[i] = cc[i][n+1];
        for (rint j = i + 1; j <= n; j++)
            ans[i] -= cc[i][j] * ans[j];
    }
    return true;
}
\end{minted}

\subsection{高斯消元求矩阵行列式}
\begin{minted}{c++}
ll det(int n) {
    ll inv, tmp, ans = 1;
    for (int i = 2; i <= n; i++) {
        for (int j = i + 1; j <= n; j++)
            if (!L[i][i] && L[j][i]) {
                ans *= -1, swap(L[i], L[j]);
                break;
            }
        inv = mod_pow(L[i][i], mod - 2);
        for (int j = i + 1; j <= n; j++) {
            tmp = L[j][i] * inv % mod;
            for (int k = i; k <= n; k++)
                L[j][k] = (L[j][k] - L[i][k] * tmp % mod + mod) % mod;
        }
    }
    for (int i = 2; i <= n; i++)
        ans = ans * L[i][i] % mod;
    return ans;
}
\end{minted}
