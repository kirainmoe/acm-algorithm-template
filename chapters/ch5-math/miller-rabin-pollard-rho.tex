\section{素数判定 \& 大数质因子分解}

\begin{minted}{c++}
namespace PollardRho {
    const int test_time = 5;
    
    default_random_engine e(time(NULL));
    uniform_int_distribution<ll> u(0, 1ll << 63);
    map<ll, ll> factors;
    
    ll quick_mul(ll a, ll b, ll mod) {
        return (a * b - (ll)((ld)a / mod * b) * mod + mod) % mod;
    }
    
    ll mod_pow(ll x, ll p, ll mod) {
        ll ans = 1;
        while (p) {
            if (p & 1)
                ans = quick_mul(ans, x, mod);
            x = quick_mul(x, x, mod), p >>= 1;
        }
        return ans;
    }
    
    bool miller_rabin(ll p) {
        if (p < 2)
            return 0;
        if (p == 2 || p == 3)
            return 1;
        ll d = p - 1, r = 0;
        for (; !(d & 1); r++, d >>= 1);
        for (ll k = 0; k < test_time; k++) {
            ll a = u(e) % (p - 2) + 2, x = mod_pow(a, d, p);
            if (x == 1 || x == p-1)
                continue;
            for (int i = 0; i < r - 1; i++) {
                x = quick_mul(x, x, p);
                if (x == p - 1)
                    break;
            }
            if (x != p - 1)
                return false;
        }
        return true;
    }
    
    ll f(ll x, ll c, ll mod) {
        return (quick_mul(x, x, mod) + c) % mod;
    }
    
    ll pollard_rho(ll x) {
        ll s = 0, t = 0, c = u(e) % (x - 1) + 1, val = 1, d;
        int step = 0, goal = 1;
        for (goal = 1;; goal <<= 1, s = t, val = 1) {
            for (step = 1; step <= goal; step++) {
                t = f(t, c, x), val = quick_mul(val, (ll)abs(t - s), x);
                if ((step % 127) == 0) {
                    d = __gcd(val, x);
                    if (d > 1)
                        return d;
                }
            }
            d = __gcd(val, x);
            if (d > 1)
                return d;
        }
    }
    
    void find(ll n) {
        if (n == 1)
            return;
        if (miller_rabin(n)) {
            factors[n]++;
            return;
        }
        ll p = n;
        while (p >= n)
            p = pollard_rho(p);
        find(p), find(n / p);
    }
    
    vector<pair<ll, ll> > solve(ll x) {
        vector<pair<ll, ll> > ans;
        factors.clear(), find(x);
        for (auto fac : factors)
            ans.emplace_back(fac);
        return ans;
    }
}
\end{minted}