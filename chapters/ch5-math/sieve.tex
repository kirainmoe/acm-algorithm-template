\section{素数、欧拉函数、莫比乌斯函数}

\subsection{线性筛}
\begin{minted}{c++}
bool notPrime[MAXN];
int primes[MAXN], phi[MAXN], mu[MAXN], cnt = 0;       // phi(n) 欧拉函数

void sieve(int n) {
    notPrime[0] = notPrime[1] = 1;
    phi[1] = mu[1] = 1;
    for (int i = 2; i <= n; i++) {
        if (!notPrime[i]) {
            primes[cnt++] = i;
            phi[i] = i-1, mu[i] = -1;
        }
        for (int j = 0; j < cnt && i * primes[j] <= n; j++) {
            notPrime[i * primes[j]] = 1;
            if (i % primes[j] == 0) {
                phi[i * primes[j]] = phi[i] * primes[j];
                mu[i * primes[j]] = 0;
                break;
            } else {
                phi[i * primes[j]] = phi[i] * (primes[j] - 1);
                mu[i * primes[j]] = -mu[i];
            }
        }
    }
}
\end{minted}

\paragraph{线性筛计算因数个数} 设 $d(i)$ 表示数 $i$ 的约数个数,$num(i)$ 表示 $i$ 的最小质因数的个数 (e.g. $i = p_1^{a_1} \cdot p_2^{a_2} \cdot ... \cdot p_n^{a_n}, p_1 < p_2 < ... < p_n$, 则 $num(i) = count(p_1) = a_1$).
\begin{itemize}
\item 若 $i$ 为素数,则 $\begin{cases} d(i) = 2 \\ num(i) = count(i) = 1 \end{cases}$.
\item 若 $i \% primes[j] \neq 0$,则 $d(i \cdot primes[j]) = d(i) \cdot d(primes[j])$,因为 $i \cdot primes[j]$ 在当前被枚举前,不包含因数 $primes[j]$,所以 $d(i \cdot primes[j]) = d(i) \cdot 2$;又因为 $j$ 是从小到大枚举的,所以 $primes[j]$ 从小到大,故有 $num(i \cdot primes[j]) = 1$.
\item 若 $i \% primes[j] = 0$,则说明 $i$ 在被枚举前,已存在质因子 $primes[j]$,且 $primes[j]$ 为 $i \cdot primes[j]$ 的最小质因子,所以 $d(i \cdot primes[j]) = \frac{d(i)}{num(i) + 1} \cdot (num(i) + 2)$,$num(i \cdot primes[j]) = num(i) + 1$.
\end{itemize}

\paragraph{线性筛计算因数和} 令 $sumd(n) = (1 +p_1 + p_1^2 +... + p_1^{r_1})(1 +p_2 + p_2^2 +... + p_2^{r_2})... (1 +p_k + p_k^2 +... + p_k^{r_k})$
\par 其中 $n = p_1^{r_1}p_2^{r_2}...p_k^{r_k}$.
\begin{itemize}
\item 若 $i$ 为素数,$sund(i) = 1 + i$, $num(i) = 1 +i$.
\item 若 $i \% primes[j] \neq 0$, $\begin{cases}sund(i \cdot primes[j]) = sumd(i) \cdot sumd(primes[j]) \\ num(i \cdot primes[j]) = 1 + primes[j]\end{cases}$
\item 若 $i \% primes[j] = 0$,则 $\begin{cases} sund(i \cdot primes[j]) = (\frac{sumd(i)}{num(i)}) \cdot (num(i) \cdot primes[j]) + 1 \\ num(i \cdot primes[j]) = num(i) \cdot primes[j] + 1 \end{cases}$
\end{itemize}

\subsection{求单值的欧拉函数}

\paragraph{定义} 对正整数 $n$ 的欧拉函数是:小于等于 $n$ 的数中,与 $n$ 互质的数的数目。
\begin{minted}{c++}
long long euler(long long n) {
    long long res = n, a = n;
    for (long long i = 2; i * i <= a; i++) {
        if (a % i == 0) {
            res = res / i * (i - 1);
            while (a % i == 0)
                a /= i;
        }
    }
    if (a > 1)
        res = res / a * (a - 1);
    return res;
}
\end{minted}