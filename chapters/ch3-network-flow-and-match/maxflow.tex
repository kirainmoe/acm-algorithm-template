\section{最大流}

\subsection{Edmonds-Karp 增广路算法}

\par 算法思想:不断地使用 BFS 寻找增广路,直至网络上不存在增广路位置。BFS 时只考虑 $f(x,y) < c(x,y)$ 的边并找到一条 $S->T$ 的路径,同时计算出路径上各边的剩余容量最小值 $minf$,则网络的流量可以增加 $minf$。根据斜对称性质,该算法增广时需要考虑正向边和反向边。找到增广路之后就回溯更新增广路及其反向边的剩余容量。
\begin{minted}{c++}
namespace EdmondsKarp {
    Edge e[MAXN << 1];
    int cnt, head[MAXN], flow[MAXN], prev[MAXN], s, t, maxflow;
    bool vis[MAXN];
  
    void init(int ts, int tt) {
        memset(head, 0, sizeof head);
        memset(flow, 0, sizeof flow);
        memset(prev, 0, sizeof prev);
        cnt = 2, s = ts, t = tt, maxflow = 0;
    }

    void add_edge(int x, int y, int z) {
        e[cnt] = (Edge) { x, y, z, head[x] };
        head[x] = cnt++;
        e[cnt] = (Edge) { y, x, 0, head[y] };   // 容量为 0 的反向边
        head[y] = cnt++;
    }

    bool bfs() {    // 寻找增广路
        memset(vis, 0, sizeof vis);
        queue<int> q;
        q.push(s);
        vis[s] = 1;
        flow[s] = INF;
        while (q.size()) {
            int cur = q.front();
            q.pop();
            for (int i = head[cur]; i; i = e[i].next) {
                if (e[i].c) {   // 存在这条边并有剩余容量
                    int y = e[i].v;
                    if (vis[y])
                        continue;
                    flow[y] = min(flow[cur], e[i].c);
                    prev[y] = i;    // 记录前驱便于更新
                    q.push(y);
                    vis[y] = 1;

                    if (y == t) {   // 到达汇点,返回增广路存在
                        return true;
                    }
                } // if
            } // for
        } // while
        return false;       // 增广路不存在
    } // bool dfs

    void update() {    // 当增广路存在,更新增广路及其反向边的剩余容量
        int x = t, i;
        while (x != s) {
            i = prev[x], e[i].c -= flow[t];
            e[i ^ 1].c += flow[t];
            x = e[i].u;
        }
        maxflow += flow[t];
    }

    int work() {        // 主算法
        while (bfs())
            update();
        return maxflow;
    }
}
\end{minted}

\subsection{Dinic 算法 (带当前弧优化)}
\par 不断在残量图上使用 BFS 求出结点的层次,构建分层图(即给结点标注 $d[i]$);然后在分层图上 DFS 寻找增广路,回溯时实时更新剩余容量。每个点可以流向多条出边。时间复杂度为 $O(n^2m)$。
\begin{minted}{c++}
namespace Dinic {
    struct Edge {
        int v, next, c;
        Edge(int v = 0, int c = 0, int nxt = 0): v(v), next(nxt), c(c) {}
    } e[maxm << 1];
    
    int head[maxm], cur[maxm], d[maxm], q[maxm], s, t, cnt = 2, first, tail;
    void add_edge(int u, int v, int c) {
        e[cnt] = Edge(v, c, head[u]);
        head[u] = cnt++;
        e[cnt] = Edge(u, 0, head[v]);
        head[v] = cnt++;
    }
    bool bfs() {
        for (int i = 0; i <= t; i++)
            d[i] = 0, cur[i] = head[i];
        first = tail = 0;
        q[tail++] = s, d[s] = 1;
        while (first < tail) {
            int cur = q[first++];
            for (int i = head[cur]; i; i = e[i].next) {
                if (e[i].c && !d[e[i].v]) {
                    q[tail++] = e[i].v;
                    d[e[i].v] = d[cur] + 1;
                    if (e[i].v == t)
                        return true;
                }
            }
        }
        return false;
    }
    int dinic(int x, int flow) {
        if (x == t)
            return flow;
        int rest = flow, k;
        for (int i = cur[x]; i && rest; i = e[i].next) {
            cur[x] = i;
            if (e[i].c && d[e[i].v] == d[x] + 1) {
                k = dinic(e[i].v, min(rest, e[i].c));
                if(!k)
                    d[e[i].v] = 0;
                e[i].c -= k, e[i ^ 1].c += k;
                rest -= k;
            }
        }
        return flow - rest;
    }
    int work() {
        int maxflow = 0, flow = 0;
        while (bfs()) 
            while (flow = dinic(s, inf))
                maxflow += flow;
        return maxflow;
    }
}
\end{minted}

\clearpage