\section{二分图}
\subsection{二分图匹配-匈牙利算法}
\begin{minted}{c++}
int match[MAXN], vis[MAXN];
bool dfs(int x) {
    // 遍历 x 的每条出边 
    for (int i = head[x], y; i; i = e[i].next) {
        y = e[i].v;
        // 如果在当前递归 DFS 的过程中,y 没有被访问过 
        if (!vis[y]) {
            vis[y] = 1;     // 先将 y 分配给 x, 标记 y 被访问
            // 如果 y 没有被匹配,那就让它与 x 匹配
            // 否则,尝试对 y 已匹配的边匹配其他的节点,然后再让 y 与 x 匹配 
            if (!match[y] || dfs(match[y])) {
                match[y] = x;
                return true;
            }
        }
    }
    // 如果走到了这里,说明该点匹配失败 
    return false;
}
int main() {
    /* ... */
    int ans = 0;
    for (int i = 1; i <= n; i++) {
        // 重设 vis 数组,表示所有点都不在增广路中 
        memset(vis, 0, sizeof vis);
        // 尝试为每个点匹配(找增广路),如果匹配成功则 ans++ 
        if (dfs(i))
            ans++;
    }
    /* ... */
}
\end{minted}

\subsection{二分图判定-染色算法}
\begin{minted}{c++}
int vis[MAXN], flag = 1;
void dfs(int x, int cur) {
    vis[x] = cur;
    for (int i = head[x]; i; i = e[i].next) {
        if (vis[e[i].v] == 0)
            dfs(vis[e[i].v], 3 - cur);          // 注意这里的小技巧,3-1=2, 3-2=1.
        else if (vis[e[i].v] == cur) {
            flag = 0;
            return;
        }
    }
}
for (int i = 0; i < n; i++) {
    if (!vis[i] && flag)
        dfs(i, 1);
    if (!flag)
        break;
}
printf(flag ? "Yes" : "No");
\end{minted}

% \subsection{二分图匹配关键点}

\clearpage