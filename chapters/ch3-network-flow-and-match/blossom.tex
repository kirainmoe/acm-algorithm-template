\section{带花树算法}

\paragraph{问题场景} 用于解决一般图最大匹配问题。

\begin{minted}{c++}
namespace Blossom {
    int head[MAXN], match[MAXN], fa[MAXN], pre[MAXN], top[MAXN], in[MAXN];
    int q[MAXN * MAXN * 2], first, tail, lcatime = 0, ecnt = 1;
    bool odd[MAXN], vis[MAXN];
    Edge e[MAXN * MAXN * 2];
    
    void add_edge(int u, int v) {
        e[ecnt] = Edge(u, v, head[u]), head[u] = ecnt++;
    }
    
    int find(int x) {
        return fa[x] == x ? x : fa[x] = find(fa[x]);
    }
    
    int lca(int x, int y) {
        lcatime++;
        while (x)
            in[x] = lcatime, x = find(top[x]);
        x = y;
        while (in[x] != lcatime)
            x = find(top[x]);
        return x;
    }
    
    void blossom(int x, int y, int k) {
        while (find(x) != k) {
            pre[x] = y, y = match[x];
            odd[y] && (odd[y] = 0, q[tail++] = y);
            find(x) == x && (fa[x] = k);
            find(y) == y && (fa[y] = k);
            x = pre[y];
        }
    }
    
    bool augment(int s) {
        for (int i = 0; i <= n; i++)
            fa[i] = i, top[i] = pre[i] = odd[i] = vis[i] = 0;
        vis[s] = 1, first = tail = 0, q[tail++] = s;
        while (first < tail) {
            int now = q[first++];
            for (int i = head[now], v, x, y, j, k; i; i = e[i].next)  {
                v = e[i].v;
                if (!vis[v]) {
                    top[v] = now, pre[v] = now,
                    odd[v] = 1, vis[v] = 1;
                    if (!match[v]) {
                        j = v;
                        while (j)
                            x = pre[j], y = match[x], match[j] = x, match[x] = j, j = y;
                        return true;
                    }
                    vis[match[v]] = 1, top[match[v]] = v, q[tail++] = match[v];
                }
                else if (find(now) != find(v) && !odd[v])
                    k = lca(now, v), blossom(now, v, k), blossom(v, now, k);
            }
        }
        return false;
    }
    
    int solve() {
        int ans = 0;
        for (int i = 1; i <= n; i++)
            if (!match[i])
                ans += augment(i);
        printf("%d\n", ans);
        for (int i = 1; i <= n; i++)
            printf("%d ", match[i]);
        return ans;
    }
};
\end{minted}