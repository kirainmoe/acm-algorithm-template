\section{Tarjan}

\subsection{求割边(桥)}
\begin{minted}{c++}
namespace CutEdge {
    int cnt = 0;
    vector<pair<int, int> > G;
    void tarjan(int x, int in_edge) {
        dfn[x] = low[x] = ++cnt;
        for (int i = head[x]; i; i = e[i].next) {
            int y = e[i].v;
            if (!dfn[v]) {
                tarjan(v, i), low[x] = min(low[x], low[y]);
                if (low[y] > dfn[x])
                    G.emplace_back(make_pair(e[i].u, e[i].v));
            }
            else if (i != (in_edge ^ 1))
                low[x] = min(low[x], dfn[y]);
        }
    }
    vector<pair<int, int> > solve() {
        for (int i = 1; i <= n; i++)
            if (!dfn[i])
                tarjan(i, 0);
        return G;
    }
}
\end{minted}

\subsection{求割点}
\begin{minted}{c++}
namespace CutPoint {
    int cnt = 0, root = 0;
    vector<int> P;
    void tarjan(int x) {
        dfn[x] = low[x] = ++cnt;
        int child_count = 0;
        for (int i = head[x]; i; i = e[i].next) {
            int y = e[i].v;
            if (!dfn[y]) {
                tarjan(y);
                low[x] = min(low[x], low[y]);
                if (low[y] >= dfn[x]) {
                    child_count++;
                    if (x != root || child_count > 1)
                        P.emplace_back(x);
                }
            } else
                low[x] = min(low[x], dfn[y]);
        }
    }
    vector<int> solve() {
        for (int i = 1; i <= n; i++)
            if (!dfn[i])
                root = i, tarjan(i);
        return P;
    }
}
\end{minted}

\subsection{求无向图点双连通分量 \& 缩点}
\paragraph{点双连通} 在一个无向图中,若任意两点间至少存在两条“点不重复”的路径,则说这个图是点双连通的。
\paragraph{点双连通分量} 在一个无向图中,点双连通的极大子图称为点双连通分量。
\paragraph{等价定义} 点双连通图定义等价于:任意两条边都在同一个简单环中。

\begin{minted}{c++}
namespace vDCC {
    int dfn[MAXN], low[MAXN], dcnt = 0, root = 0, ans = 0;
    bool cut[MAXN];      // 记录点 i 是不是割点
    stack<int> s;
    vector<int> dcc[MAXN];
    void tarjan(int x) {
        dfn[x] = low[x] = dcnt++;
        s.push(x);
        if (x == root && head[x] == -1) {       // 孤立点,单独成图 
            dcc[ans++].push_back(x);
            return;
        }
        for (int i = head[x]; i; i = e[i].next) {
            int y = e[i].v;
            if (!dfn[y]) {
                tarjan(y), low[x] = min(low[x], low[y]);
                if (low[y] >= dfn[x]) {     // x 是割点 
                    cut[x] = 1;
                    int tmp;
                    do {
                        tmp = s.top(), s.pop();
                        dcc[ans].push_back(tmp);
                    } while (tmp != y);
                    dcc[ans].push_back(x);
                    ans++;
                }
            } else
                low[x] = std::min(low[x], dfn[y]);
        }
    }
    int solve(int n) {
        for (int i = 1; i <= n; i++)
            if (!dfn[i])
                root = i, tarjan(i);
        return ans; // 返回点双连通分量个数
    }

    /* 缩点 */
    Edge ec[MAXN << 1];
    int hc[MAXN], cdcnt = 2, new_id[MAXN], num;
    void add_vdcc_edge(int u, int v) {
        ec[cdcnt] = (Edge) {u, v, hc[u]};
        hc[u] = cdcnt++;
    }
    void shrink() {
        num = ans;
        // 给割点编号
        for (int i = 1; i <= n; i++)
            if (cut[i])
                new_id[i] = ++num;
        for (int i = 0; i < ans; i++) {          // ans 是 v-DCC 数 -1
            for (int j = 0; j < dcc[i].size(); j++) {
                int x = dcc[i][j];
                if (cut[x])
                    add_vdcc_edge(i, new_id[x]), add_vdcc_edge(new_id[x]. i);
                else
                    c[x] = i;           // 除了割点之外,标记其它的点只属于一个 v-DCC
            }
        }
        // 节点数: ans,边数: cdcnt / 2
        // for (int i = 2; i < tc; i += 2)
        //     printf("%d %d\n", ec[i].u, ec[i].v);
    }
}    
\end{minted}

\subsection{求无向图的边双连通分量 \& 缩点}
\paragraph{边双连通} 如果任意两点至少存在两条边不重复路径,则称该图为边双连通的。
\paragraph{等价定义} 边双连通图的定义等价于:任意一条边至少在一个简单环中。
\begin{minted}{c++}
namespace eDCC {
    // 为了进行异或运算找到边,必须设定前向星 cnt 起点为 2 
    int dfn[MAXN], low[MAXN], dcnt = 0;
    int c[MAXN], dcc = 0;
    bool is_bridge[MAXN];
    void tarjan(int x, int in_edge) {
        dfn[x] = low[x] = dcnt++;
        for (int i = head[x]; i; i = e[i].next) {
            int y = e[i].v;
            if (!dfn[y]) {
                tarjan(y, i), low[x] = min(low[x], low[y]);
                if (low[y] > dfn[x])      // x 是割边 
                    is_bridge[i] = is_bridge[i ^ 1] = 1;    // 标记该边及其反向边为桥 
            } else if (i != (in_edge ^ 1))
                low[x] = std::min(low[x], dfn[y]);
        }
    }
    void dfs(int x) {
        c[x] = dcc;
        for (int i = head[x]; i; i = e[i].next) {
            int y = e[i].v;
            // 节点 y 已被访问或者 (x,y) 是桥 
            if (c[y] || is_bridge[i])
                continue;
            dfs(y);
        }
    }
    void solve(int n) {
        for (int i = 1; i <= n; i++)
            if (!dfn[i])
                tarjan(i, 0);
        for (int i = 1; i <= n; i++)
            if (!c[i])
                dcc++, dfs(i);
        printf("There are %d e-DCCs.\n", dcc);
        for (int i = 1; i <= n; i++)
            printf("node %d belongs to e-dcc %d\n" , i, c[i]);
    }

    // 缩点
    Edge ec[MAXN << 1];
    int hc[MAXN], cecnt = 2;
    void add_cut_edge(int u, int v) {
        ec[cecnt] = (Edge) {u, v, hc[u]};
        hc[u] = cecnt++;
    }
    void shrink() {
        for (int i = 2; i <= cnt; i++) {    // cnt 是原图的边数
            int x = e[i].x, y = e[i].y;
            if (c[x] == c[y])
                continue;                   // x, y 同属一个 e-DCC, 无事可做
            add_cut_edge(x, y);             // 否则将 x, y 加入新图中
        }
        // 新图有 dcc 个点, cecnt / 2 条边
        // for (int i = 2; i < cecnt; i += 2)
        //     printf("%d %d\n", ec[i].u, ec[i].v);
    }    
}    
\end{minted}

\subsection{求有向图强连通分量 \& 缩点}
\begin{minted}{c++}
namespace SCC {
    int dfn[MAXN], low[MAXN], c[MAXN];    // c[i] 表示节点 i 所属的 scc 编号 
    int scccnt = 0, order = 1;
    bool instack[MAXN];
    vector<int> scc[MAXN];
    stack<int> s;
    void tarjan(int x) {
        dfn[x] = low[x] = order++;
        s.push(x), instack[x] = 1;
        for (int i = head[x]; i; i = e[i].next) {
            int y = e[i].v;
            if (!dfn[y])
                tarjan(y), low[x] = min(low[x], low[y]);
            else if (instack[y])
                low[x] = min(low[x], dfn[y]);
        }
        if (dfn[x] == low[x]) {
            int tmp;
            do {
                tmp = s.top(), s.pop();
                c[tmp] = scccnt, instack[tmp] = 0;
                scc[scccnt].push_back(tmp);
            } while (tmp != x);
            scccnt++;
        }
    }

    // 缩点
    Edge ec[MAXN];
    int hc[MAXN], ccnt = 1;
    void add_scc_edge(int u, int v) {
        ec[ccnt] = (Edge) { u, v, hc[u] };
        hc[u] = ccnt++;
    }
    void shrink() {
        for (int x = 1; x <= n; x++) {
            for (int i = head[x]; i; i = e[i].next) {
                int y = e[i].v;
                if (c[x] == c[y])
                    continue;
                add_scc_edge(c[x], c[y]);
            }
        }    
    }
}
\end{minted}
\clearpage