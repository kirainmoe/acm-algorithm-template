
\section{欧拉回路}
\begin{minted}{c++}
stack<int> ans;
multiset<int> e[MAXN];
void dfs(int x) {
  // 遍历与 x 相连的每条无向边
  for (multiset<int>::iterator itor = e[x].begin(); itor != e[x].end(); itor = e[x].begin()) {
    // 删除边(注意是无向图,要删两次)并对下一个节点遍历
    int tmp = *itor;
    to[x].erase(itor), to[u].erase(to[u].find(x));
    dfs(tmp);
  }
  // DFS 完成后,将当前点加入队列
  ans.push(x);
}
while (!ans.empty()) {  // 倒序输出栈中的内容即为答案
  printf("%d", ans.top());
  ans.pop();
}
\end{minted}
\par 此外也有一种删边的做法是每次搜索边 $e_is$ 的时候,标记 $used[e_ij]=1$ 来表示删边,适用于用前向星存储图时的情况。

\section{哈密尔顿路径}
\begin{minted}{c++}
int map[MAXN][MAXN], ans[MAXN];
bool vis[MAXN];
void hamilton() {
  int s = 1, t, ansi = 2;     // 任选一个节点为起点
  memset(vis, 0, sizeof vis);
  for (int i = 1; i <= n; i++)
    if (map[s][i]) {          // 找到一个与起点相连的点为终点
      t = i;
      break;
    }
  ans[0] = s, ans[1] = t;
  vis[s] = 1, vis[t] = 1;        // 设置起点、终点的访问状态
  for (;;) {
    for (;;) {
      int i;
      for (i = 1; i <= n; i++)
        if (map[t][i] && !vis[i]) {
          ans[ansi++] = i;
          vis[i] = true;
          t = i;
          break;
        }
      if (i > n)
        break;
    }
    std::reverse(ans, ans + ansi);
    std::swap(s, t);
    for (;;) {
      int i;
      for (i = 1; i <= n; i++)
        if (map[t][i] && !vis[i]) {
          ans[ansi++] = i;
          vis[i] = true;
          t = i;
          break;
        }
      if (i > n)
        break;
    }
    int mid = 0;
    if (!map[s][t]) {
      for (int i = 1; i < ansi - 2; i++)
        if (map[s][ans[i +1]] && map[ans[i]][t]) {
          mid = i + 1;
          break;
        }
      std::reverse(ans + mid, ans + ansi);
      t = ans[ansi - 1];
    }
    
    if (ansi == n)
      return;
    for (int i = 1; i <= n; i++)
      if (!vis[i]) {
        int j;
        for (j = 1; j < ansi - 1; j++)
          if (map[ans[j]][i]) {
            mid = j;
            break;
          }
        if (map[ans[mid]][i]) {
          t = i, mid = j;
          break;
        }
      }
    s = ans[mid-1];
    std::reverse(ans, ans + mid);
    std::reverse(ans + mid, ans + ansi);
    ans[ansi++] = t;
    vis[t] = 1;
  }
}
\end{minted}

\clearpage