\section{树 \& 子树的重心}
\paragraph{定义} 找到一个点,其所有的子树中最大的子树节点数最少,那么这个点就是这棵树的重心,删去重心后,
生成的多棵树尽可能平衡.
\paragraph{性质}
\begin{itemize}
\item 树中所有点到某个点的距离和中,到重心的距离和是最小的;如果有两个重心,那么他们的距离和一样.
\item 把两个树通过一条边相连得到一个新的树,那么新的树的重心在连接原来两个树的重心的路径上.
\item 把一个树添加或删除一个叶子,那么它的重心最多只移动一条边的距离.
\end{itemize}
\paragraph{应用} 树的重心在树的点分治中有重要的作用,可以避免 $O(n^2)$ 的极端复杂度(从退化链的一端出发),保证NlogN
的复杂度.

\begin{minted}{c++}
int sz[MAXN], maxcnt[MAXN], rt;         // 每个结点的儿子所在子树的最大结点数, rt 为所求重心
void getCenter(int u, int fa) {
    maxcnt[u] = 0;
    for (int i = head[u]; i; i = e[i].next) {
        if (e[i].v == fa)
            continue;
        getCenter(e[i].v, u);
        maxcnt[u] = max(maxcnt[u], sz[e[i].v]);
    }
    maxcnt[u] = max(maxcnt[u]. n - sz[u]);
    if (maxcnt[u] < maxcnt[rt])
        rt = u;
}    
\end{minted}

\paragraph{原理} 利用重心的性质:把两个树通过一条边相连得到一个新的树,那么新的树的重心在连接原来两个树的重心的路径上。由于一棵树的重心可能不只有一个,因此默认保存深度更深的重心,然后在深度更深的重心与根节点之间查找另一个重心。
\paragraph{性质} $a$ 可能是以 $rt$ 为根的子树的重心,当且仅当满足 $size[rt] - size[a] > size[a]$。

\begin{minted}{c++}
// 子树大小 & 父节点 & 深度 & 以 i 为根的子树较深的重心
int size[maxn], fa[maxn], depth[maxn], c[maxn];

// 在 (a, b) 之间寻找以 rt 为根的子树重心
void update(int rt, int a, int b) {
    while (depth[a] > depth[rt] && size[rt] - size[a] > size[a])
        a = fa[a];
    while (depth[b] > depth[rt] && size[rt] - size[b] > size[b])
        b = fa[b];
    c[rt] = depth[a] > depth[b] ? a : b;
}

void dfs(int u, int p) {
    fa[u] = p, size[u] = 1, c[u] = u;
    for (int i = head[u]; i; i = e[i].next) {
        int v = e[i].v;
        if (v == p)
            continue;
        depth[v] = depth[u] + 1, dfs(v, u);
        size[u] += size[v];
        update(u, c[u], c[v]);
    }
}

void solve(int n) {
    depth[1] = 1, dfs(1, 0);
    int ans[10] = {0};
    for (int i = 1; i <= n; i++) {
        int cnt = 0;
        ans[cnt++] = c[i];
        if (fa[c[i]] && size[i] - size[c[i]] == size[c[i]])
            ans[cnt++] = fa[c[i]];
        sort(ans, ans + cnt);
        for (int j = 0; j < cnt; j++)
            printf("%d%c", ans[j], " \n"[j == cnt-1]);
    }
}
\end{minted}

\section{树的直径}
任选一个点找到最远的点,然后再从那个最远点 s 找离 s 最远的点 t,dist(s, t) 即为树的直径。
\begin{minted}{c++}
int diameter() {
    queue<pair<int, int> > q;
    q.push(make_pair(1, 0)), vis[1] = 1;
    int idx = 0, maxl = 0;
    while (!q.empty()) {
        int cur = q.front().first, len = q.front().second;
        q.pop();
        if (len > maxl)
            idx = cur, maxl = len;
        for (int i = head[cur], v; i; i = e[i].next) {
            v = e[i].v;
            if (!vis[v])
                q.push(make_pair(v, len + 1)), vis[v] = 1;
        }
    }
    memset(vis, 0, sizeof vis);
    q.push(make_pair(idx, 0)), maxl = 0, vis[idx] = 1;
    while (!q.empty()) {
        int cur = q.front().first, len = q.front().second;
        q.pop();
        maxl = max(maxl, len);
        for (int i = head[cur], v; i; i = e[i].next) {
            v = e[i].v;
            if (!vis[v])
                q.push(make_pair(v, len + 1)), vis[v] = 1;
        }
    }
    return maxl;
}
\end{minted}