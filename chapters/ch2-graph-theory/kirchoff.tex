\section{Kirchoff 矩阵树定理}

\begin{itemize}
    \item 问题场景:解决一张图(有向图、无向图)的生成树个数计数问题。
    \item 使用前提:计数的图都允许重边,但是不允许自环。
\end{itemize}

\subsection{无向图中的矩阵树定理}
\par 设 $G$ 是一个有 $n$ 个顶点的无向图,定义:
\begin{itemize}
\item 度数矩阵 $D(G)=\begin{cases} d_{ij}=deg(i), i = j \\ d_{ij} = 0, i \neq j \end{cases}$
\item 邻接矩阵 $A(G)=\begin{cases} a_{ij}=a_{ji}=c(i,j), i \neq j \\ a_{ij}=a_{ji}=0, i=j \end{cases}$,其中 $c_(i,j)$ 为点 $i$ 与 $j$ 相连的边的数量。
\item 拉普拉斯 (Laplace) 矩阵(或 Kirchoff 基尔霍夫矩阵) $L(G)=D(G)-A(G)$
\item 图 $G$ 的所有生成树个数 $t(G)$
\end{itemize}
\par 则无向图的矩阵树定理描述为:
\begin{itemize}
\item 行列式描述:对于 $\forall$ i,有 $t\left(G\right)=det\left[L\left(G,i\right)\right]$,其中 $L\left(G,i\right) $表示 Laplace 矩阵中去掉第 i 行和第 i 列后构成的子矩阵,$det\left(L\right)$ 表示取 L 的行列式。
\item 特征值描述:设 $\lambda_1,\lambda_2,\ldots,\lambda_{n-1}$ 为 $L\left(G\right)$ 的 $n-1$ 个\textbf{非零特征值},则 $t\left(G\right)=\frac{1}{n}\left(\lambda_1,\lambda_2,\ldots,\lambda_{n-1}\right)$
\item 性质:对于无向图的 Laplace 矩阵,它的 $n-1$ 阶主子式都相等。
\end{itemize}

\subsection{有向图中的矩阵树定理}
\par 设 $G$ 是一个有 $n$ 个顶点的有向图,定义:
\begin{itemize}
\item 出度矩阵 $D^{out}\left(G\right)=\begin{cases}d_{ij}=deg^{out}(i), i=j \\ d_{ij} = 0, i \neq j\end{cases}$ 
\item 入度矩阵 $D^{in}(G) = \begin{cases}d_{ij}=deg^{in}(i), i=j \\ d_{ij} = 0, i \neq j\end{cases}$
\item 邻接矩阵 $A(G) = \begin{cases}a_{ij}=a_{ji}=c(i,j) \\ a_{ij}=a_{ji}=0, i = j \end{cases}$,其中 $c(i, j)$ 为点 i 与点 j 相连的边的数量.
\item 出度 Laplace 矩阵 $L^{out}(G)=D^{out}(G)-A(G)$
\item 入度 Laplace 矩阵 $L^{in}(G)=D^{in}(G)-A(G)$
\item 图 G 所有以 r 为根的所有根向树形图(基图为树,所有边指向父亲)的个数为 $t^{root}\left(G,r\right)$
\item 图 G 所有以 r 为根的所有叶向树形图(基图为树,所有边指向儿子)的个数为 $t^{leaf}\left(G,r\right)$
\end{itemize}
\par 则有向图的矩阵树定理描述为:
\begin{itemize}
\item 根向形式:对于 $\forall k$,有 $t^{root}\left(G,k\right)=det\left[L^{out}\left(G,k\right)\right]$
\item 叶向形式:对于 $\forall k$,有 $t^{leaf}\left(G,k\right)=det\left[L^{in}\left(G,k\right)\right]$
\item 如果要统计一张有向图中所有的根向/叶向树形图,只要枚举所有的根 $k$ 并对$t^{root}\left(G,r\right)$ 或 $t^{leaf}(G,r)$ 求和即可
\end{itemize}
\par BEST定理:	设 $G$ 是有向欧拉图,那么 $G$ 的不同欧拉回路总数 $ec\left(G\right)=t^{root}\left(G,k\right)\Pi_{v\in V}\left(deg\left(v\right)-1\right)!$
对于欧拉图 $G$ 的任意两个节点 $k,k^\prime$,都有 $t^{root}\left(G,k\right)=t^{root}\left(G,k^\prime\right)$,且欧拉图G所有节点的出度和入度相等。

\begin{minted}{c++}
namespace MatrixTree {
    ll L[maxn][maxn];

    ll mod_pow(ll a, ll b) {
        ll ans = 1ll;
        while (b) {
            if (b & 1)
                ans = ans * a % mod;
            a = a * a % mod, b >>= 1;
        }
        return ans;
    }

    void add_edge(int u, int v, ll val, bool dir) {
        if (!dir)
            L[u][u] = (L[u][u] + val) % mod,
            L[v][v] = (L[v][v] + val) % mod,
            L[u][v] = (L[u][v] - val + mod) % mod,
            L[v][u] = (L[v][u] - val + mod) % mod;
        else
            L[v][v] = (L[v][v] + val + mod) % mod,
            L[u][v] = (L[u][v] - val + mod) % mod;
    }

    ll det(int n) {
        ll inv, tmp, ans = 1;
        for (int i = 2; i <= n; i++) {
            for (int j = i + 1; j <= n; j++)
                if (!L[i][i] && L[j][i]) {
                    ans *= -1, swap(L[i], L[j]);
                    break;
                }
            inv = mod_pow(L[i][i], mod - 2);
            for (int j = i + 1; j <= n; j++) {
                tmp = L[j][i] * inv % mod;
                for (int k = i; k <= n; k++)
                    L[j][k] = (L[j][k] - L[i][k] * tmp % mod + mod) % mod;
            }
        }
        for (int i = 2; i <= n; i++)
            ans = ans * L[i][i] % mod;
        return ans;
    }
}
\end{minted}
