\subsection{最小曼哈顿距离生成树}

\begin{minted}{c++}
struct P {
    int x, y, id;
    bool operator < (const P &rhs) const {
        return x != rhs.x ? x < rhs.x : y < rhs.y;
    }
}p[MAXN];

namespace ManhattanMST {
    int tot, fa[MAXN], a[MAXN], b[MAXN];
    pair<int, int> bit[MAXN];
    struct Edge {
        int u, v, w;
        bool operator < (const Edge &rhs) const {
            return w < rhs.w;
        }
    } edge[MAXN << 2];
    int getfather(int x) { return x == fa[x] ? x : (fa[x] = getfather(fa[x])); }
    int dis(P a, P b) { return abs(a.x - b.x) + abs(a.y - b.y); }
    int lowbit(int x) { return x & (-x); }
    void insert(int x, int val, int pos) {
        for(int i = x; i; i -= lowbit(i))
            if(val < bit[i].first)
                bit[i] = {val, pos};
    }
    int query(int x, int lmt) {
        int ret1 = INT_MAX, ret2 = -1;
        for(int i = x; i <= lmt; i += lowbit(i))
            if(ret1 > bit[i].first)
                ret1 = bit[i].first, ret2 = bit[i].second;
        return ret2;
    }
    int solve() {
        tot = 0;
        for(int dir = 0; dir <= 3; ++ dir) {
            if (dir & 1)
                for(int i = 1; i <= n; ++ i)
                    swap(p[i].x, p[i].y);
            else if (dir)
                for(int i = 1; i <= n; ++ i)
                    p[i].x = - p[i].x;
            sort(p + 1, n + p + 1);
            for(int i = 1; i <= n; ++ i)
                a[i] = b[i] = p[i].y - p[i].x;
            sort(b + 1, n + b + 1);
            int lmt = unique(b + 1, n + b + 1) - b - 1;
            for(int i = 1; i <= lmt; ++ i) bit[i] = {INT_MAX, -1};
            for(int i = n; i; -- i) {
                int pos = lower_bound(b + 1, lmt + b + 1, a[i]) - b;
                int ans = query(pos, lmt);
                if(ans != -1) {
                    edge[++ tot].u = p[i].id;
                    edge[tot].v = p[ans].id;
                    edge[tot].w = dis(p[i], p[ans]);
                }
                insert(pos, p[i].x + p[i].y, i);
            }
        }
        sort(edge + 1, tot + edge + 1);
        for(int i = 1; i <= n; ++ i) fa[i] = i;
        int cnt = 0, ret = 0;
        for(int i = 1; i <= tot; ++ i) {
            int u = getfather(edge[i].u);
            int v = getfather(edge[i].v);
            if(u == v) continue;
            fa[u] = v;
            ++ cnt; ret += edge[i].w;
            if(cnt == n - 1) break;
        }
        return ret;
    }
}
\end{minted}