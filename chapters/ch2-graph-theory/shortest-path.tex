\section{最短路}

\subsection{Dijkstra}
\begin{minted}{c++}
int dist[MAXN], vis[MAXN];
int dijkstra(int s, int t) {
    priority_queue<pair<int, int> > pq;
    memset(vis, 0, sizeof vis), fill(dist, dist + n, inf);
    dist[s] = 0, pq.push(make_pair(0, s));
    while (!pq.empty()) {
        int cur = pq.top().second;
        pq.pop(); 
        if (vis[cur])
            continue;
        vis[cur] = 1;
        for (int i = head[cur]; i; i = e[i].next) {
            int y = e[i].v;
            if (dist[y] > dist[cur] + e[i].w) {
                dist[y] = dist[cur] + e[i].w;
                pq.push(make_pair(-dist[y], y));
            }
        }
    }
    return dist[t];
}
\end{minted}

\subsection{SPFA 算法}
\paragraph{原始版本} 适合有负权边的图或边数较少的图。
\begin{minted}{c++}
int dist[MAXN], vis[MAXN];
int SPFA(int s, int t) {
    memset(vis, 0, sizeof vis), fill(dist, dist + n, inf);
    queue<int> Q;
    Q.push(s), vis[s] = 1, dist[s] = 0;
    while (!Q.empty()) {
        int cur = Q.front();
        Q.pop(), vis[cur] = 0;  
        for (int i = head[cur], v, w; i; i = e[i].next) {
          v = e[i].v, w = e[i].w;
          if (dist[v] > dist[cur] + w) {
            dist[v] = dist[cur] + w;
            if (!vis[v])
              Q.push(v), vis[v] = 1;
          }
        }
    }
    return dist[t];
}
\end{minted}

\paragraph{SLF 优化}:使用双端队列优化,设从 $u$ 拓展出了 $v$、队首元素 $k$,如果 $dist[v] < dist[k]$,则将 $v$ 加入队首,否则加入队尾。
\begin{minted}{c++}
int dist[MAXN], vis[MAXN];
int SPFA(int s, int t) {
    memset(vis, 0, sizeof vis), fill(dist, dist + n, inf);
    deque<int> Q;
    Q.push_back(s), vis[s] = 1, dist[s] = 0;
    while (!Q.empty()) {
        int cur = Q.front();
        Q.pop_front(), vis[cur] = 0;  
        for (int i = head[cur], v, w; i; i = e[i].next) {
            v = e[i].v, w = e[i].w;
            if (dist[v] > dist[cur] + w) {
                dist[v] = dist[cur] + w;
                if (!vis[v])
                    if (Q.size() && dist[v] >= dist[Q.front()])
                        Q.push_back(v), vis[v] = 1;
                    else
                        Q.push_front(v), vis[v] = 1;
            }
        }
    }
    return dist[t];
}    
\end{minted}

\paragraph{LLL 优化}:设队首元素为 $k$,每次松弛时加入判断,设队列中所有 $dist$ 的平均值为 $x$,若 $dist[k] > x$,则将 $k$ 移到队尾,查找下一个元素;直到使得$dist[k] \le x$ 再进行松弛。
\begin{minted}{c++}
int dist[MAXN], vis[MAXN];
int SPFA(int s, int t) {
    int u, v, num = 0;
    ll x = 0;
    deque<int> q;
    fill(dist, dist + n, inf), memset(vis, 0, sizeof vis);
    dist[s] = 0, vis[s] = 1, q.push_back(s), num++;
    while (q.size()) {
        u = q.front(), q.pop_front(), num--, x -= dist[u];
        while (num && dist[u] > x / num)
            q.push_back(u), u = q.front(), q.pop_front();
        vis[u] = 0;
        for (int i = head[u]; i; i = e[i].next) {
            v = e[i].v;
            if (dist[v] > dist[u] + e[i].w) {
                dist[v] = dist[u] + e[i].w;
                if (!vis[v]) {
                    vis[v] = 1, num++, x += dist[v];
                    if (q.size() && dist[v] < dist[q.front()])
                        q.push_front(v);
                    else
                        q.push_back(v);
                }
            }
        }
    }
    return dist[t];
}        
\end{minted}
\clearpage