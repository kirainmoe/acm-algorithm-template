\subsection{最小瓶颈路}
\paragraph{问题描述} 给定一个加权无向图,并给定无向图中两个结点 u 和 v,求 u 到 v 的一条路径,使得路径上边的最大权值最小。
\par 可以证明这个“最大权值最小”的边一定在最小生成树上。对于询问两个点 u, v 的最小瓶颈路的问题,我们对求完的最小生成树用 Tarjan 或者倍增求一遍最近公共祖先 LCA,两者到达 LCA 的路径中的最大边就是最小瓶颈路了。
\begin{minted}{c++}
int query(int x, int y) {
  int d = 0;
  if (depth[x] < depth[y])
    swap(x, y);
  for (d = 0; (1 << (d + 1)) <= depth[x]; d++);
  int ans = -1;
  for (int i = d; i >= 0; i--)
    if (depth[x] - (1 << i) >= depth[y]) {
      ans = max(ans, cost[x][i]);
      x = ancestor[x][i];
    }
  if (x == y)
    return ans;
  for (int i = d; i >= 0; i--)
    if (ancestor[x][i] > 0 && ancestor[x][i] != ancestor[y][i]) {
      ans = max(ans, max(cost[x][i], cost[y][i]));
      x = ancestor[x][i], y = ancestor[y][i];
    }
  ans = max(ans, max(cost[x][0], cost[y][0]));
  return ans;
}
\end{minted}