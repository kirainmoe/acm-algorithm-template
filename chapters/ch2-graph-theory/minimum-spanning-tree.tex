\section{最小生成树}
\subsection{Kruskal 算法}
适用于一般图。
\begin{minted}{c++}
int p[MAXN];
int find(int x) {
    return x == p[x] ? x : p[x] = find(p[x]);
}
int main() {
    // ...
    for (int i = 1; i <= n; i++)
        p[i] = i;
    sort(e, e + m);
    int cnt = 0, ans = 0;
    for (int i = 1; i <= ecnt && cnt < n; i++) {
        int px = find(e[i].u), py = find(e[i].v);
        if (px != py) {
            p[px] = py;
            ans += e[i].w;
            cnt++;
        }
    }
    if (cnt != n)
        printf("No solution\n");
    else
        printf("%d", ans);
    return 0;
}

\end{minted}

\clearpage

\subsection{堆优化的 Prim 算法}
适用于边多点少的稠密图。
\begin{minted}{c++}
struct Edge {
    int u, v, w, next;
} e[MAXM << 1];
struct Node {
    int v, w;
    bool operator < (const Node b) const {
        return w > b.w;
    }
};
priority_queue<Node> pq;
int n, m, head[MAXN], cnt = 1;
bool vis[MAXN];
void add_edge(int u, int v, int w) {
    e[cnt] = (Edge) { u, v, w, head[u] };
    head[u] = cnt++;
}

int main() {
    // ...
    vis[0] = 1;
    pq.push((Node) {1, 0});
    LL ans = 0;
    int cnt = 0;
    while (!pq.empty() && cnt <= n) {
        Node cur = pq.top();
        pq.pop();
        if (vis[cur.v])
            continue;
        vis[cur.v] = 1;
        ans += cur.w;
        cnt++;
        for (int i = head[cur.v]; i; i = e[i].next) {
            if (!vis[e[i].v])
                pq.push((Node) { e[i].v, e[i].w });
        }
    } 
    if (cnt != n) {
        printf("No solution!");
    } else {
        printf("%lld", ans);
    }
}
\end{minted}
\clearpage