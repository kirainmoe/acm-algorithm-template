\section{2-SAT 问题}

\paragraph{定义} 简单的说就是给出 $n$ 个集合,每个集合有两个元素,已知若干个 $<a, b>$ ,表示 $a$ 与 $b$ 矛盾(其中 $a$ 与 $b$ 属于不同的集合)。然后从每个集合选择一个元素,判断能否一共选 $n$ 个两两不矛盾的元素。
\paragraph{原理} 首先建图:假设两个集合 $\{a_1, b_1\}$ 和 $\{a_2, b_2\}$,如果 $a_1, b_2$ 冲突,那么连有向边 $(a_1, b_1)$ 和 $(b_2, a_2)$,然后跑一遍 Tarjan 有向图缩点,判断是否有一个集合的两个元素同时在同一个 SCC 中,如果有则无解,否则有解。
\paragraph{输出方案} Tarjan 算法求强连通分量时用了栈,求得各点所在的 SCC 编号相当于反拓扑序。对于任意集合可以表示为 $\{x, \neg x\}$;如果变量 $\neg x$ 的拓扑序在 $x$ 之后(即 $topo(\neg x) \ge topo(x)$),则取 $x$ 为真。

\subsection{写法一 (tarjan 缩点)}
\begin{minted}{c++}
namespace TwoSAT {
    #define clear(x) memset(x, 0, sizeof(x))
    int head[maxn], dfn[maxn], low[maxn], c[maxn], stk[maxn];
    int top = 0, scccnt = 0, order = 1, cnt = 1;
    bool instack[maxn];
    struct Edge {
        int u, v, next;
        Edge(int u = 0, int v = 0, int next = 0): u(u), v(v), next(next) {}
    } e[maxm];
    void add_edge(int u, int v) {
        e[cnt] = Edge(u, v, head[u]);
        head[u] = cnt++;
    }
    void init() {
        clear(dfn), clear(low), clear(c), clear(instack), clear(head);
        scccnt = 0, order = 1, cnt = 1, top = 0;
    }
    void tarjan(int x) {
        dfn[x] = low[x] = order++;
        stk[++top] = x, instack[x] = 1;
        for (int i = head[x]; i; i = e[i].next) {
            int y = e[i].v;
            if (!dfn[y])
                tarjan(y), low[x] = min(low[x], low[y]);
            else if (instack[y])
                low[x] = min(low[x], dfn[y]);
        }
        if (dfn[x] == low[x]) {
            int tmp;
            do {
                tmp = stk[top--];
                c[tmp] = scccnt, instack[tmp] = 0;
            } while (tmp != x);
            scccnt++;
        }
    }
    void shrink(int n) {
        for (int x = 0; x <= n; x++)
            if (!dfn[x])
                tarjan(x);
    }
    bool solve(int n) {
        shrink(n);
        for (int i = 0; i <= n; i += 2)
            if (c[i] == c[i+1])
                return false;
        return true;
    }
}
\end{minted}

\subsection{写法二 (暴力)}
\begin{minted}{c++}
struct Twosat {
  int n;
  vector<int> g[maxn * 2];
  bool mark[maxn * 2];
  int s[maxn * 2], c;
  bool dfs(int x) {
    if (mark[x ^ 1]) return false;
    if (mark[x]) return true;
    mark[x] = true, s[c++] = x;
    for (int i = 0; i < (int)g[x].size(); i++)
      if (!dfs(g[x][i])) return false;
    return true;
  }
  void init(int n) {
    this->n = n;
    for (int i = 0; i < n * 2; i++) g[i].clear();
    memset(mark, 0, sizeof(mark));
  }
  void add_clause(int x, int y) {  // 这个函数随题意变化
    g[x].push_back(y ^ 1);         // 选了 x 就必须选 y^1
    g[y].push_back(x ^ 1);
  }
  bool solve() {
    for (int i = 0; i < n * 2; i += 2)
      if (!mark[i] && !mark[i + 1]) {
        c = 0;
        if (!dfs(i)) {
          while (c > 0) mark[s[--c]] = false;
          if (!dfs(i + 1)) return false;
        }
      }
    return true;
  }
};
\end{minted}
\clearpage