\section{最近公共祖先 (LCA)}
\subsection{倍增法求 LCA}
\begin{minted}{c++}
int anc[MAXN][MAXLOG + 5], depth[MAXN];
bool vis[MAXN];
void dfs(int rt, int dep, int p) {
    anc[rt][0] = p, depth[rt] = dep, vis[rt] = 1;
    for (int i = 1; (1 << i) <= dep; i++)
        anc[rt][i] = anc[anc[rt][i-1]][i-1];
    for (int i = head[rt]; i; i = e[i].next)
        if (!vis[e[i].v])
            dfs(e[i].v, dep + 1, rt);
}
int lca(int a, int b) {
    if (depth[a] < depth[b])
        swap(a, b);
    int delta = depth[a] - depth[b];
    for (int i = 0; i <= MAXLOG; i++)
        if (delta & (1 << i))
            a = anc[a][i];
    if (a == b)
        return b;
    for (int i = MAXLOG; i >= 0; i--)
        if (anc[a][i] != anc[b][i])
            a = anc[a][i], b = anc[b][i];
    return anc[a][0];
}
\end{minted}

\subsection{Tarjan 求 LCA}
\begin{minted}{c++}
struct Edge {
    int v, next, res;
} e[MAXN << 1], query[MAXM << 1];
int head[MAXN], qhead[MAXN], q[MAXN], cnt = 1, qcnt = 2;
bool vis[MAXN];
void add_query(int x, int y) {
    query[qcnt] = (Edge) { y, qhead[x], 0 };
    qhead[x] = qcnt++;
}
int find(int x) {
    return x == p[x] ? x : p[x] = find(p[x]);
}
void tarjan(int t) {
    vis[t] = 1;
    for (int i = head[t]; i; i = e[i].next)
        if (!vis[e[i].v]) {
            tarjan(e[i].v);
            p[e[i].v] = t;
        }
    for (int i = qhead[t]; i; i = query[i].next)
        if (vis[query[i].v]) {
            query[i].res = find(query[i].v);
            query[i^1].res = query[i].res;
        }
}
void solve(int n, int m) {
    for (int i = 0; i < m; i++) {
        read(a), read(b);
        add_query(a, b);
        add_query(b, a);
    }
    for (int i = 0; i < n; i++)
        p[i] = i;
    tarjan(s);
    for (int i = 2; i < qcnt; i += 2) {
        printf("%d\n", query[i].res);
    }
}
\end{minted}

% todo: 树链剖分求 LCA