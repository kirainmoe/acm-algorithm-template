\section{分治}
\subsection{二分}
\begin{minted}{c++}
int x, y, flag, m, ans = -1;
int a[MAXN];
while (x < y) {
    m = x + (y - x) / 2;        // 找中点,向下取整
    if (a[m] == flag) {         // 恰好中点就是结果
        ans = m;
        break;
    } else if (a[m] > flag) {
        y = m;                  // 往左边找
    } else {
        x = m + 1;              // 往右边找,注意起点 +1
    }
}
\end{minted}

\subsection{三分法}
\noindent \par 三分查找是在二分查找分出了两个区间(左区间,右区间)的情况下,再对左区间或右区间进行一次二分,以快速确定最值。三分法要求序列是一个有凹凸性的函数。步骤如下:

\begin{itemize}
    \item 取得区间的中间值 $mid = \lfloor \frac{(left + right)}{2} \rfloor$
    \item 取右区间的中间值 $rmid = \lfloor \frac{(mid + right)}{2} \rfloor$
    \item 判断 $a[rmid]$ 和 $a[mid]$ 的关系,若 $a[mid]$ 比 $a[rmid]$ 更接近最值,舍弃右区间搜索左区间;否则搜索左区间。
\end{itemize}

\begin{minted}{c++}
// 三分求函数的最小值
int l = 0, r = 10000000, ans, ll, rr;
while (l + 2 < r) {
    ll = (l + l + r) / 3;
    rr = (l + r + r) / 3;
    if (func(ll) < func(rr))
        r = rr;
    else
        l = ll;
}
ans = l;
for (int i = l + 1; i <= r; i++)
    if (func(ans) > func(i))
        ans = i;
\end{minted}

\section{CDQ 分治求三维偏序}
\begin{minted}{c++}
struct Node {
    int a, b, c, cnt, id;
    bool operator < (const Node &t) const {
        if (a != t.a)
            return a < t.a;
        else {
            if (b != t.b)
                return b < t.b;
            else
                return c < t.c;
        }
    }
    bool operator == (const Node &t) const {
        return a == t.a && b == t.b && c == t.c;
    } 
} t[MAXN], tmp[MAXN];   // tmp 是归并用的数组
int n, k, cnt = 1;
int ans[MAXN], f[MAXN];
void cdq(int l, int r) {
    if (r - l <= 1)
        return;
    int m = (l + r) >> 1;
    cdq(l, m), cdq(m, r);
    int i = l, j = m, idx = l;
    // 统计影响并将左侧可能对右侧造成影响的节点加入树状数组
    // 如果右侧的节点不受左侧的影响,那么直接更新答案
    while (i < m && j < r) {
        if (t[i].b <= t[j].b) {
            T.add(t[i].c, t[i].cnt);
            tmp[idx++] = t[i++];
        } else {
            ans[t[j].id] += T.query(t[j].c);
            tmp[idx++] = t[j++];
        }
    }
    while (j < r) {
        ans[t[j].id] += T.query(t[j].c);
        tmp[idx++] = t[j++];
    }
    for (int x = l; x < i; x++)
        T.add(t[x].c, -t[x].cnt); 
    while (i < m)
        tmp[idx++] = t[i++];
    for (int i = l; i < r; i++)
        t[i] = tmp[i];
}
\end{minted}