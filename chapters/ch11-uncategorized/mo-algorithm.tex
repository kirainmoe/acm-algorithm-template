\section{莫队算法 (带修改)}
\begin{minted}{c++}
struct Query {
    int l, r, time, id;
} q[MAXM];              // 查询结构体
struct Mod {
    int pos, color;
} mods[MAXN];           // 修改内容结构体
int n, m, a[MAXN], block[MAXN],     // 分块
    res[MAXN], cnt[MAXL],     // res 保存询问的回答
    qcnt, mcnt, ans, time;    // time:当前时间

bool cmp(const Query a, const Query b) {
    return (block[a.l] ^ block[b.l])
        ? (block[a.l] < block[b.l])
        : (block[a.r] ^ block[b.r])
        ? (block[a.r] < block[b.r])
        : a.time < b.time;
    // 不带修改的写法
    /* return (block[a.l] ^ block[a.l])
        ? block[a.l] < block[b.l]
        : ((block[a.l] & 1) ? a.r < b.r : a.r > b.r); */    
}
void add(int pos) {
    if (cnt[a[pos]] == 0)
        ans++;
    cnt[a[pos]]++;
}
void del(int pos) {
    cnt[a[pos]]--;
    if (cnt[a[pos]] == 0)
        ans--;
}
void work(int t, int i) {
    if (mods[t].pos >= q[i].l && mods[t].pos <= q[i].r) {
        cnt[a[mods[t].pos]]--;
        if (cnt[a[mods[t].pos]] == 0)
            ans--;
        cnt[mods[t].color]++;
        if (cnt[mods[t].color] == 1)
            ans++;
    }
    swap(mods[t].color, a[mods[t].pos]);
}

// main
/* 第一步,计算块大小,读入数据并分块;略去添加查询 */
int blockSize = pow(n, 2.0/3.0),
    blockCnt = ceil((double)n / blockSize);
for (int i = 1; i <= n; i++) {
    read(a[i]);
    block[i] = (i - 1) / blockSize + 1;
}
// 如果莫队是带修改的,那么查询的时间轴为当前的 mcnt
/* 第二步,对所有的查询询问排序 */
sort(q + 1, q + qcnt + 1, cmp);
/* 第三步,处理所有询问,对两个指针进行移动直到移到查询区间为止。注意 l,r 的初值 */
int l = 1, r = 0;
for (int i = 1; i <= qcnt; i++) {
    while (l < q[i].l)
        del(l++);
    while (l > q[i].l)
        add(--l);
    while (r < q[i].r)
        add(++r);
    while (r > q[i].r)
        del(r--);
    // 如果当前时间不在查询时间范围内,那就要处理修改
    while (time < q[i].time)
        work(++time, i);
    while (time > q[i].time)
        work(time--, i);
    res[q[i].id] = ans;     // 最后将答案离线到答案数组中
}
\end{minted}