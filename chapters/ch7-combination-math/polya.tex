\section{Polya 定理}

设 $G$ 是 $p$ 个对象的一个置换群,用 $m$ 种颜色涂染 $p$ 个对象,则不同的染色方案为 $L = \frac{1}{G}(m^{c(g_1)} + m^{c(g_2)} + ... +m^{c(g_n)})$,其中 $G = \{g_1, g_2, ..., g_s\}$,$c(g_i)$ 为置换 $g_i$ 的循环节数目。

\par 例题:给定一个 $n$ 个点,$n$ 条边的环,有 $n$ 种颜色,给每个顶点染色,问有多少种本质不同的染色方案,答案对 $1e9+7$ 取模。注意本题的本质不同,定义为:只需要不能通过旋转与别的染色方案相同。

\begin{minted}{c++}
long long n, m, divisor[MAXN];
int cnt = 0;
void getDivisor(long long x) {
    cnt = 0;
    long long i = 1LL;
    for (; i * i < x; i++)
        if (x % i == 0LL)
            divisor[cnt++] = i, divisor[cnt++] = x / i;
    if (i * i == x)
        divisor[cnt++] = i;
}
long long euler(long long n) {
    long long res = n, a = n;
    for (long long i = 2LL; i * i <= a; i++) {
        if (a % i == 0) {
            res = res / i * (i-1);
            while (a % i == 0)
                a /= i;
        }
    }
    if (a > 1LL)
        res = res / a * (a-1);
    return res;
}
long long cg(long long x) {
    return __gcd(n, x);
}
void solve(int n) {
    getDivisor(n);
    long long ans = 0LL;
    for (int i = 0; i < cnt; i++)
        ans = (ans + euler(divisor[i]) * mod_pow(m, cg(n / divisor[i]))) % MOD;
    ans = ans * inv(n);
}
\end{minted}
