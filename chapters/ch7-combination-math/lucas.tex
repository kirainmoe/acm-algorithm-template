\section{Lucas 定理和扩展 Lucas}

\paragraph{Lucas 定理} $C_n^m \mod p = C_{\frac{n}{p}}^\frac{m}{p} \cdot C_{n \mod p}^{m \mod p} \mod p$,即 $lucas(n, m) \mod p = lucas(\frac{n}{p}, \frac{m}{p}) \cdot C_{n \mod p}^{m \mod p} \mod p$. 当 $m = 0$ 时,$lucas(n, m, p)$ 返回 1.

\paragraph{应用条件} Lucas 定理应用前提:$p$ 为质数。

\begin{minted}{c++}
void init(ll p) {
    fac[0] = 1ll;
    for (int i = 1; i <= p; i++)
        fac[i] = fac[i-1] * (ll)i % p;
}
ll C(ll n, ll m, ll p) { 
    if (m > n)
        return 0LL;
    return fac[n] * inv(fac[m] * fac[n - m], p) % p;
    /* n, m 较大的时候不能打表,直接计算 */
}
ll lucas(ll n, ll m, ll p) {
    if (m == 0)
        return 1LL;
    return C(n % p, m % p, p) * lucas(n / p, m / p, p) % p;
}
\end{minted}

\paragraph{扩展 Lucas 定理} 当 $n, m$ 较大且 $p$ 不为质数的时候,令 $p = p_1^{k_1} \cdot ... p_n^{k_n}$,列出同余方程组:

$$\begin{cases}ans \equiv c_1 (\mod p_1^{k_1}) \\ ans \equiv c_2 (\mod p_2^{k_2}) \\ ... \\ ans \equiv c_n (\mod p_n^{k_n})\end{cases}$$

其中 $c_1 ... c_n$ 是对于每一个 $C_n^m \mod p_i^{k_i}$ 求出的答案,然后使用中国剩余定理合并即可。
