\section{GNU pbds Reference}
\subsection{头文件和命名空间定义}
\begin{minted}{c++}
#include <ext/pb_ds/assoc_container.hpp>    // 基本组件
#include <ext/pb_ds/tree_policy.hpp>        // set/map/rbt/splay
#include <ext/pb_ds/hash_policy.hpp>        // hash table
#include <ext/pb_ds/priority_queue.hpp>        // 优先队列
using namespace __gnu_pbds;                    // 命名空间
\end{minted}

\subsection{Hash Table}

比 map 更快, 近似认为 $O(n)$,建议用 gp\_hash\_table.

\begin{minted}{c++}
cc_hash_table<int, int> h1;                // 拉链法哈希表
gp_hash_table<int, int> h2;                // 探测法哈希表
h2.insert(std::make_pair(key, value));
h2.erase(key);
for (gp_hash_table<int, int>::iterator = h2.begin(); itor != h2.end(); itor++) {
    cout << itor->first << " " << itor->second << endl;
}
if (h2.find(key) != h2.end())    return;
h2.clear();
cout << h2.size();
\end{minted}

\subsection{Priority Queue}
\begin{minted}{c++}
/* definition: priority_queue<value_type, cmp_func, tag> pq; */
priority_queue<pair<int, int>, greater<pair<int, int> >, pairing_heap_tag> heap;

int len = pq.size();
bool isEmpty = pq.empty();
heap::point_iterator itor = pq.push(el);    // 返回元素的迭代器
int cur = pq.top();
pq.erase(itor);
pq.modify(itor, new_value);        // 修改元素迭代器指向的元素
pq.clear();                        // 清空优先队列
pq.join(pq2);                    // 合并优先队列
\end{minted}

\noindent \par 各种堆的效率比较:
\par \includegraphics[height=3cm]{chapters/ch12-language-and-libs/image/pbds_pq.png}

\par 使用 pb\_ds 的优先队列优化 Dijkstra:如果目标点的 dist 存在优先队列中,直接修改;否则 push. 

\begin{minted}{c++}
typedef priority_queue<pair<int, int> > heap;
vector<heap::point_iterator> id;
id.reserve(MAXM);

// 松弛时
if (id[y] != 0)
    pq.modify(id[y], make_pair(-dist[y], y));
else
    id[y] = pq.push(make_pair(-dist[y], y));
\end{minted}

\subsection{Tree}
\begin{minted}{c++}
/* 各种树的封装 */
/* 定义格式为 tree<key_type, value_type, cmp_function, tree_type_tag, [tree_order_statistics_node_update]> T; */
/* 最后一个是可选的,加上会额外获得 order_of_key() 和 find_by_order(),第二个如果没有映射类型则为 null_type */
/* tree_tag 可以是 rb_tree_tag(红黑树),splay_tree_tag(伸展树) */
typedef tree<int, int, std::less<int>, rb_tree_tag> pbds_map;
pbds_map t;                                             // 大概和 map<int, int> mp 差不多 
t.insert(make_pair(key, value)); t[key] = value;        // 插入元素
for (pbds_map::iterator itor = t.begin(); itor != t.end(); itor++) {    // 遍历
    std::cout << itor->first << " " << itor->second << std::endl;
}
pbds_map::iterator ft = t.find(key);                    // 根据 key 查找元素
t.erase(key);                                           // 根据 key 删除元素
t.clear();                                              // 清空
int len = t.size();   bool isEmpty = t.empty();
pbds_map t1;
t.join(t1);     /* 合并  */       t.split(v, t1);  // key <= v 的留在 t 中,其余分到 t1 中
int lp = t.lower_bound(x) - t, up = t.upper_bound(x) - t;
pbds_map::iterator r = t.find_by_order(k);              // 找到树中第 k+1 大/小的数
int order = t.order_of_key(k);                          // 返回 key 为 k 的元素是第几大

/* 第一个参数必须为字符串类型,tag也有别的tag,但pat最快,与tree相同,node_update支持自定义 */
typedef trie<string, null_type, trie_string_access_traits<>, pat_trie_tag, trie_prefix_search_node_update> tr;
tr.insert(s);           // 插入 s 
tr.erase(s);            // 删除 s 
tr.join(b);             // 将另一棵 Trie 树 b 并入 tr 
// 遍历 Trie 树
// pair 中第一个是起始迭代器,第二个是终止迭代器,遍历过去就可以找到所有字符串了。 
pair<tr::iterator,tr::iterator> range = base.prefix_range(x);
for(tr::iterator it = range.first; it != range.second; it++)
    cout << *it << ' ' << endl;
\end{minted}


