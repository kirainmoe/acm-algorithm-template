\section{字符串处理函数}
\subsection{<cstring> 头文件函数}
\begin{table}[h]
\centering
\small
\begin{tabular}{|c|c|c|}
\hline 函数原型 & 作用 & 返回值 \\
\hline memcpy(dest, source, size) & 从 source 数组中复制内容到 dest 数组中 & *dest \\
\hline strcpy(dest, source) & 复制字符串 & *dest \\
\hline strncpy(dest, source, size) & 复制字符串并指定字符数 & *dest \\
\hline strcat(dest, source) & 将 source 接到 dest 之后 & *dest \\
\hline strncat(dest, source, size) & 将 source 的 size 个字符接到 dest 之后 & *dest \\
\hline memcmp(arr1, arr2, size) & 比较 arr1 和 arr2 前 size 个字节的内存 & 相同 0, 1<2: <0, 1>2: >0 \\
\hline strcmp(str1, str2) & 比较字符串 str1 和 str2 & 同上 \\
\hline strncmp(str1, str2, size) & 比较字符串 str1 和 str2 的前 size 个字节 & 同上 \\
\hline memchr(arr, char, size) & 在 arr 的前 size 个字节中寻找字符 char 的位置 & 指针指向目标,找不到返回 null \\
\hline strchr(str, char) & 在字符串 str 中寻找字符 char 第一次出现的位置 & 返回第一个位置的指针,同上 \\
\hline strrchr(str, char) & 在字符串 str 中寻找字符 char 最后出现的位置 & 同上 \\
\hline strcspn(str1, str2) &  字符串 str1 开头连续几个字符都不含 str2 中的字符 & 返回结果 int \\
\hline strspn(str1, str2) & 字符串 str1 开头连续有几个字符都在 str2 中 & 返回结果 int \\
\hline strpbrk(str1, str2) & 在 str1 中搜索 str2 中字符第一次出现的位置 & 同上 \\
\hline strstr(str1, str2) & 在 str1 中检索子串 str2 第一次出现的位置 & 同上 \\
\hline strtok(str1, sep) & 将 str1 中第一个 sep 中含的字符设为 0 & 返回分割后字符串的地址 \\
\hline
\end{tabular}
\caption{cstring 函数合集}
\label{tab:Margin_settings}
\end{table}

\subsection{STL string}
\begin{itemize}
    \item 迭代器:begin(), end(), rbegin(), rend(), 其中后两者是反着迭代的。
    \item size(), length(): int 返回字符串的长度, max\_size(): int 返回字符串的最大可存储长度。
    \item capacity(): int 返回当前已分配空间长度。
    \item reserve(size) 表示预分配 size 长度的空间;clear() 表示将字符串清空。
    \item empty(): bool 返回当前字符串是否为空。
    \item shrink\_to\_fit() 表示减小字符串占用的空间 capacity 来适合它的 size.
    \item append(string/char[]/...) 和 += 同效,将字符串插入到已有的 string 后面。erase(iterator) 删除指定迭代器的字符。
    \item push\_back(char) 把字符插入到以后的 string 后面;pop\_back() 起反作用。
    \item replace(from, size, str2) 将 str1 从 from 往下数的 size 个字符替换为 str2,str2 长度任意。也可以使用迭代器指定位置。 replace(from, size, str2, t1, t2)  则将选定的部分替换为 str2 的 t1~t2 字符。
    \item str1.swap(str2) 表示将 str1 与 str2 交换。
    \item c\_str() 函数将 string 转换为 char[] 数组;copy(chr[], len, firstPos) 表示从 chr[] 中拷贝字符到 string 中。
    \item find(str) 寻找子串在 str 第一次出现的位置,找不到返回 npos, 否则返回位置的下标 (size\_t)。 rfind(str) 寻找最后一次出现的位置。
    \item substr(pos, len) 从字符串 str 的第 pos 位开始向下截取 len 位形成新的字符串。
    \item find\_first\_of, find\_last\_of, find\_first\_not\_of, find\_last\_not\_of,传入 string,作用即字面意思,返回第一个/最后一个传入的 string 中任意字符出现的位置;后两者的作用则相反,返回第一个/最后一个不在 string 中的任意字符出现的位置。
    \item compare(pos, len, str, spos, n) 分别从第一个字符串的 pos 和第二个字符串 str 的 spos 开始向下比较 n 个字符,len 是从 pos 往后数第一个字符的长度。
    \item getline(cin, str) 读一整行字符串,类似 gets.
\end{itemize}