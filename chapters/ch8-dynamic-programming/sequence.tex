\section{最长公共子序列 (LCS)}
状态转移方程:
$$f(i, j) = max\begin{cases} f(i-1, j)\\ f(i, j-1) \\ f(i-1, j-1) + 1, A[i] = B[j] \end{cases}$$

\begin{minted}{c++}
for (int i = 1; i <= n; i++)
    for (int j = 1; j <= n; j++) {
        dp[i][j] = max(dp[i-1][j], dp[i][j-1]);
        if (a[i-1] == b[j-1])
            dp[i][j] = max(dp[i][j], dp[i-1][j-1] + 1);
    }
\end{minted}

\section{最长上升子序列}
\paragraph{朴素做法} $O(n^2)$:$f(i)$ 表示以 $a_i$ 结尾的 LIS 长度,则有状态转移方程 $f(i) = max\{f(j)\} + 1, 1 \le j < i$.
\paragraph{优化做法}  $O(nlogn)$
\begin{itemize}
    \item 设置一个单调栈(满足栈底到栈顶的元素单调递增)s,然后将第一个元素加入栈中。
    \item 接下来开始逐个加入数列中的元素,设当前待入栈的元素为 ai.若 ai > 栈顶元素 s[top], 则直接让 ai 入栈.
    \item 若 ai <= 栈顶元素 s[top], 则在栈中二分查找到第一个小于等于 ai 的元素的位置 pos,将 s[pos] 替换为 ai.
    \item 重复上述步骤,直至所有数都被处理完成.
    \item 此时栈中的元素个数 s.size 即为 LIS 的答案,但注意栈中元素并不是组成 LIS 的元素。
\end{itemize}
\paragraph{输出方案} 用 $lower$ 数组保存最长上升子序列的长度,设置一个 $pos2$ 数组,记录一下数组 $a$ 中每个元素在 $lower$ 数组中出现的位置;然后从数组 $a$ 的最后一个元素开始到第一个元素寻找最长上升子序列。

\begin{minted}{c++}
// 写法 1,不需要输出方案
int stk[MAXN], top = 0;
vector<int> a;
stk[top] = a[0];
for (int i = 1; i < a.size(); i++)
    if (a[i] > stk[top])
        stk[++top] = a[i];
    else {
        int pos = lower_bound(stk, stk + top + 1, a[i]) - stk;
        stk[pos] = a[i];
    }
int ans = top + 1;        // stk.size 即为答案
\end{minted}

\begin{minted}{c++}
// 写法 2,可以输出方案
int pos[maxn], answer[maxn];
void lis(int a[], int lower [], int n){
    lower[1] = a[1], pos[1] = 1;
    int index = 1;
    for(int i = 2; i <= n; i++) {
        if(a[i] >= lower[index])
            lower[++index] = a[i], pos[i] = index;
        else {
            int p = upper_bound(lower + 1,lower + index + 1,a[i]) - lower - 1;
            lower[p] = a[i], pos[i] = p;   // 记录原数组中每个元素在 lower 数组中出现的位置
        }
    }
    int maxx = inf;
    for(int i = n; i >= 1; i--) {
        if (index == 0)
            break;
        if(pos[i] == index && maxx > a[i])
            answer[index] = i, index--, maxx = a[i];
    }
}
\end{minted}

\section{最小编辑距离}
\paragraph{状态表示} $dp[i][j]$ 表示 A 串从第 0 个字符开始到第 i 个字符、和 B 串从第 0 个
字符开始到第 j 个字符,这两个字串的编辑距离(字符串的下标从1开始)。

\begin{minted}{c++}
int dp[maxn][maxn];
char a[maxn], b[maxn];
int editDistance() {
    int len1 = strlen(a + 1), len2 = strlen(b + 1);
    memset(dp, 0x3f, sizeof dp);
    for(int i = 1; i< = len1; i++)
        dp[i][0] = i;
    for(int j = 1; j <= len2; j++)
        dp[0][j] = j;
    for(int i = 1; i <= len1; i++) {
        for(int j = 1; j <= len2; j++) {
            int flag = a[i] != b[j];
            dp[i][j] = min(dp[i-1][j] + 1, min(dp[i][j-1] + 1, dp[i-1][j-1] + flag));
        }
    }
    return dp[len1][len2];
}
\end{minted}

\section{最大连续子段和}
状态转移方程:$f(i) = max\{f(i-1) + a_i, a_i\}$.