\section{数位 DP}
\paragraph{问题场景} 处理出某一区间 $[l, r]$ 范围内,满足条件的数的个数。

\paragraph{一般解法}

\begin{itemize}
    \item \textbf{数位处理}:处理不多于 $i$ 位的数中,有多少个数满足条件,用 $dp[i]$ 表示。
    \item \textbf{状态拓展}:大部分题目与数字本身有一定的关系(不能出现/必须出现特定数字),则状态数组需要多加一维/多维进行转移:$dp[i][j]$ 表示第 $i$ 位数字为 $j$ 且满足条件的数字个数。
    \item \textbf{处理区间端点}:用预处理的信息计算出 $0 \sim r$ 和 $0 \sim l-1$ 闭区间满足条件的数字个数,然后求解 $[l, r]$ 范围的答案,即为 $r$ 端的答案减去 $l-1$ 端的答案。
    \item \textbf{具体实现}:首先在第当前考虑的第 $i$ 位上固定一个数字,那么后面就可以随便填。
\end{itemize}

\paragraph{DFS 函数的参量}
\begin{itemize}
\item 基本量:数字位数 $pos$,最高位限制 $lim$
\item 判断前导 0 的标志:$lead$
\item 一般需要记录数位中的前一位(或前几位):$pre$
\item 其它用于区分状态的参量
\end{itemize}

\paragraph{最高位标记}

当给定的区间 $[0, r]$ 不同时,数位 DP 统计时的位数限制也不同,如 $r=1234$ 那么,当第一位为 1 的时候,第二位的取值范围 $[0, 2]$;当第一位为 0 的时候第二位则可以取 $[0, 9]$。为了分清这样的情况,引入 $lim$ 变量:
\begin{itemize}
    \item 若当前位 $lim = 1$ ,且已经取到了当前位可以取得的最大数字,则下一位 $lim = 1$
    \item 若当前位 $lim = 1$,但取到的数字小于可以取得的最大数字,则下一位 $lim = 0$
    \item 若当前位 $lim = 0$,则下一位 $lim = 0$
\end{itemize}

\paragraph{记忆化搜索}DFS 的时候,可以将已经搜索过的状态记录下来,那么下一次在遇到\textbf{完全相同}的状态的时候,就可以直接返回;所以 DP 数组的维度需要和 DFS 函数的参量(除了 $lim, lead$)相同。

\par \textbf{DFS 参量完全相同} 是状态相同的必要非充分条件。需要注意的是 $lim = 1$ 和 $lead = 1$ 的时候不能直接取记忆值:

\begin{minted}{c++}
typedef long long ll;
ll dp[MAXL][MAXD];

/**
* pos: 当前枚举的位
* pre: 之前的状态,例如上一位,视需要的状态不同可能有不同的保存方式
* lead: 是否有前导 0
* limit: 当前位是否有限制
*/
ll dfs(int pos, int pre, int lead, int limit) {
    // 递归边界,返回 1 表示枚举得当前数合法
    if (pos == -1)
        return 1;
    // 如果 lim 和 lead 都不为 true,满足记忆化条件则可以返回
    if (!limit && !lead && dp[pos][pre])
        return dp[pos][pre];
    ll cur = 0;
    // 枚举当前位
    int maxd = limit ? a[pos] : 9;
    for (int i = 0; i <= maxd; i++) {
        // 需要针对多种情况具体分析,例如是否有限制,是否有前导 0 的影响等等
        if (lead && !i)
            cur += dfs(pos - 1, i, true, i == maxd && limit);
        else if (lead && !i)
            cur += dfs(pos - 1, i, true, i == maxd && limit);
        else if (....)
            cur += ...
    }
    // 记忆化存下结果
    if (!limit && !lead)
        dp[pos][pre] = cur;
    return cur;
}

ll solve(ll x) {
    int pos = 0, init_state = ?;        // init_state 表示初始位之前的状态
    // 拆数
    while (x) {
        a[pos++] = x % 10;
        x /= 10;
    }
    return dfs(pos - 1, init_state, true, true);
}
\end{minted}
