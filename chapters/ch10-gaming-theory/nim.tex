\section{Nim 游戏}
\subsection{定义}
\noindent\par Nim 游戏:有两个玩家,轮流进行操作;是公平游戏,即面对同一局面两个玩家所能进行的操作是相同的;一个玩家是输掉当且仅当他无法进行操作。
\noindent\par 一般形式:给定 $n$ 堆物品,第 $i$ 堆物品有 $A_i$ 个;两名玩家分别轮流行动,每次可以任选一堆取走任意多个物品,可把一堆取完但不可以不取,取走最后一件物品者获胜。两人都采取最优策略,问先手能否必胜。

\subsection{概念和性质}
\noindent\par 必胜点和必败点的概念:
\begin{itemize}
    \item P点:必败点,换而言之,就是谁处于此位置,则在双方操作正确的情况下必败。
    \item N点:必胜点,处于此情况下,双方操作均正确的情况下必胜。
\end{itemize}
\noindent\par 必胜点和必败点的性质:
\begin{itemize}
    \item 所有终结点是 必败点 P 。(我们以此为基本前提进行推理,换句话说,我们以此为假设)
    \item 从任何必胜点N 操作,至少有一种方式可以进入必败点 P。
    \item 无论如何操作,必败点P 都只能进入 必胜点 N。
\end{itemize}

\subsection{结论}
\noindent\par 一个 Nim 游戏中的状态是必败状态当且仅当每个子游戏的异或和为0。
\noindent\par Nim 博弈先手必胜,当且仅当 $A_1 xor A_2 xor ... xor A_n \neq 0$ 
\noindent\par \textbf{SG定理:} 游戏和的SG函数等于各个游戏SG函数的Nim和。
\noindent\par \textbf{mex (minimal excludant) 运算}:施加于集合的运算,表示最小的不属于这个集合的非负整数。对于任意状态 $x$,定义 $SG(x) = mex(S)$,其中 $S$ 是 $x$ \textbf{后继状态}的 SG 函数值的集合。

\subsection{SG 函数打表模板}
\begin{minted}{c++}
// f[N]: 可改变当前状态的方式,N 为方式的种类,f[N] 要在 getSG 之前先预处理
// 比如,如果每次可以取走 1,3,4 个石子,那么 f[] = {1, 3, 4}
// SG[i]: 0~n 的 SG 函数值; S[]: 为x后继状态的集合
int f[N], SG[MAXN], S[MAXN];
void  getSG(int n) {
    int i, j;
    memset(SG, 0, sizeof SG);
    // 初始化最终状态的 SG 值,如 SG[0] = 0;
    for(i = 1; i <= n; i++){
        // 每一次都要将上一状态的后继集合重置
        memset(S, 0, sizeof S);
        // 将后继状态的 SG 函数值进行标记
        for(j = 0; f[j] <= i && j <= N; j++)
            S[SG[i - f[j]]] = 1;
        for(j = 0; ; j++)       // mex 运算
            if (!S[j]) {
                SG[i] = j;
                break;
            }
    }
}
\end{minted}