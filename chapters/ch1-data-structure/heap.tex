\section{堆}

\subsection{普通二叉堆}

\begin{minted}{c++}
class Heap {    // priority_queue
public:
    long long val[MAXN];
    int cnt, tmp, rt;
    void insert(int x) {
        val[++cnt] = x, tmp = cnt;
        while (tmp != 0) {
            int par = tmp / 2;
            if (val[par] > x)
                std::swap(val[par], val[tmp]);
            else
                break;
            tmp = par;
        }
    }
    int top() { return val[1]; }
    void pop() {
        if (cnt == 0)    // empty
            return;
        rt = 1, val[rt] = val[cnt];
        while (2 * rt < cnt) {
            int lc = rt << 1, rc = (rt << 1) + 1;
            if (rc >= cnt || val[lc] < val[rc])
                if (val[rt] > val[lc])
                    std::swap(val[rt], val[lc]), rt = lc;
                else
                    break;
            else
                if (val[rt] > val[rc])
                    std::swap(val[rt], val[rc]), rt = rc;
                else
                    break;
        }
        cnt--;
    }
    bool empty() { return cnt == 0; }
};    
\end{minted}

\subsection{左偏树(可并堆)}

\par \noindent 左偏树与配对堆一样,是一种可并堆,具有堆的性质,并且可以快速合并。左偏树每个节点的左儿子的 $dist$ 都大于等于右儿子的 $dist$.

\par \noindent 对于一棵二叉树,我们定义\textbf{外节点}为左儿子或右儿子为空的节点,定义一个外节点的 $dist$ 为 $1$ 
一个不是外节点的节点 $dist$ 为其到子树中\textbf{最近}的外节点的距离加一。空节点的 $dist$ 为 $0$。
一棵 $n$ 个节点的二叉树,根的 $dist$ 不超过 $\lceil \log (n+1) \rceil$.

\par \noindent 可并堆也可以用 pb\_ds 实现。

\begin{minted}{c++}
namespace Leftist {
    int fa[maxn], val[maxn], dis[maxn], ch[maxn][2];
    int findrt(int x) {
        while (fa[x])
            x = fa[x];
        return x;
    }
    int merge(int x, int y) {
        if (!x || !y)
            return x | y;
        if (val[x] > val[y] || (val[x] == val[y] && x > y))
            swap(x, y);
        ch[x][1] = merge(ch[x][1], y), fa[ch[x][1]] = x;
        if (dis[ch[x][0]] < dis[ch[x][1]])
            swap(ch[x][0], ch[x][1]);
        dis[x] = dis[ch[x][1]] + 1;
        return x;
    }
    void pop(int x) {
        val[x] = -1, fa[ch[x][0]] = fa[ch[x][1]] = 0;
        merge(ch[x][0], ch[x][1]);
    }
    // 合并 (x, y) 所在堆:merge(findrt(x), findrt(y))
    // 查询 x 所在堆的最小数:val[findrt(x)]
    // 插入元素:令 val[cnt] = x, fa[cnt] = 0, dist[cnt] = 0
    // 删除堆顶:pop(x)
}
\end{minted}
\clearpage
