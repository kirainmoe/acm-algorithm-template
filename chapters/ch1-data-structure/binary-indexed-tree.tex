\section{树状数组}

\par \noindent $C[i] = A[i - 2^k+1] + A[i - 2^k+2] + ... + A[i]$
\par \noindent $sum[i] = C[i] + C[i-2^{k1}] + C[(i - 2^{k1}) - 2^{k2}] + .....$

\subsection{单点修改,区间求和}

\par Tips: lowbit(x) => (x \& (-x)) 的含义:
    当 $x$ 为 $0$ 时,结果为 $0$。
    当 $x$ 为奇数时,结果为 $1$。
    当 $x$ 为偶数时,结果为 $x$ 中 $2$ 的最大次方的因子。

\begin{minted}{c++}
class BinaryIndexedTree {
public:
    int n, c[MAXN];
    int lowbit(int x) {
        return x & (-x);
    }
    void add(int i, int value) {
        while (i <= n)
            c[i] += value, i += lowbit(i);
    }
    int sum(int x) {
        int res = 0;
        while (x > 0)
            res += c[x], x -= lowbit(x);
        return res;
    }
    void scale(int c) {
        for (int i = 1; i < n; i++)
            sum[i] *= c;
    }    
};
\end{minted}

\subsection{区间修改,区间求和}

\par \noindent 引入差分序列,试树状数组支持区间修改。树状数组中维护的是差分序列,需要有一个数组维护原数据。

\begin{minted}{c++}
class IntervalBinaryIndexedTree {
public:
    int n;
    ll a[MAXN], data[MAXN];
    int lowbit(int x) {
        return x & (-x);
    }
    void add(int pos, LL addVal) {
        while (pos <= n)
            a[pos] += addVal, pos += lowbit(pos);
    }
    ll sum(int pos) {
      ll res = 0;
      while (pos > 0)
          res += a[pos], pos -= lowbit(pos);
      return res;
    }

    /* 向区间 [x,y] 加上 val, 在点 x 加上 val, y+1 加上 -val */
    void update(int x, int y, ll val) {
        add(x, val), add(y + 1, -val);
    }
    /* 单点查询时,需要加上该点的原数据 */
    ll query(int x) {
        return data[x] + sum(x);
    }
} T;
\end{minted}

\subsection{二维树状数组(矩阵求和与修改)}
\begin{minted}{c++}
class BinaryIndexedTree2D {
public:
    int a[MAXN][MAXN], a1[MAXN][MAXN], a2[MAXN][MAXN], a3[MAXN][MAXN];
    int lowbit(int x) {
        return x & (-x);
    }
    void add(int x, int y, int val) {
        for (int i = x; i <= n; i += lowbit(i))
            for (int j = y; j <= m; j += lowbit(j)) {
                a[i][j] += val;
                a1[i][j] += val * (x - 1);
                a2[i][j] += val * (y - 1);
                a3[i][j] += val * (x - 1) * (y - 1);
            }
    }
    int query(int x, int y) {
        int t1 = 0, t2 = 0, t3 = 0, t4 = 0;
        for (int i = x; i > 0; i -= lowbit(i))
            for (int j = y; j > 0; j -= lowbit(j))
                t1 += a[i][j], t2 += a1[i][j], t3 += a2[i][j], t4 += a3[i][j];
        return t1 * x * y + t4 - t2 * y - t3 * x;
    }
    /* 在矩形 $(x_1, y_1), (x_2, y_2)$ 上加上 val */
    void update_matrix(int x1, int y1, int x2, int y2, int val) {
        add(x2 + 1, y2 + 1, tmp), add(x1, y1, tmp);
        add(x2 + 1, y1, -tmp), add(x1, y2 + 1, -tmp);
    }
    /* 查询矩形 (x1, y1) (x2, y2) 所有元素的和 */
    int sum_matrix(int x1, int y1, int x2, int y2) {
        return query(x2, y2) + query(x1 - 1, y1 - 1)
            - query(x2, y1 - 1) - query(x1 - 1, y2);
    }
};
\end{minted}

\subsection{树状数组维护 MEX}
\begin{minted}{c++}
class MEX {
public:
    int n, a[maxn], b[maxn];	
    inline int lowbit(int x) {
        return x & (-x);
    }
    void init() {
        for (int i = 0; i <= n; i++)
            a[i] = 0, b[i] = 0;
    }
    void update(int x, int val) {
        if (x > n)
            return;
        b[x] += val;
        if ((b[x] == 1 && val == 1) || (b[x] == 0 && val == -1))
            while (x <= n)
                a[x] += val, x += lowbit(x);
    }
    int query(int x) {
        int ans = 0;
        while (x > 0)
            ans += a[x], x -= lowbit(x);
        return ans;
    }
    int mex() {
        int l = 1, r = n, mid, cnt, ans;
        while (l <= r) {
            mid = (l + r) >> 1;
            cnt = tree[idx[x]].query(mid);
            if (cnt == mid)
                ans = mid, l = mid + 1;
            else
                r = mid - 1;
        }
        return ans;
    }
};
\end{minted}
\clearpage
