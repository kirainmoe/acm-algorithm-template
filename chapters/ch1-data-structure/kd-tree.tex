\section{K-D Tree}

\subsection{求第 k 近点对距离}

\begin{minted}{c++}
namespace KDTree {
    int n, k;
    priority_queue<int, vector<int>, greater<int> > q;
    struct node {
        int x, y;
    } s[maxn];
    bool cmp1(node a, node b) { return a.x < b.x; }
    bool cmp2(node a, node b) { return a.y < b.y; }
    int lc[maxn], rc[maxn], L[maxn], R[maxn], D[maxn], U[maxn];
    void maintain(int x) {
        L[x] = R[x] = s[x].x;
        D[x] = U[x] = s[x].y;
        if (lc[x])
            L[x] = min(L[x], L[lc[x]]), R[x] = max(R[x], R[lc[x]]),
            D[x] = min(D[x], D[lc[x]]), U[x] = max(U[x], U[lc[x]]);
        if (rc[x])
            L[x] = min(L[x], L[rc[x]]), R[x] = max(R[x], R[rc[x]]),
            D[x] = min(D[x], D[rc[x]]), U[x] = max(U[x], U[rc[x]]);
    }
    int build(int l, int r) {
        if (l > r) return 0;
        int mid = (l + r) >> 1;
        double av1 = 0, av2 = 0, va1 = 0, va2 = 0;  // average variance
        for (int i = l; i <= r; i++) av1 += s[i].x, av2 += s[i].y;
        av1 /= (r - l + 1);
        av2 /= (r - l + 1);
        for (int i = l; i <= r; i++)
            va1 += (av1 - s[i].x) * (av1 - s[i].x),
                va2 += (av2 - s[i].y) * (av2 - s[i].y);
        if (va1 > va2)
            nth_element(s + l, s + mid, s + r + 1, cmp1);
        else
            nth_element(s + l, s + mid, s + r + 1, cmp2);
        lc[mid] = build(l, mid - 1);
        rc[mid] = build(mid + 1, r);
        maintain(mid);
        return mid;
    }
    int sq(int x) { return x * x; }
    int dist(int a, int b) {
        return max(sq(s[a].x - L[b]), sq(s[a].x - R[b])) +
               max(sq(s[a].y - D[b]), sq(s[a].y - U[b]));
    }
    void query(int l, int r, int x) {
        if (l > r) return;
        int mid = (l + r) >> 1, t = sq(s[mid].x - s[x].x) + sq(s[mid].y - s[x].y);
        if (t > q.top()) q.pop(), q.push(t);
        int distl = dist(x, lc[mid]), distr = dist(x, rc[mid]);
        if (distl > q.top() && distr > q.top()) {
            if (distl > distr) {
            query(l, mid - 1, x);
            if (distr > q.top()) query(mid + 1, r, x);
            } else {
            query(mid + 1, r, x);
            if (distl > q.top()) query(l, mid - 1, x);
            }
        } else {
            if (distl > q.top()) query(l, mid - 1, x);
            if (distr > q.top()) query(mid + 1, r, x);
        }
    }
    ll solve(int _n, int _k) {
        n = _n, k = _k * 2;
        for (int i = 1; i <= k; i++)
            q.push(0);
        for (int i = 1; i <= n; i++)
            read(s[i].x), read(s[i].y);
        build(1, n);
        for (int i = 1; i <= n; i++)
            query(1, n, i);
        return q.top();
    }
}
\end{minted}

\subsection{动态维护二维空间信息}
\begin{minted}{c++}
namespace KDTree {
    int n, op, xl, xr, yl, yr, lstans;
    struct node {
        int x, y, v;
    } s[maxn];
    bool cmp1(int a, int b) { return s[a].x < s[b].x; }
    bool cmp2(int a, int b) { return s[a].y < s[b].y; }
    double a = 0.725;
    int rt, cur, d[maxn], lc[maxn], rc[maxn], L[maxn], R[maxn], D[maxn], U[maxn],
            siz[maxn], sum[maxn];
    int g[maxn], t;
    void print(int x) {
        if (!x) return;
        print(lc[x]);
        g[++t] = x;
        print(rc[x]);
    }
    void maintain(int x) {
        siz[x] = siz[lc[x]] + siz[rc[x]] + 1;
        sum[x] = sum[lc[x]] + sum[rc[x]] + s[x].v;
        L[x] = R[x] = s[x].x;
        D[x] = U[x] = s[x].y;
        if (lc[x])
            L[x] = min(L[x], L[lc[x]]), R[x] = max(R[x], R[lc[x]]),
            D[x] = min(D[x], D[lc[x]]), U[x] = max(U[x], U[lc[x]]);
        if (rc[x])
            L[x] = min(L[x], L[rc[x]]), R[x] = max(R[x], R[rc[x]]),
            D[x] = min(D[x], D[rc[x]]), U[x] = max(U[x], U[rc[x]]);
    }
    int build(int l, int r) {
        if (l > r) return 0;
        int mid = (l + r) >> 1;
        double av1 = 0, av2 = 0, va1 = 0, va2 = 0;
        for (int i = l; i <= r; i++) av1 += s[g[i]].x, av2 += s[g[i]].y;
        av1 /= (r - l + 1);
        av2 /= (r - l + 1);
        for (int i = l; i <= r; i++)
            va1 += (av1 - s[g[i]].x) * (av1 - s[g[i]].x),
                    va2 += (av2 - s[g[i]].y) * (av2 - s[g[i]].y);
        if (va1 > va2)
            nth_element(g + l, g + mid, g + r + 1, cmp1), d[g[mid]] = 1;
        else
            nth_element(g + l, g + mid, g + r + 1, cmp2), d[g[mid]] = 2;
        lc[g[mid]] = build(l, mid - 1);
        rc[g[mid]] = build(mid + 1, r);
        maintain(g[mid]);
        return g[mid];
    }
    void rebuild(int& x) {
        t = 0;
        print(x);
        x = build(1, t);
    }
    bool bad(int x) { return a * siz[x] <= (double)max(siz[lc[x]], siz[rc[x]]); }
    void insert(int& x, int v) {
        if (!x) {
            x = v;
            maintain(x);
            return;
        }
        if (d[x] == 1) {
            if (s[v].x <= s[x].x)
                insert(lc[x], v);
            else
                insert(rc[x], v);
        } else {
            if (s[v].y <= s[x].y)
                insert(lc[x], v);
            else
                insert(rc[x], v);
        }
        maintain(x);
        if (bad(x)) rebuild(x);
    }
    int query(int x) {
        if (!x || xr < L[x] || xl > R[x] || yr < D[x] || yl > U[x]) return 0;
        if (xl <= L[x] && R[x] <= xr && yl <= D[x] && U[x] <= yr) return sum[x];
        int ret = 0;
        if (xl <= s[x].x && s[x].x <= xr && yl <= s[x].y && s[x].y <= yr)
            ret += s[x].v;
        return query(lc[x]) + query(rc[x]) + ret;
    }
    void add(int x, int y, int v) {
        cur++, s[cur].x = x, s[cur].y = y, s[cur].v = v;
        insert(rt, cur);
    }
    ll query(int _xl, int _yl, int _xr, int _yr) {
        xl = _xl, yl = _yl, xr = _xr, yr = _yr;
        return query(rt);
    }
}
\end{minted}