\section{线段树}

\subsection{单点/区间修改,区间查询}

\par 以维护区间和、区间积为例,根据情况修改 lazytag 和传递过程,还可以维护区间最值、区间异或等等:

\begin{minted}{c++}
class SegTree {
public:
    #define lson (rt << 1)          
    #define rson (rt << 1) + 1      
    long long sum[MAXN << 2], lmul[MAXN << 2], lplus[MAXN << 2], mod;
    void pushUp(int rt) {
        sum[rt] = (sum[lson] + sum[rson]) % mod;
    }
    void pushDown(int rt, int l, int r) {
        int mid = (l + r) >> 1;
        sum[lson] = (sum[lson] * lmul[rt] % mod + lplus[rt] * (mid - l + 1)) % mod;
        sum[rson] = (sum[rson] * lmul[rt] % mod + lplus[rt] * (r - mid)) % mod;
        lmul[lson] = (lmul[lson] * lmul[rt]) % mod;
        lmul[rson] = (lmul[rson] * lmul[rt]) % mod;
        lplus[lson] = (lplus[lson] * lmul[rt] % mod + lplus[rt]) % mod;
        lplus[rson] = (lplus[rson] * lmul[rt] % mod + lplus[rt]) % mod;
        lplus[rt] = 0, lmul[rt] = 1;
    }
    void build(int rt, int l, int r) {
        lmul[rt] = 1, lplus[rt] = 0;
        if (l == r) {
            read(sum[rt]);
            sum[rt] %= mod;
            return;
        }
        int mid = (l + r) >> 1;
        build(lson, l, mid);
        build(rson, mid + 1, r);
        pushUp(rt);
    }
    void multiply(int rt, int l, int r, int ul, int ur, long long val) {
        if (ur < l || ul > r)
            return;
        if (ul <= l && r <= ur) {
            sum[rt] = (sum[rt] * val) % mod;
            lmul[rt] = (lmul[rt] * val) % mod;
            lplus[rt] = (lplus[rt] * val) % mod;
            return;
        }
        pushDown(rt, l, r);
        int mid = (l + r) >> 1;
        if (ul <= mid)
            multiply(lson, l, mid, ul, ur, val);
        if (mid < ur)
            multiply(rson, mid + 1, r, ul, ur, val);
        pushUp(rt);
    }
    void add(int rt, int l, int r, int ul, int ur, long long val) {
        if (ur < l || ul > r)
            return;
        if (ul <= l && r <= ur) {
            sum[rt] = (sum[rt] + val * (r - l +  1)) % mod;
            lplus[rt] = (lplus[rt] + val) % mod;
            return;
        }
        pushDown(rt, l, r);
        int mid = (l + r) >> 1;
        if (ul <= mid)
            add(lson, l, mid, ul, ur, val);
        if (mid < ur)
            add(rson, mid + 1, r, ul, ur, val);
        pushUp(rt);
    }
    void singleAdd(int rt, int l, int r, int pos, long long val) {
        if (l == r) {
        sum[rt] += val;
        return;
        }
        pushDown(rt);
        int mid = (l + r) >> 1;
        if (pos <= mid)
        singleAdd(lson, l, mid, pos, val);
        else
        singleAdd(rson, mid + 1, r, pos, val);
        pushUp(rt);
    }
    long long query(int rt, int l, int r, int ql, int qr) {
        if (qr < l || ql > r)
            return 0LL;
        if (ql <= l && r <= qr)
            return sum[rt] % mod;
        pushDown(rt, l, r);
        int mid = (l + r) >> 1;
        return (
            query(lson, l, mid, ql, qr) + query(rson, mid + 1, r, ql, qr)
        ) % mod;
    }
} T;
\end{minted}

\subsection{二维线段树(线段树套线段树,矩阵区间最值)}
\par 维护矩阵区间最大值/最小值。
\begin{minted}{c++}
class SegTree2D {
public:
    int maxm[MAXN << 1][MAXN << 1], minm[MAXN << 1][MAXN << 1];
    void pushUp(int x, int y, int dir) {
        if (dir == 1)        // x-axis direction
            maxm[x][y] = max(maxm[x << 1][y], maxm[x << 1 | 1][y]),
            minm[x][y] = min(minm[x << 1][y], minm[x << 1 | 1][y]);
        if (dir == 2)        // y-axis direction
            maxm[x][y] = max(maxm[x][y << 1], maxm[x][y << 1 | 1]),
            minm[x][y] = min(minm[x][y << 1], minm[x][y << 1 | 1]);
    }
    void build2D(int rtx, int rty, int l, int r, int u) {
        if (l == r) {
            if (u > 0)
                maxm[rtx][rty] = minm[rtx][rty] = a[u][l];
            else
                pushUp(rtx, rty, 1);    // update 1D
        } else {
            int mid = (l + r) >> 1;
            build2D(rtx, rty << 1, l, mid, u);
            build2D(rtx, rty << 1 | 1, mid + 1, r, u);
            pushUp(rtx, rty, 2);        // update 2D
        }
    }
    void build(int rt, int u, int d) {        // u:up, d:down
        if (u == d)
            build2D(rt, 1, 1, n, u);
        else {
            int mid = (u + d) >> 1;
            build(rt << 1, u, mid), build(rt << 1 | 1, mid + 1, d);
            build2D(rt, 1, 1, n, -1);        // -1: more than one line
        }
    }
    void update2D(int rtx, int rty, int l, int r, int ql, int qr, int u, int val) {
        if (ql <= l && qr >= r) {
            if (u > 0)
                maxm[rtx][rty] = minm[rtx][rty] = val;
            else
                pushUp(rtx, rty, 1);
        } else {
            int mid = (l + r) >> 1;
            if (ql <= mid)
                update2D(rtx, rty << 1, l, mid, ql, qr, u, val);
            if (mid + 1 <= qr)
                update2D(rtx, rty << 1 | 1, mid + 1, r, ql, qr, u, val);
            pushUp(rtx, rty, 2);
        }
    }
    void update(int rt, int u, int d, int qu, int qd, int ql, int qr, int val) {
        if (u == d)
            update2D(rt, 1, 1, n, ql, qr, u, val);
        else {
            int mid = (u + d) >> 1;
            if (qu <= mid)
                update(rt << 1, u, mid, qu, qd, ql, qr, val);
            if (mid + 1 <= qd)
                update(rt << 1 | 1, mid + 1, d, qu, qd, ql, qr, val);
            update2D(rt, 1, 1, n, ql, qr, -1, val);
        }
    }
    pair<int, int> query2D(int rtx, int rty, int l, int r, int ql, int qr, int u) {
        if (ql <= l && r <= qr)
            return make_pair(maxm[rtx][rty], minm[rtx][rty]);
        else {
            int mid = (l + r) >> 1;
            pair<int, int> res = make_pair(-INF, INF), tmp;
            if (ql <= mid) {
                tmp = query2D(rtx, rty << 1, l, mid, ql, qr, u);
                res.first = max(res.first, tmp.first), res.second = min(res.second, tmp.second);
            }
            if (mid + 1 <= qr) {
                tmp = query2D(rtx, rty << 1 | 1, mid + 1, r, ql, qr, u);
                res.first = max(res.first, tmp.first), res.second = min(res.second, tmp.second);
            }
            return res;    
        }
    }
    pair<int, int> query(int rt, int u, int d, int qu, int qd, int ql, int qr) {
        if (qu <= u && d <= qd)
            return query2D(rt, 1, 1, n, ql, qr, u == d ? u : -1);
        else {
            int mid = (u + d) >> 1;
            pair<int, int> res = make_pair(-INF, INF), tmp;
            if (qu <= mid) {
                tmp = query(rt << 1, u, mid, qu, qd, ql, qr);
                res.first = max(res.first, tmp.first), res.second = min(res.second, tmp.second);
            }
            if (mid + 1 <= qd) {
                tmp = query(rt << 1 | 1, mid + 1, d, qu, qd, ql, qr);
                res.first = max(res.first, tmp.first), res.second = min(res.second, tmp.second);
            }
            return res;
        }
    }
} T;
\end{minted}

\subsection{可持久化线段树(主席树)}
\par 可持久化线段树的主要思想:保存每次插入操作的历史版本,实际上是在动态开点线段树的基础上,通过复用某些未修改的节点,创建 $n$ 棵线段树。每进行一次修改时,产生新的节点数 = 树的高度,是 $O(n\log n)$ 级别的。
\begin{minted}{c++}
/**
* 求静态区间 $k$ 大值
* 如果要求区间 $[1,r]$ 区间 $k$ 小值,那么只需要找到插入 $r$ 时的版本,通过权值线段树可以求 $k$ 大值。
* 求 $[l, r]$ 区间最小值呢?前缀和呗,用 $[1, r]$ 的信息减去 $[1, l-1]$ 的信息即可。
*/
class PersistSegTree {
private:
    static const int MAXN = 2 * 1e5 + 50,
        MAXM = 2 * 1e5 + 50,
        MAX_NLOGM = MAXN << 5;
public:
    struct Node {
        int val;
        Node *lc, *rc;
    } t[MAX_NLOGM];
    Node* root[MAXM];
    
    int cnt;
    
    PersistSegTree(): cnt(0) {}
    void pushUp(Node *x) {
        x->val = x->lc->val + x->rc->val;
    }
    Node* build(int l, int r, int a[]) {
        Node *cur = &t[++cnt];
        if (l == r) {
            cur->val = a[l];
            return cur;
        }
        int m = (l + r) >> 1;
        cur->lc = build(l, m, a), cur->rc = build(m + 1, r, a), pushUp(cur);
        return cur;
    }
    Node* update(Node *now, int l, int r, int pos, int val) {
        t[++cnt] = *now;
        Node *cur = &t[cnt];
        if (l == r) {
            cur->val += val;
            return cur;
        }
        int m = (l + r) >> 1;
        if (pos <= m)
            cur->lc = update(now->lc, l, m, pos, val);
        else
            cur->rc = update(now->rc, m + 1, r, pos, val);
        pushUp(cur);
        return cur;
    }
    int query(Node *p, Node *q, int l, int r, int k) {
        if (l == r)
            return l;
        int m = (l + r) >> 1, lcnt = p->lc->val - q->lc->val;
        if (k <= lcnt)
            return query(p->lc, q->lc, l, m, k);
        else
            return query(p->rc, q->rc, m + 1, r, k - lcnt);
    }
} T;

// 区间查询:anspos = T.query(T.root[r], T.root[l-1], 1, t, k);
\end{minted}

\subsection{动态开点线段树}
\par\noindent 有的时候,线段树需要维护的区间很大很大,但是实际用到的节点很少;那么干脆就不要开这么多的节点,用到的时候再向内存要。
\par\noindent 例如要建立一个权值线段树,但是在线操作不让离散化,值域又是 $\infty$ 级别的。即使这个区间的范围很大,但是如果询问 $q$ 比较少的话,我们只需要 $q \log inf$ 个节点,就可以办到。
\par\noindent 由于是动态开点的,左右子节点的下标代数关系就不再成立,因此节点中需要记录当前节点的左子树和右子树根节点位置。

\begin{minted}{c++}
class DynamicSegTree {
public: 
    struct Node {
        int val, lc, rc;
    } a[maxn << 2];
    int cnt = 0;

    // 修改操作,若 $rt$ 不存在,则首先动态开点;否则直接更新
    void update(int &rt, int l, int r, int pos, int val) {
        if (!rt)
            rt = ++cnt;
        // ...
    }
    // 查询操作,若 rt 不存在直接返回空值(可能是 0 或者反向最值)
    int query(int rt, int l, int r, int ql, int qr) {
        if (!rt)
            return 0;
        // ...
    }
};
\end{minted}

\subsection{Segment Tree Beats! (区间取 min)}
\begin{minted}{c++}
class SegmentTreeBeats {
private:
    static const int maxn = 1000050;
    struct Node {
        int maxv, secmaxv, maxcnt;
        ll sum;
    } t[maxn << 2];
public:
    #define lson (rt << 1)
    #define rson (rt << 1) | 1
    void pushUp(int rt) {
        t[rt].maxcnt = 0, t[rt].sum = t[lson].sum + t[rson].sum;
        t[rt].maxv = max(t[lson].maxv, t[rson].maxv);
        t[rt].secmaxv = max(t[lson].secmaxv, t[rson].secmaxv);
        if (t[lson].maxv ^ t[rson].maxv)
            t[rt].secmaxv = max(t[rt].secmaxv, min(t[lson].maxv, t[rson].maxv));
        if (t[lson].maxv == t[rt].maxv)
            t[rt].maxcnt += t[lson].maxcnt;
        if (t[rson].maxv == t[rt].maxv)
            t[rt].maxcnt += t[rson].maxcnt;
    }
    void build(int rt, int l, int r, int a[]) {
        if (l == r) {
            t[rt].sum = t[rt].maxv = a[l];
            t[rt].secmaxv = -1, t[rt].maxcnt = 1;
            return;
        }
        int mid = (l + r) >> 1;
        build(lson, l, mid, a), build(rson, mid + 1, r, a);
        pushUp(rt);
    }
    void pushMin(int rt, int v) {
        if (v >= t[rt].maxv)
            return;
        t[rt].sum += 1ll * (v - t[rt].maxv) * t[rt].maxcnt;
        t[rt].maxv = v;
    }
    void pushDown(int rt) {
        pushMin(lson, t[rt].maxv), pushMin(rson, t[rt].maxv);
    }
    void setMin(int rt, int l, int r, int ul, int ur, int val) {
        if (val >= t[rt].maxv)
            return;
        if (ul <= l && r <= ur && t[rt].secmaxv < val) {
            pushMin(rt, val);
            return;
        }
        int mid = (l + r) >> 1;
        pushDown(rt);
        if (ul <= mid)
            setMin(lson, l, mid, ul, ur, val);
        if (mid < ur)
            setMin(rson, mid + 1, r, ul, ur, val);
        pushUp(rt);
    }
    ll query(int rt, int l, int r, int ql, int qr, int op) {
        if (ql <= l && r <= qr)
            return op == 2 ? t[rt].sum : t[rt].maxv;
        int mid = (l + r) >> 1;
        pushDown(rt);
        ll ans = 0;
        if (op == 2) {
            ans = 0;
            if (ql <= mid)
                ans += query(lson, l, mid, ql, qr, op);
            if (mid < qr)
                ans += query(rson, mid + 1, r, ql, qr, op);
        } else {
            ans = -1;
            if (ql <= mid)
                ans = max(ans, query(lson, l, mid, ql, qr, op));
            if (mid < qr)
                ans = max(ans, query(rson, mid + 1, r, ql, qr, op));
        }
        pushUp(rt);
        return ans;
    }
} T;

\end{minted}

\clearpage

