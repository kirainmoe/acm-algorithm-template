\section{平衡二叉树}
\subsection{替罪羊树}
\begin{minted}{c++}
class ScapeGoat {
public:
    struct Node {
        /* p: 父节点编号; size: 以该节点为根的子树节点个数(包括自身)
            num: 当前节点维护的数; ch[0]: 左儿子, ch[1]: 右儿子 */
        int p, num, size, ch[2];
    };

    const double alpha;     /* 旋转因子 0.5 <= α <= 1 */
    Node t[MAXN];           /* 节点池 */
    int cnt, root,          /* cnt: 记录节点个数,root: 记录根节点编号 */
        sum, idx[MAXN];     /* idx[] 和 sum 用于重构时储存链状数据在节点池中的下标和节点数 */
    #define lson(x) t[x].ch[0]
    #define rson(x) t[x].ch[1]

    /* 构造函数,默认旋转因子为 0.75,树根为 1,需要构建两个 -inf 和 inf 的虚拟节点便于后序操作 */
    ScapeGoat(): alpha(0.75), cnt(2), root(1) {
        t[1].num = -INF, t[1].size = 2, t[1].ch[1] = 2;
        t[2].num = INF, t[2].size = 1, t[2].p = 1;
    }
    /**
        * 检查以 x 为根的子树是否平衡
        * 替罪羊树的平衡机制:当 0.5 <= alpha(旋转因子) <= 1 时,对于某个结点 x,如果该结点满足:
        * 1. size(lch(x)) <= alpha * size(x) --- x 的左子树节点数小于等于(旋转因子 * x 的子节点数)
        * 2. size(rch(x)) <= alpha * size(x) --- x 的右子树节点数小于等于(旋转因子 * x 的子节点数)
        * 即:这个节点两棵子树的 size 都不超过以该节点为根的子树的 size,那么就称该节点及其子树是平衡的。
        * 如果子树不是平衡的,那么直接重构成完全二叉树即可。
        */
    bool balance(int x) {
        return (
            ((double) t[x].size * alpha >= (double) t[lson(x)].size)
            && ((double) t[x].size * alpha >= (double) t[rson(x)].size)
        );
    }
    /* 将平衡树的元素变成链存在 idx[] 数组中 */
    void recycle(int x) {
        if (lson(x))
            recycle(lson(x));
        idx[++sum] = x;
        if (rson(x))
            recycle(rson(x));
    }
    /* 根据 idx[] 数组建立平衡树 */
    int build(int l, int r) {
        if (l > r)
            return 0;
        int mid = (l + r) >> 1, x = idx[mid];
        // 递归建立左右子树并为左右儿子写入父节点信息,然后更新子树根节点的 size
        lson(x) = build(l, mid - 1), rson(x) = build(mid + 1, r);
        t[lson(x)].p = t[rson(x)].p = x;
        t[x].size = t[lson(x)].size + t[rson(x)].size + 1;
        return x;
    }
    /* 重建替罪羊树以保证平衡结构 */
    void rebuild(int x) {
        sum = 0;        // 重置 sum,重新写入信息到 idx[] 数组中
        recycle(x);     // 拍扁成链
        int fa = t[x].p,            // 保存 x 当前的父节点
            son = t[fa].ch[1] == x, // 如果 son = 1, 更新右儿子;否则更新左儿子
            cur = build(1, sum);
        t[fa].ch[son] = cur, t[cur].p = fa; // 更新父节点和当前节点数据
        if (x == root)
            root = cur;
    }
    /* 将数字 x 插入到当前树中 */
    void insert(int x) {
        int u = root, cur = ++cnt;
        t[cur].size = 1, t[cur].num = x;        // 插入数 x 到新分配的节点中
        while (1) {
            t[u].size++;                        // 从根节点向下更新节点的数量
            int son = (x >= t[u].num);
            if (t[u].ch[son])
                u = t[u].ch[son];
            else {
                t[u].ch[son] = cur, t[cur].p = u;
                break;
            }
        }
        // 检查平衡性
        int flag = 0;
        for (int i = cur; i; i = t[i].p)
            if (!balance(i))
                flag = i;
        // 如果不平衡,则暴力重构
        if (flag)
            rebuild(flag);
    }
    /* 删除编号为 x 的数 */
    void erase(int x) {
        if (lson(x) && rson(x)) {
            int cur = lson(x);
            while (rson(cur))
                cur = rson(cur);
            t[x].num = t[cur].num, x = cur;
        }
        int son = lson(x) ? lson(x) : rson(x),
            k = rson(t[x].p) == x;
        t[t[x].p].ch[k] = son, t[son].p = t[x].p;

        for (int i = t[x].p; i; i = t[i].p)
            t[i].size--;                // 更新节点数目
        if (x == root)
            root = son;                 // 特殊处理删除的节点为根的情况
    }
    /* 查找数 x 在树节点池 t[] 数组中的下标 */
    int getPos(int x) {
        int u = root;
        while (1) {
            if (t[u].num == x)
                return u;
            else
                u = t[u].num < x ? rson(u) : lson(u);
        }
    }
    /* 询问 x 在树中的排名 */
    int getRank(int x) {
        int u = root, ans = 0;
        while (u)
            if (t[u].num < x) {
                ans += t[lson(u)].size + 1;
                u = rson(u);
            } else
                u = lson(u);
        return ans;
    }
    /* 询问第 k 个数的值 */
    int getKth(int k) {
        int u = root;
        k++;
        while (1)
            if (t[lson(u)].size == k - 1)
                return t[u].num;
            else if (t[lson(u)].size >= k)
                u = lson(u);
            else
                k -= t[lson(u)].size + 1, u = rson(u);
        return t[u].num;
    }
    /* 获取 x 的前驱结点 */
    int getForward(int x) {
        int u = root, ans = -INF;
        while (u) {
            if (t[u].num < x)
                ans = max(ans, t[u].num), u = rson(u);
            else
                u = lson(u);
        }
        return ans;
    }
    /* 获取 x 的后继结点 */
    int getBackword(int x) {
        int u = root, ans = INF;
        while (u) {
            if (t[u].num > x)
                ans = min(ans, t[u].num), u = lson(u);
            else
                u = rson(u);
        }
        return ans;
    }
} T;    
\end{minted}

\subsection{Treap (指针版)}
\begin{minted}{c++}
class Treap {
public: 
    struct Node {
        int val /* 当前结点维护的数值 */, size /* 子树结点和本身的数个数 */,
            cnt /* 值为 val 的数个数 */,  rd   /* 优先级,通常是随机值 */;
        Node *ch[2];    /* ch[0] 左儿子,ch[1] 右儿子 */
    } t[MAXN], nil;     /* 结点池 & 空指针 */
    
    int cnt;            /* 结点池节点总数 */
    Node *root, *NIL;   /* 根节点 & 空指针 */
    
    #define lson(p) p->ch[0]
    #define rson(p) p->ch[1]
    
    Treap(): cnt(0), root(NULL) {
        srand(time(0));
        nil = (Node) { 0, 0, 0, 0, 0, 0 }, NIL = &nil;
        build();
    }
    
    /**
    * 初始化操作,首先插入两个初始结点(inf 和 -inf) 并构造其关系 
    * 令树根为无穷小结点 (1) ,不要忘记 pushUp 
    */
    void build() {
        alloc(-INF), alloc(INF);
        root = &t[1], rson(root) = &t[2];
        pushUp(root);
    }
    
    /**
    * 在结点池中为新结点分配空间 
    */
    Node* alloc(int val) {
        t[++cnt].val = val, t[cnt].cnt = t[cnt].size = 1, t[cnt].rd = rand();
        t[cnt].ch[0] = t[cnt].ch[1] = NIL;
        return &t[cnt];
    }
    
    /**
    * 利用左右儿子的信息更新当前结点的 size, size[p] = size[lson] + size[rson] + cnt[p]
    */
    void pushUp(Node *p) {
        p->size = lson(p)->size + rson(p)->size + p->cnt;
    }
    
    /**
    * 右旋操作,注意 p 传入的是指针引用 
    *    P                     L              1. *ls = L
    *  /   \                     \            2. lson[rt]: L => rson[ls]: Q
    * L     R  == zig(P) ==>      P           3. rson[ls]: Q => P(*rt)
    *  \                        /  \          4. *rt => *ls(L)
    *   Q                      Q    R         5. pushUp(rt): L, pushUp(rson[rt]): P
    */
    void zig(Node *&p) {
        Node *ls = lson(p);
        lson(p) = rson(ls), rson(ls) = p;
        p = ls;
        pushUp(p), pushUp(rson(p));
    }
    
    /**
    * 左旋操作同理 
    *    P                     L              1. *rs = P
    *  /   \                     \            2. rson[rt]: P => lson[rs]: Q
    * L     R  <== zag(L) ==      P           3. son[rs]: Q => L(*rt)
    *  \                        /  \          4. *rt => *rs(L)
    *   Q                      Q    R         5. pushUp(rt): P, pushUp(lson[p]): L
    */
    void zag(Node *&p) {
        Node *rs = rson(p);
        rson(p) = lson(rs), lson(rs) = p;
        p = rs;
        pushUp(p), pushUp(lson(p));
    }
    
    /**
    * 插入一个新的结点,rt 为当前操作子树的指针,val 是待插入的数 
    * 1. 如果 rt 是空,说明当前 rt 是叶结点,可以直接分配空间
    * 2. 如果 rt 指向结点的 val 等于待插入数,说明有相同的数,直接让 cnt 自增 
    * 3. 如果 val < 当前 val,那么向左子树插入,完成后判断优先级关系,不满足则右旋
    * 4. 如果 val > 当前 val,那么向右子树插入,完成后判断优先级关系,不满足则左旋
    * 5. 最后调用 pushUp 更新 size 和 cnt
    * 无论何时,根节点的 rd 值应该始终大于等于左右儿子的 rd 值 
    */
    void insert(Node *&rt, int val) {
        if (rt == NIL) {
            rt = alloc(val);
            return;
        } else if (rt->val == val) {
            rt->cnt++;
        } else if (val < rt->val) {
            insert(lson(rt), val);
            if (rt->rd < lson(rt)->rd)
                zig(rt);
        } else {
            insert(rson(rt), val);
            if (rt->rd < rson(rt)->rd)
                zag(rt);
        }
        pushUp(rt);
    }
    
    /**
    * 删除结点,rt 为当前操作的子树的指针,val 是待删除的数值,每次只删除一次 
    * 1. 如果 rt 为空,说明代表该值的结点不存在 
    * 2. 如果 rt 指向的当前结点的值等于待删除的数值,则分两种情况处理:
    *   (a) 如果当前结点维护的数的数量不止一个 (rt->cnt > 1),那么让其减一并 pushUp 更新
    *   (b) 如果当前结点维护的数只有一个,则判断其是否存在左右儿子:
    *       (i) 如果不存在左右儿子,说明这是一个叶结点,直接将其删除(rt = NIL)
    *       (ii) 如果存在左右儿子,则判断
    *           (1) 如果右儿子为空或左儿子的优先级大于右儿子,右旋并从右子树查找删除 
    *           (2) 反之则左旋并从左子树查找删除
    * 3. 其他情况则根据 val 和 rt->val 的大小情况进入左右子树查找删除
    * 4. 最后不要忘记 pushUp 更新结点信息 
    */
    void erase(Node *&rt, int val) {
        if (rt == NIL) {
            return;
        } else if (rt->val == val) {
            if (rt->cnt > 1) {
                rt->cnt--;
            } else if (lson(rt) != NIL || rson(rt) != NIL) {
                if (rson(rt) == NIL || lson(rt)->rd > rson(rt)->rd)
                    zig(rt), erase(rson(rt), val);
                else
                    zag(rt), erase(lson(rt), val);
            } else {
                rt = NIL;
                return;
            }
        } else if (val < rt->val)
            erase(lson(rt), val);
        else
            erase(rson(rt), val);
        pushUp(rt);
    }

    /**
    * 查询某个数 val 在所给的以 rt 为根的子树中的排名,注意 Treap 满足二叉搜索树的性质 
    * 如果 rt 为空,说明是叶子结点,那么返回 0
    * 如果 val == rt->val,那么直接返回在左子树中 val 所在节点的左子树的 size + 1 
    * 如果 val 在左子树中,则递归进入左子树中继续查询;
    * 如果在右子树中,排名即为 rt 左子树的 size + rt 的 cnt + 右子树中的查询结果 
    */ 
    int rank(Node *rt, int val) {
        if (rt == NIL)
            return 0;
        else if (rt->val == val)
            return lson(rt)->size + 1;
        else if (val < rt->val)
            return rank(lson(rt), val);
        else
            return lson(rt)->size + rt->cnt + rank(rson(rt), val);
    }
    
    /**
    * 查询在以 rt 为根的子树中排名为第 k 大的数 
    * 如果 rt 为空,说明到了叶子结点都没有找到,随便返回啥都行 
    * 如果 rt 的左儿子的 size 大于 k,说明第 k 大的数在左子树中
    * 否则如果 rt 左儿子的 size + rt->cnt >= k,说明 rt 指向的数就是第 k 大的数 
    * 其他情况就在右子树中寻找 k - lson(rt)->size - rt->cnt 大的数 
    */
    int kth(Node *rt, int rk) {
        if (rt == NIL)
            return INF;
        else if (lson(rt)->size >= rk)
            return kth(lson(rt), rk);
        else if (lson(rt)->size + rt->cnt >= rk)
            return rt->val;
        else
            return kth(rson(rt), rk - lson(rt)->size - rt->cnt);
    }
    
    /**
    * 求 val 的前驱(前驱定义为比 val 小的第一个数) 
    */
    int prev(int val) {
        int ans = -INF;
        Node *rt = root;
        while (rt != NIL) {
            if (val == rt->val) {
                if (lson(rt) != NIL) {
                    rt = lson(rt);
                    while (rson(rt) != NIL)
                        rt = rson(rt);
                    ans = rt->val;
                }
                break;
            }
            if (rt->val < val && rt->val > ans)
                ans = rt->val;
            rt = val < rt->val ? lson(rt) : rson(rt);
        }
        return ans;
    }
    
    /**
    * 求 val 的后继(后继定义为比 val 大的第一个数) 
    */
    int next(int val) {
        int ans = INF;
        Node *rt = root;
        while (rt != NIL) {
            if (val == rt->val) {
                if (rson(rt) != NIL) {
                    rt = rson(rt);
                    while (lson(rt) != NIL)
                        rt = lson(rt);
                    ans = rt->val;
                }
                break;
            }
            if (rt->val > val && rt->val < ans)
                ans = rt->val;
            rt = val < rt->val ? lson(rt) : rson(rt);
        }
        return ans;
    }
} T;    
\end{minted}
\clearpage

\subsection{FHQTreap (无旋 Treap)}
\begin{minted}{c++}
class FHQTreap {
public:
    struct Node {
        int val, size, cnt, rd;
        Node *ch[2];
    } t[MAXN], nil, *root, *NIL;
    int cnt;
    
    #define lson(x) x->ch[0]
    #define rson(x) x->ch[1]
    
    FHQTreap(): cnt(0) {
        srand((int)time(NULL));
        nil = (Node) { 0, 0, 0, 0, 0, 0 }, root = NIL = &nil;
    }
    void pushUp(Node *p) {
        p->size = lson(p)->size + rson(p)->size + p->cnt;
    }
    void split(Node *now, int k, Node *&x, Node *&y) {
        if (now == NIL)
            x = y = NIL;
        else if (now->val <= k)
            x = now, split(rson(now), k, rson(now), y);
        else
            y = now, split(lson(now), k, x, lson(now));
        pushUp(now);
    }
    void rankSplit(Node *now, int k, Node *&x, Node *&y) {
        if (now == NIL)
            x = y = NIL;
        else {
            if (k <= lson(now)->size)
                y = now, split(lson(now), k, x, lson(now));
            else
                x = now, split(rson(now), k - lson(now)->size - 1, rson(now), y);
            pushUp(now);
        }
    }
    Node* merge(Node *a, Node *b) {
        if (a == NIL || b == NIL)
            return a == NIL ? b : a;
        if (a->rd < b->rd) {
            rson(a) = merge(rson(a), b), pushUp(a);
            return a;
        } else {
            lson(b) = merge(a, lson(b)), pushUp(b);
            return b;
        }
    }
    Node* alloc(int val) {
        t[++cnt] = (Node) { val, 1, 1, rand(), NIL, NIL };
        return &t[cnt];
    }
    void insert(int val) {
        Node *x, *y;
        split(root, val, x, y), root = merge(merge(x, alloc(val)), y);
    }
    void erase(int val) {
        Node *x, *y, *z;
        split(root, val, x, z), split(x, val-1, x, y);
        y = merge(lson(y), rson(y)), root = merge(merge(x, y), z);
    }
    int rank(int val) {
        Node *x, *y;
        split(root, val-1, x ,y);
        int res = x->size - 1;
        root = merge(x, y);
        return res;
    }
    Node *kth(Node *cur, int k) {
        while (1)
            if (k <= lson(cur)->size)
                cur = lson(cur);
            else
                if (k <= lson(cur)->size + cur->cnt)
                    return cur;
                else
                    k = k - (lson(cur)->size + cur->cnt), cur = rson(cur);
    }
    int prev(int t) {
        Node *x, *y;
        split(root, t-1, x, y);
        int ans = kth(x, x->size)->val;
        root = merge(x, y);
        return ans;
    } 
    int next(int t) {
        Node *x, *y;
        split(root, t, x, y);
        int ans = kth(y, 1)->val;
        root = merge(x, y);
        return ans;
    }
};    
\end{minted}
\clearpage


\subsection{伸展树 Splay (指针版)}
\begin{minted}{c++}
class Splay {
private:
    static const int MAXN = 200050,
        INF = 0x3f3f3f3f;
public:
    struct Node {
        int val,  /* 结点维护的数值 */
            size, /* 子树结点和本身的数个数 */
            cnt;  /* 值为 val 的数个数 */
        Node *ch[2], *p;    /* 左右儿子和父亲指针 */
    } t[MAXN]  /* 结点池 */,     nil  /* 逻辑空结点 */;
    Node *root /* 根节点指针 */, *NIL /* 逻辑空结点指针 */;
    
    int cnt;   /* 结点池中结点总数 */
    
    #define lson(x) x->ch[0]
    #define rson(x) x->ch[1]
    #define son(x, d) x->ch[d]
    #define p(x) x->p
    
    /* 构造函数 */
    Splay(): cnt(0) {
        nil = (Node) { 0, 0, 0, 0, 0, 0 }, root = NIL = &nil;
    }
    /**
    * 利用左右儿子信息更新当前结点的 size 
    */
    void pushUp(Node *p) {
        p->size = lson(p)->size + rson(p)->size + p->cnt;
    }
    /**
    * 根据所给数组,直接建立平衡树
    */
    Node* build(int l, int r, Node *p, int a[], int d, int offset) {
        if (l > r)
            return NIL;
        int m = (l + r) >> 1;
        Node *mid = alloc(m + offset, a[m], p);
        if (p != NIL)
            son(p, d) = mid;
        else if (!offset)
            root = mid;
        build(l, m - 1, mid, a, 0, offset);
        build(m + 1, r, mid, a, 1, offset);
        pushUp(mid);
        return mid;
    }    
    /**
    * 单旋操作函数, x 为待操作的结点,dir 是方向 (0左旋,1右旋) 
    * splay 每个结点都记录了它的父亲,因此可以直接传入操作结点,并得到它的父亲
    * 也因此,在操作修改父子关系的时候,不要忘记修改子结点的父亲 
    * 下面的注释以 zig(右旋) 为例,zag(左旋) 是同理的 
    *    P                           L              
    *  /   \                           \ 
    * L     R  == rotate(L, 1) ==>      P 
    *  \                              /  \        
    *   Q                            Q    R   
    */
    void rotate(Node *x, bool dir) {
        /* step1: 获得操作结点的父结点 */
        Node *y = p(x);                 // P
        // pushDown(x), pushDown(y);
        /* step2: 将当前结点的右儿子赋给父结点的左儿子 */
        son(y, !dir) = son(x, dir);     // lson(P) <= rson(L): Q
        if (son(x, dir) != NIL)
            son(x, dir)->p = y;             // parent(Q) <= P
        /* step3: 将操作结点接到其父结点的父结点 */
        p(x) = p(y);                    // parent(L) <= parent(P)
        /* step4: 判断父结点的父结点是否为空(如果为空说明当前父结点是根)
                    否则就断开祖先结点与父结点的联系,建立祖先结点与操作结点的联系。
                    更新父结点的父结点的左/右子树信息;两个节点随意旋转不会改变相对祖先结点的方向 */ 
        if (p(y) != NIL) {
            if (y == son(p(y), dir))    // 这一步在图中就没法体现了 
                son(p(y), dir) = x;
            else 
                son(p(y), !dir) = x;
        }
        /* step5: 最后将操作结点旋转方向上的儿子设为原先的父结点 */
        son(x, dir) = y, p(y) = x;      // rson(L) <= P, parent(P) <= L
        /* step6: 不要忘记更新结点的信息 */
        pushUp(y), pushUp(x);           // 先 y 后 x,顺序不能变 
    }
    /**
    * 伸展(splay) 操作,将结点 x 伸展到 y 的子节点处
    * 如果要将 x 旋转到根,那么 y 可以传入 Splay::NIL 
    * 分三种情况:
    * (1) 如果当前结点的父结点即为(子树的)根节点,单旋即可
    * (2) 设 x, y=p(x), z=p(y),如果 x, y 都是对应父亲相同方向上的子结点,则同时对两个结点右旋或左旋
    * (3) 如果 x, y 是对应父亲不同方向上的子结点,那么对两者进行反方向的旋转,可以直接用 while 完成 
    */
    void splay(Node *x, Node *y) {
        Node *tmp;
        while (p(x) != y) {             // 循环进行旋转操作 
            // pushDown(x);
            // if (y != NIL)    pushDown(y);
            tmp = x->p;
            if (x == lson(tmp)) {       // 如果 x 是左儿子 
                // 当前 tmp 的父结点不是 y,并且 tmp 也是祖先的左儿子 
                // 都在左侧一条链上的情况,进行 zig-zig 
                if (p(tmp) != y && tmp == lson(tmp->p))
                    rotate(tmp, 1);
                rotate(x, 1);
            } else {                    // x 是右儿子 
                // 都在右侧一条链上的情况,进行 zag-zag 
                if (p(tmp) != y && tmp == rson(tmp->p))
                    rotate(tmp, 0);
                rotate(x, 0);
            }
        }
        if (y == NIL) {
            root = x;   // 如果 y 是NIL,说明旋转后 x 是根 
        }
    }
    Node* alloc(int val, Node *p) {
        t[++cnt] = (Node) {
            val, 1, 1, NIL, NIL, p      // val, size, cnt, lson, rson, p
        };
        return &t[cnt];
    }
    /**
    * 插入一个新数到 splay 中 
    */
    void insert(int val) {
        if (root == NIL) {      // root 为 NIL 说明树为空 
            cnt = 0, root = alloc(val, NIL);
            return;
        }
        Node *x = root, *y;
        while (1) {
            /* 插入新节点的时候,从根节点向下更新以 i 为根的子树结点数 */
            x->size++;
            if (val == x->val) {
                x->cnt++;
                pushUp(x);
                y = x;
                break;
            } else if (val < x->val) {
                if (lson(x) != NIL)
                    x = lson(x);
                else {          // 在左儿子处插入新节点 
                    lson(x) = y = alloc(val, x);
                    break;
                }
            } else {
                if (rson(x) != NIL)
                    x = rson(x);
                else {
                    rson(x) = y = alloc(val, x);
                    break;
                }
            }
        }
        
        splay(y, NIL);      // 完成插入后将新节点旋转到根 
    }
    /**
    * 查找一个数字所在的结点并返回指针 
    * 操作后,被查找的数会被旋转到根节点 
    */
    Node *search(int val) {
        if (root == NIL)
            return NIL;         // 不存在的结点 
        Node *x = root, *y = NIL;
        while (1) {
            if (val == x->val) {
                y = x;
                break;
            } else if (val < x->val) {
                if (lson(x) != NIL)
                    x = lson(x);
                else
                    break;          // 404
            } else {
                if (rson(x) != NIL)
                    x = rson(x);
                else
                    break;          // 404
            }
        }
        splay(x, NIL);       // 完成查找后将离 val 最近的结点伸展到根部
        return y; 
    }
    /**
    * 查询在以 rt 为根的子树中的最小值,只需要一直往左儿子走,走到底就可以 
    */
    Node *minNode(Node *rt) {
        Node *y = p(rt);
        while (lson(rt) != NIL)
            rt = lson(rt);
        splay(rt, y);       // 记录原先 rt 的根节点,将最小值旋到子树的根部 
        return rt;
    }
    /**
    * 根据值删除结点(如果一个数有多个,只删除一个) 
    */
    void erase(int val) {
        if (root == NIL)
            return;         // splay 为空 
        Node *x = search(val),  // search 之后,值为 val 的结点会被旋转到根部 
                *y;
        if (x == NIL)       // 不存在值为 val 的结点 
            return; 
        if (x->cnt > 1) {   // 值为 val 的数有两个以上,直接减掉一个 
            x->cnt--;
            pushUp(x);
            return; 
        } else if (lson(x) == NIL && rson(x) == NIL) {  // x 没有左右儿子,直接删除 
            root = NIL, cnt = 0;    // 因为 search(val) 将值为 val 的数旋到了根,所以如果没有左右子结点说明树为空 
            return;
        } else if (lson(x) == NIL) {
            root = rson(x);
            p(rson(x)) = NIL;
            return;
        } else if (rson(x) == NIL) {
            root = lson(x);
            p(lson(x)) = NIL;
            return;
        } else {       // 左右子树都为空,从右子树中找最小替换根 
            y = minNode(rson(x));       // 完成后最小值会被旋到 x 的右儿子 
            p(y) = NIL, p(lson(x)) = y, lson(y) = lson(x);
            pushUp(y);
            root = y;
        }
    }
    /**
    * 求某个数是第几大 
    * 直接旋到根,然后返回左儿子的 size + 1 
    */
    int rank(int val) {
        Node *x = search(val);
        return x == NIL ? 0 /* 404 */ : lson(x)->size + 1;
    }
    /**
    * 求第 k 大的数是多少 
    */
    Node* kth(int k) {
        if (root == NIL || k > root->size)    // 不存在或比当前数的数量大 
            return NIL;
        Node *x = root;
        while (1) {
            // pushDown(x);
            if (lson(x)->size + 1 <= k && k <= lson(x)->size + x->cnt)
                break;      // 当前 x 指向即为目标
            else if (k <= lson(x)->size)
                x = lson(x);    // 到左子树中找
            else
                k -= lson(x)->size + x->cnt, x = rson(x); 
        }
        splay(x, NIL);      // 将 x 旋到根部 
        return x;
    } 
    /* 查询前驱结点(第一个小于 num 的结点) */
    int prev(int val) {
        Node *u = root;
        int res = -INF;
        while (u != NIL) {
            if (u->val < val && u->val > res)
                res = u->val;
            if (val > u->val)
                u = rson(u);
            else
                u = lson(u);
        }
        return res;
    }
    /* 查询后继结点(第一个大于 num 的结点) */
    int next(int val) {
        Node *u = root;
        int res = INF;
        while (u != NIL) {
            if (u->val > val && u->val < res)
                res = u->val;
            if (val < u->val)
                u = lson(u);
            else
                u = rson(u); 
        }
        return res;
    }
} T;    
\end{minted}


\noindent\par Splay 的基本区间操作(例题:NOI2005 维护数列): 
\begin{itemize}
    \item 操作区间 [l, r]:首先先把 (l-1) 旋到根节点,再把 (r+1) 旋到根节点右儿子上,那么现在根节点的右儿子的左儿子(即 r+1 的左儿子)即为所求区间。
    \item Splay 维护区间最大子序列:每个结点记录 lmx, rmx, mmx; 分别表示:从区间左端点 l/右端点 r 开始的连续的前缀最大子序列,以及该区间的最大子序列。
    \item 转移方程 $lmx = max(lson->lmx, lson->sum + x->val + rson->lmx)$,$rmx$ 同理对偶;$mmx = max(lson->mmx, rson->mmx, lson->rmx + x->val + rson->lmx)$.
    \item 将区间 [l, r] 变为同一个数:打标记并修改值,$sum = size \times val$,如果待修改的值 $val > 0$,那么 $lmx = rmx = mmx = sum$;否则 $lmx = rmx = 0, mmx = val$.
    \item 将区间 [l, r] 翻转:打上修改标记,并交换左右儿子,然后交换当前结点的 $lmx$ 和 $rmx$.
\end{itemize}


\subsection{平衡树测试数据}
1. 插入,2. 删除,3. 查询某数排名,4. 按排名查数,5. 查某个数前驱,6. 求某个数后继。
\begin{lstlisting}
Input:               Output:
20                   964673
1 964673             964673
5 968705             1
4 1                  964673
3 964673             3
5 965257             1
1 915269             1
1 53283              964673
3 964673             964673
3 53283
3 53283
1 162641
5 973984
1 948119
2 915269
2 53283
6 959161
1 531821
1 967521
2 531821
1 343410
\end{lstlisting}
\clearpage