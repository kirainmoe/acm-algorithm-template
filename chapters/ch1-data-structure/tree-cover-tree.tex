\section{树套树}

\subsection{线段树套平衡树 (Treap)}

\paragraph{应用场景} 维护多维度的信息,例如查询某个数在区间内的排名、查询区间第 $k$ 大,单点修改,查询区间内 $k$ 的前驱和后继等等。

\begin{minted}{c++}
class SegTree {
public:
    Treap::Node* t[MAXN << 2];
    SegTree() {
        fill(t, t + (MAXN << 2), T.NIL);
    }
    void build(int rt, int l, int r) {
        for (int i = l; i <= r; i++)
            T.insert(t[rt], a[i]);
        if (l != r) {
            int mid = (l + r) >> 1;
            build(rt << 1, l, mid);
            build(rt << 1 | 1, mid + 1, r);
        }
    }
    void update(int rt, int l, int r, int x, int y) {
        T.erase(t[rt], a[x]);
        T.insert(t[rt], y);
        if (l == r)
            return;
        int mid = (l + r) >> 1;
        if (x <= mid)
            update(rt << 1, l, mid, x, y);
        else
            update(rt << 1 | 1, mid + 1, r, x, y);
    }
    int rank(int rt, int l, int r, int ql, int qr, int k) {
        if (qr < l || ql > r )
            return 0;
        if (ql <= l && r <= qr)
            return T.rank(t[rt], k);
        else {
            int mid = (l + r) >> 1;
            return rank(rt << 1, l, mid, ql, qr, k) + rank(rt << 1 | 1, mid + 1, r, ql, qr, k);
        }
    }
    int kth(int x, int y, int k) {
        int minx = 0, maxx = 1e8, mid, tmp;
        while (minx < maxx) {
            mid = (minx + maxx + 1) >> 1;
            if ((tmp = rank(1, 1, n, x, y, mid)) < k)
                minx = mid;
            else 
                maxx = mid - 1;
        }
        return maxx;
    }
    int prev(int rt, int l, int r, int ql, int qr, int k) {
        if (qr < l || ql > r )
            return -INT_MAX;
        if (ql <= l && r <= qr)
            return T.prev(t[rt], k);
        else {
            int mid = (l + r) >> 1;
            return max(prev(rt << 1, l, mid, ql, qr, k), prev(rt << 1 | 1, mid + 1, r, ql, qr, k));
        }
    }
    int next(int rt, int l, int r, int ql, int qr, int k) {
        if (qr < l || ql > r )
            return INT_MAX;
        if (ql <= l && r <= qr)
            return T.next(t[rt], k);
        else {
            int mid = (l + r) >> 1;
            return min(next(rt << 1, l, mid, ql, qr, k), next(rt << 1 | 1, mid + 1, r, ql, qr, k));
        }
    }    
};
\end{minted}


\subsection{树状数组套动态开点权值线段树}

\paragraph{应用场景} 查询一段区间内数的个数 + 单点修改。

\begin{minted}{c++}
const int maxn = 2e5 + 10, lim = 2e5;

// 动态开点权值线段树
class SegTree {
public:
    int tot;
    struct Node {
        ll sum;
        int lc, rc;
    } t[maxn * 80];

    SegTree(): tot(0) {}
    void pushUp(int x) {
        t[x].sum = t[t[x].lc].sum + t[t[x].rc].sum;
    }
    void update(int &root, int l, int r, int x, int val) {
        if (!root)
            root = ++tot;
        if (l == r) {
            t[root].sum += val;
            return;
        }
        int mid = (l + r) >> 1;
        if (x <= mid)
            update(t[root].lc, l, mid, x, val);
        else
            update(t[root].rc, mid + 1, r, x, val);
        pushUp(root);
    }
    int query(int root, int l, int r, int ql, int qr) {
        if (!root)
            root = ++tot;
        if (l >= ql && r <= qr)
            return t[root].sum;
        int mid = (l + r) >> 1, ans = 0;
        if (ql <= mid)
            ans += query(t[root].lc, l, mid, ql, qr);
        if (qr > mid)
            ans += query(t[root].rc, mid + 1, r, ql, qr);
        return ans;
    }
} T;

/* 树状数组 */
// rt[i] 保存每个单点对应的权值线段树根节点,rt1, rt2 分别表示区间 [l, r] 的树根
int rt[maxn], rt1[maxn], rt2[maxn], cnt1, cnt2, n, m;
int lowbit(int x) {
    return x & (-x);
}
// 单点修改:update_BIT(i, a[i], $\pm$ a[i]) 
void update_BIT(int pos, int x, int val) {
    for (int i = pos; i <= n; i+= lowbit(i))
        T.update(rt[i], 1, lim, x, val);
}
// 求区间 [l, r] 的端点在两端的根
void locate(int l, int r) {
    cnt1 = cnt2 = 0;
    for (int i = l-1; i > 0; i -= lowbit(i))
        rt1[cnt1++] = rt[i];
    for (int i = r; i > 0; i-= lowbit(i))
        rt2[cnt2++] = rt[i];
}
// 求区间 [l, r] 有多少个数字 k
ll ask(int l, int r, int k) {
    ll ans = 0;
    if (r == k) {
        for (int i = 0; i < cnt1; i++)
            ans -= T.t[rt1[i]].sum;
        for (int i = 0; i < cnt2; i++)
            ans += T.t[rt2[i]].sum;
        return ans;
    }
    int mid = (l + r) >> 1;
    if (k <= mid) {
        for (int i = 0; i < cnt1; i++)
            rt1[i] = T.t[rt1[i]].lc;
        for (int i = 0; i < cnt2; i++)
            rt2[i] = T.t[rt2[i]].lc;
        return ask(l, mid, k);
    } else {
        for (int i = 0; i < cnt1; i++)
            ans -= T.t[T.t[rt1[i]].lc].sum;
        for (int i = 0; i < cnt2; i++)
            ans += T.t[T.t[rt2[i]].lc].sum;
        for (int i = 0; i < cnt1; i++)
            rt1[i] = T.t[rt1[i]].rc;
        for (int i = 0; i < cnt2; i++)
            rt2[i] = T.t[rt2[i]].rc;
        return ans + ask(mid + 1, r, k);
    }
}

\end{minted}

\clearpage