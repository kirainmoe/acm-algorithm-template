\section{Link-Cut Tree}

\paragraph{动态树问题}

\par 要求维护一个\textbf{有根树森林},支持树的分割、合并等操作,即支持在线维护树的连通性,并且在一条链上或路径上做修改。

\paragraph{实链剖分}

\par 动态树 / LCT 维护的是一个森林,希望通过实链剖分,使得动态树的链成为我们指定的、便于求解的链;
对于节点 $u$ 及 $u$ 的所有子树与 $u$ 的连边 $(u, v_i) \in E$,选择一条边 $(u, v*)$,以该边进行剖分,则称 $(u, v*)$ 为实边,$(u, v_i), v_i \neq v*$ 为虚边。

\paragraph{辅助树}

\begin{itemize}
\item 若干棵 splay 构成了辅助树,每棵 splay 维护原树中的一条路径
\item 每棵“辅助树”维护的是“一棵原树”
\item 若干棵辅助树构成了 LCT,维护整个森林
\item 辅助树中 splay 的根节点的父节点,指向\textbf{原树中当前链的父节点},即当前链最顶端的节点它的父节点。
\item 每个连通块中恰好有一个节点的父节点为空。
\end{itemize}

\begin{minted}{c++}
class LinkCutTree {
public:
    #define lson(x) ch[x][0]
    #define rson(x) ch[x][1]
    int ch[maxn][2], fa[maxn], size[maxn], stk[maxn], rev[maxn];
    inline bool isRoot(int x) {
        return lson(fa[x]) != x && rson(fa[x]) != x;
    }
    inline bool get(int x) {
        return rson(fa[x]) == x;
    }
    inline void pushUp(int x) {
        size[x] = size[lson(x)] + size[rson(x)] + 1;
    }
    inline void reverse(int x)  {
        swap(lson(x), rson(x));
        rev[x] ^= 1;
    }
    inline void pushDown(int x) {
        if (!rev[x]) 
            return;
        if (lson(x))
            reverse(lson(x));
        if (rson(x))
            reverse(rson(x));
        rev[x] = 0;
    }
    
    void rotate(int x) {
        int y = fa[x], z = fa[y], k = get(x), w = ch[x][!k];
        if (!isRoot(y))
            ch[z][get(y)] = x;
        ch[x][!k] = y, ch[y][k] = w;
        if (w)
            fa[w] = y;
        fa[y] = x, fa[x] = z;
        pushUp(y);
    }
    
    void splay(int x) {
        int top = 0;
        stk[++top] = x;
        for (int i = x; !isRoot(i); i = fa[i]) 
            stk[++top] = fa[i];
        while (top)
            pushDown(stk[top--]);
        int y, z;
        while (!isRoot(x)) {
            y = fa[x], z = fa[y];
            if (!isRoot(y))
                rotate((get(x) ^ get(y)) ? x : y);
            rotate(x);
        }
        pushUp(x);
    }

    // 访问节点 x,使从根节点到 x 的路径成为实链
    void access(int x) {
        for (int y = 0; x; x = fa[y = x])
            splay(x), rson(x) = y, pushUp(x);
    }
    // 将节点 x 变成 Splay 中的根
    void makeRoot(int x) {
        access(x), splay(x), reverse(x);
    }
    // 判断连通性,如果 find(x) == find(y) 则 x, y 在同一棵树中
    int find(int x) {
        access(x), splay(x);
        while (lson(x))
            pushDown(x), x = lson(x);
        splay(x);
        return x;
    }
    // 拉出 x=>y 的路径形成一个 splay
    void split(int x, int y) {
        makeRoot(x), access(y), splay(y);
    }
    // 连边 (x, y)
    void link(int x, int y) {
        makeRoot(x);
        if (find(y) != x)
            fa[x] = y;
    }
    // 断开边 (x, y)
    void cut(int x, int y) {
        makeRoot(x);
        if (find(y) == x && fa[y] == x && !rson(y)) {
            fa[y] = rson(x) = 0;
            pushUp(x);
        }
    }
    
    void init() {
        memset(ch, 0, sizeof ch), memset(fa, 0, sizeof fa);
        memset(size, 0, sizeof size), memset(rev, 0, sizeof rev);
    }
} T;
\end{minted}
\clearpage