\section{虚树}

\paragraph{问题场景}

\par 给定一棵树,有多次询问,每组询问对树上的一些关键点,询问它们的某些信息,满足所有询问中的关键点数量\textbf{总和}与树的大小同阶。此时可以对原树建立一棵只包含关键点和 LCA 的虚树,可以将问题的复杂度压缩到 $O(Q \log_2N + f(Q))$ 的等级。

\par 对一棵有根树而言,虚树中包含了所有询问中的关键点和它们的 LCA,尽可能地压缩了树的大小。同时,虚树当中对于连接两个点 $u, v$,边 $(u, v)$ 的边权定义为 :原树中 $u$ 到 $v$ 的最短路径。

\paragraph{关键性质}

\par 在一棵有根树中,任选 k 个不重复的点,这 k 个点两两之间的不同 LCA 数量不超过 k-1 个(即 k 个点两两之间的 LCA 数量不超过自身的阶数)。

\paragraph{构造虚树}

\begin{itemize}
    \item 首先对树上所有点进行 DFS,预处理处各个点的 DFS 序和 LCA。
    \item 对所有的关键点按照 DFS 序排序,然后顺序将关键点插入;新增一个超级根 (0 号节点)指向原来的树根。
    \item 用栈 $S$ 来维护虚树的右链,初始时  $S$ 内只有 $0$ 号节点。如果此时要加入节点 $u$:
    \begin{itemize}
        \item 首先求出 $u$ 与栈顶元素 $v = S[top]$ 的 LCA 点 $p$。
        \item 如果 $p = v$,说明 $(p, v)$ 是一条链,此时将 $v$ 入栈。
        \item 如果 $p \neq v$,那就说明 $u, v$ 在 $p$ 的不同子树中,此时不断退出栈顶元素 $S[top]$,直到栈顶的 **第二个元素** (PS. 加入超级节点的原因)的深度 $depth[S[top-1]] < depth[p]$。
        \item 在退栈的过程中,将栈顶元素 $v_0$ 和第二个元素 $v_1$ 连边 $(v_1, v_0)$。
        \item 重复该过程直到所有的点都被加入。
    \end{itemize}
\end{itemize}

\begin{minted}{c++}
namespace VirtualTree {
    int anc[maxn][maxlog + 5], dfn[maxn], depth[maxn], vdepth[maxn];
    bool vis[maxn];
    vector<int> VT[maxn];

    void dfs(int rt, int dep, int p) {
        anc[rt][0] = p, dfn[rt] = dcnt++, depth[rt] = dep, vis[rt] = 1;
        for (int i = 1; (1 << i) <= dep; i++)
            anc[rt][i] = anc[anc[rt][i-1]][i-1];
        for (int i = head[rt]; i; i = e[i].next) {
            if (e[i].v == p)
                continue;
            if (!vis[e[i].v])
                dfs(e[i].v, dep + 1, rt);
        }
    }

    int lca(int a, int b) {
        if (depth[a] < depth[b])
            swap(a, b);
        int d = depth[a] - depth[b];
        for (int i = 0; (1 << i) <= d; i++)
            if ((1 << i) & d)
                a = anc[a][i];
        if (a == b)
            return a;
        for (int i = maxlog ; i >= 0; i--) {
            if (anc[a][i] != anc[b][i])
                a = anc[a][i], b = anc[b][i];
        }
        return anc[a][0];
    }

    inline bool cmp(const int &i, const int &j) {
        return dfn[i] < dfn[j];
    }

    void build_virtual_tree(int node[], int k) {
        static int st[maxn];
        sort(node, node + k, cmp), sz = 0;
        for (int i = 0; i < k; i++) {
            if (i == 0)
                assert(node[i] == 1);
            if (sz <= 1) {
                st[++sz] = node[i];
                continue;
            }
            int u = node[i], p = lca(u, st[sz]);
            if (p == st[sz]) {
                st[++sz] = u;
                continue;
            }
            while (sz > 1 && dfn[p] <= dfn[st[sz-1]])
                VT[st[sz-1]].emplace_back(st[sz]), sz--;
            if (p != st[sz])
                VT[p].emplace_back(st[sz]), st[sz] = p;
            st[++sz] = u;
        }
        for (int i = 1; i < sz; i++)
            VT[st[i]].emplace_back(st[i+1]);
    }

    void calc_vt_depth(int x, int p) {
        fa[x] = p, vdepth[x] = vdepth[p] + 1;
        for (int i : VT[x])
            calc_vt_depth(i, x);
    }
}

// 处理路径 (u, v) 上的更新和询问 
ll a[maxn], w[maxn];
// 更新
while (u != v) {
    if (rdp[u] < rdp[v])
        swap(u, v);
    w[u] += x, a[u] += x, u = fa[u];
}
a[u] += x;
// 询问
ll ans = 0;
while (u != v)  {
    if (rdp[u] < rdp[v])
        swap(u, v);
    if (depth[u] - depth[fa[u]] - 1)
        ans += w[u] * (depth[u] - depth[fa[u]] - 1);    // 用 depth 统计 
    ans += a[u];
}
ans += a[u];
\end{minted}