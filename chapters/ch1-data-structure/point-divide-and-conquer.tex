\section{点分治}
\paragraph{问题场景}
\noindent \par 点分治是大规模处理树上路径问题的工具。大意是找到一个点,递归统计其所有子树的答案,然后利用容斥原理或其它方式合并答案,最后得到整棵树的答案。
\paragraph{步骤}
\begin{itemize}
    \item 找到整棵树的重心点 $rt$,由 $rt$ 向下递归求解。
    \item 统计以 $rt$ 为根的子树的答案 $ans_{rt}^\prime$,并使用容斥原理、染色法等去除不合法的答案,得到 $rt$ 的最终答案 $ans_{rt}$
    \item 对 $rt$ 的子树 $ch_i$ 求解,同样先找到以 $ch_i$ 为根的子树的重心 $r^\prime$,然后从重心 $r^\prime$ 向下递归求解,回到步骤 (1).
\end{itemize}
\par 例:计算一棵树上距离为 $k$ 的点是否存在
\begin{minted}{c++}
struct Edge {
    int u, v, next;
    ll w;
};
struct Dist {
    ll w;
    int par;
    bool operator < (const Dist &b) const {
        return w < b.w;
    }
};

const int MAXN = 10050, MAXM = 150, INF = 0x3f3f3f3f;
    
int n, m, cnt = 1, minv = INF, rt = 0, tot = 0, dcnt = 0;

bool vis[MAXN];
int head[MAXN], qry[MAXM], ans[MAXM], size[MAXN], son[MAXN];
Edge e[MAXN << 1];
Dist d[MAXN];

void findRoot(int u, int p) {
    size[u] = 1, son[u] = 0;
    for (int i = head[u], v; i; i = e[i].next) {
        if (vis[(v = e[i].v)] || v == p)
            continue;
        findRoot(v, u);
        size[u] += size[v];
        son[u] = max(son[u], size[v]);
    }
    son[u] = max(son[u], tot - size[u]);
    if (son[u] < minv)
        minv = son[u], rt = u;        
}
void getDist(int u, int p, int par, ll dt) {
    d[dcnt++] = (Dist) { dt, par };
    for (int i = head[u], v; i; i = e[i].next) {
        if (vis[(v = e[i].v)] || p == v)
            continue;
        getDist(v, u, par, dt + e[i].w);
    }
}
void solve(int cur) {
    dcnt = 0;
    for (int i = head[cur], v; i; i = e[i].next) {
        if (vis[(v = e[i].v)])
            continue;
        getDist(v, cur, v, e[i].w);
    }
    d[dcnt++] = (Dist){ 0ll, 0 };
    sort(d, d + dcnt);
    
    for (int i = 0; i < m; i++) {
        if (ans[i])
            continue;
        int l = 0;
        while (l < dcnt && d[l].w + d[dcnt-1].w < qry[i])
            l++;
        while (l < dcnt && !ans[i]) {
            if (qry[i] - d[l].w < d[l].w)
                break;
            int pos = lower_bound(d, d + dcnt, (Dist){ qry[i] - d[l].w, 0 }) - d;
            while (pos < dcnt && d[pos].par == d[l].par)
                pos++;
            if (pos < dcnt && d[pos].w + d[l].w == qry[i])
                ans[i] = 1;
            l++;
        }
    }
}
void work(int x) {
    vis[x] = 1;
    solve(x);
    for (int i = head[x], v; i; i = e[i].next) {
        if (vis[(v = e[i].v)])
            continue;
        rt = 0, minv = INF, tot = size[v];
        findRoot(v, 0);
        work(rt);
    }
}
int main() {
    // ...
    tot = n;
    findRoot(1, 0);
    work(rt);
    // ...
    return 0;
}
\end{minted}