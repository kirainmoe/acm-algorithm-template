\section{AC 自动机}

\par 在 Trie 树的基础上,为节点增加 fail 指针;当前节点失配的时候,将匹配指针转移到 fail 指针指向的节点。
\vspace{0.5em}

\par\textbf{建树}
\begin{itemize}
    \item 根节点指向的所有节点的 $fail$ 指针都指向根节点
    \item 不存在的节点,$fail$ 指针指向根节点
    \item 普通节点,字符为 $s$ 的 $fail$ 指针,指向它的父节点的 $fail$ 指针指向节点 $fail[p]$ 沿 $s$ 走到的节点。
\end{itemize}

\par\textbf{匹配}
\begin{itemize}
    \item 如果走到了不存在的节点,则将匹配指针移到 fail 指针指向的节点
    \item 从根节点开始匹配,原理与 Trie 树相同,匹配指针沿着 $p[i]$ 所在的字母向下走
    \item 如果失配,则沿着 fail 指针移动,若匹配上则继续匹配,否则不断沿着 fail 指针走。
\end{itemize}

\par\textbf{Fail 指针}
\begin{itemize}
    \item 每个节点 $s$ 有一个失配指针 $p$ ,所有的 $s$ 和它们的 $p$ 构成的树形结构称为 fail 树。
    \item fail 树上每个节点所代表的字符串,是其所有子树所代表的字符串的后缀 ⇒ 一个节点所有祖先,代表的字符串都是这个节点代表的字符串的后缀,如下图所示。
    \item 重要性质:每个节点的 fail 指针,都指向当前节点代表的字符串的最长后缀(如果存在)。
\end{itemize}

\par\textbf{Fail 指针的用法}
\begin{itemize}
    \item 统计每个模式串 p 在文本串 t 当中出现的次数:将 t 在 AC 自动机的上匹配同时建立 fail 树,当经过某个节点时,对答案的贡献为:这个节点所有祖先的权值之和。利用树上差分将经过的所有节点计数 + 1。
    \item 一个模式串 $p_i$ 在其它模式串中出现的次数统计:如果 $p_i$ 在其它的模式串中出现,那么其它模式串的链上一定有一个节点的 fail 指针指向该节点,直接统计该节点在 fail 树上的子节点个数即可。
\end{itemize}

\begin{center}
    \includegraphics[height=6cm]{chapters/ch4-string-and-palindrome/images/fail.png}
\end{center}

\begin{minted}{c++}
namespace AhoCorasickAutomaton {
    #define clear(x) memset(x, 0, sizeof x)
    int trie[maxn][26], has[maxn], fail[maxn], cnt;
    int sz[maxn], pos[maxn], q[maxn];
    void init() {
        clear(trie), clear(has), clear(fail), cnt = 0;
    }
    inline int idx(char s) {
        return s - 'a';
    }
    int insert(char* s, int l) {
        int u = 0;
        for (int i = 0, c; i < l; i++) {
            c = idx(s[i]);
            if (!trie[u][c])
                trie[u][c] = ++cnt;
            u = trie[u][c];
            sz[u]++;
        }
        has[u]++;
        return u;
    }
    void get_fail() {
        int head = 0, tail = 0;
        for (int i = 0; i < 26; i++)
            if (trie[0][i])
                fail[trie[0][i]] = 0, q[tail++] = trie[0][i];
        while (head < tail) {
            int u = q[head++];
            for (int i = 0; i < 26; i++)
                if (trie[u][i])
                    fail[trie[u][i]] = trie[fail[u]][i],
                    q[tail++] = trie[u][i];
                else
                    trie[u][i] = trie[fail[u]][i];
        }
    }
    void solve() {
        for (int i = cnt; i >= 0; i--)
            sz[fail[q[i]]] += sz[q[i]];
        for (int i = 0; i < n; i++)
            printf("%d\n", sz[pos[i]]);
    }
}    
\end{minted}