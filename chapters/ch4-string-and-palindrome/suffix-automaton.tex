\section{后缀自动机}
处理与子串相关问题的在线线性算法。
\paragraph{名词解释}
\begin{itemize}
\item  $len$ 表示从当前 $endpos$ 可向前延伸的长度最大值。设当前构造的 $SAM$ 已得到的子串为 $s$, 从当前字符向前数 $[0, len]$ 个字符得到的新子串 $t$,$t$ 一定只作为 $s$ 的后缀出现。
\item 如:对于母串 $abcdabcdbbcd$, 有三个子串 $d, cd, bcd$ 的 $endpos$ 集合相同,而从 $endpos$ 向前最多可以延伸 $3$ 个字符满足条件,所以 $len = 3$.
\item  $link$ 表示后缀连接。定义一个子串 $v \in endpos(v)$ ($endpos(v)$ 称为 $v$ 的 endpos 等价类),该等价类中长度最长的子串为 $w$, 则 $w$ 的“最长的且不在该 endpos 等价类中的后缀”记为 $t$,令 $link(v) = t$.
\item 如:对于母串 $abcdabcdcd$, $abcd, bcd$ 在同一个 endpos 等价类中。设 $v = bcd$, 则显然 $w = abcd$, $w$ 的最长的且不再该等价类中的后缀显然为 $cd$,则令 $link(bcd) = cd$. Tips: 后缀也有长度为 0 的情况,即空后缀。
\item  $nxt[i]$ 就是沿着字符 $'a' + i$ 走可以到达的下一个状态。
\end{itemize}

\paragraph{性质}

\begin{itemize}
\item 字符串 $s$ 的一个后缀自动机包含关于字符串 $s$ 的所有子串的信息。任意从初始状态 $t_0$ 开始的路径,如果我们将转移路径上的标号写下来,都会形成 $s$ 的一个 子串 。反之每个 $s$ 的子串对应从初始状态 $t_0$  开始的某条路径。
\item 每个状态 $s$ 代表的子串是区间 $[len_{link(s)}+1, len_s]$.
\item 一个长度为 $n$ 的字符串,它的 SAM 节点个数最多有 $2n-1$ 个,连边最多有 $3n-4$ 条。
\item 树形结构的性质:设字符串长度为 $n$,考虑 extend 操作中 $cur$ 变量的值(代表当前状态在节点池中的下标),该节点对应的状态是:执行 extend 操作时的当前字符串,得到的 $n$ 个节点对应了$n$ 个不同的终点,第 $i$ 个状态对应 $S_{1...i}$。如果我们将 SAM 看作一棵树,树根为 $0$ 号节点(初始状态),其余节点 $v$ 满足其父亲为该节点的后缀连接 $link(v)$。这棵树叫 $parent$ 树。
\item $parent$ 树中的每个节点的终点集合,等于其子树内所有终点节点对应的终点集合的并集。
\item $parent$ 树中,如果节点 $a$ 是 $b$ 的祖先,则节点 $a$ 对应的字符串是节点 $b$ 对应的字符串的后缀。
\item 构成的 $parent$ 树存在一些与树相关的性质,如 $S_{1...p}$ 和 $S_{1...q}$ 的最长公共后缀对应的是 $p, q$ 对应节点间的 LCA 的字符串。
\item 除了初始状态(节点 $0$)以外,每个状态 $i$ 对应的字串数量是 $len(i) - len(link(i))$,因此计算时可以自上而下计算。
\end{itemize}

\begin{minted}{c++}
class SuffixAutomaton {
private:
    static const int MAXN = 100050;
    static const int MAXLOG = 25;
public:
    // 需要维护 right 集合时,加上以下动态开点线段树的代码
    struct Node {
        int val, lson, rson;
    } a[MAXN * 50];
    int tot;        // 权值线段树节点个数

    void pushUp(int rt) {
        a[rt].val = a[a[rt].lson].val + a[a[rt].rson].val;
    }
    void update(int &rt, int l, int r, int pos) {
        if (!rt)
            rt = ++tot, a[rt].lson = a[rt].rson = a[rt].val = 0;
        if (l == r) {
            a[rt].val++;
            return;
        }
        int mid = (l + r) >> 1;
        if (pos <= mid)
            update(a[rt].lson, l, mid, pos);
        else
            update(a[rt].rson, mid + 1, r, pos);
        pushUp(rt);
    }
    int merge(int x, int y) {
        if (!x || !y)
            return x + y;
        int z = ++tot;
        a[z].lson = merge(a[x].lson, a[y].lson);
        a[z].rson = merge(a[x].rson, a[y].rson);
        pushUp(z);
        return z;
    }
    int query(int rt, int l, int r, int k) {
        if (!rt)
            return 0;
        if (l == r)
            return l;
        int mid = (l + r) >> 1;
        if (a[a[rt].lson].val >= k)
            return query(a[rt].lson, l, mid, k);
        if (a[a[rt].rson].val >= k - a[a[rt].lson].val)
            return query(a[rt].rson, mid + 1, r, k - a[a[rt].lson].val);
        return -1;
    }

    /* 后缀自动机 */
    struct State {
        int len, link, nxt[26];
        State(int _len = 0, int _link = 0): len(_len), link(_link) {
            memset(nxt, 0, sizeof nxt);
        }
    } st[MAXN << 1];    // 最多有 2n-1 个节点,开两倍空间

    // sz: 状态个数,last: 上一次插入的字符对应状态,cnt:当前产生子串个数,len 字符串长度
    // ws wv 基数排序数组, endpos[i] 表示 i 状态所代表的 endpos 集合线段树的树根
    int sz, last, cnt, len;
    int endpos[MAXN << 1], ws[MAXN << 1], wv[MAXN << 1], pos[MAXN << 1];
    int f[MAXN << 1][MAXLOG];
    
    int extend(int c, int idx) {
        int cur = sz++, p = last;
        st[cur] = State(st[last].len + 1);

        endpos[cur] = 0;
        update(endpos[cur], 1, len, idx + 1);  // 更新当前点的 endpos,注意权值线段树值域范围

        while (p != -1 && !st[p].nxt[c])
            st[p].nxt[c] = cur, p = st[p].link;
        if (p == -1)
            st[cur].link = 0;
        else {
            int q = st[p].nxt[c];
            if (st[q].len == st[p].len + 1)
                st[cur].link = q;
            else {
                int clone = sz++;
                st[clone] = State(st[p].len + 1, st[q].link);
                memcpy(st[clone].nxt, st[q].nxt, sizeof st[q].nxt);
                
                endpos[clone] = 0;              // 为克隆节点新建 endpos,但不建树

                while (p != -1 && st[p].nxt[c] == q)
                    st[p].nxt[c] = clone, p = st[p].link;
                st[q].link = st[cur].link = clone;
            }
        }
        last = cur;   
        cnt += st[cur].link == -1 ? st[cur].len : st[cur].len - st[st[cur].link].len;   // 字串个数
        return cnt;          
    }
    
    // 基于基数排序的拓扑排序,保证状态间的拓扑关系,即子状态在后,父状态在前
    // 后缀自动机更新信息时,需要先更新子状态 s,再更新父状态 link[s]
    void toposort() {
        memset(ws, 0, sizeof ws);
        memset(wv, 0, sizeof wv);
        for (int i = 0; i < sz; i++)
            wv[st[i].len]++;            // 排序的关键字是 len
        for (int i = 1; i < sz; i++)
            wv[i] += wv[i-1];
        for (int i = 0; i < sz; i++)
            ws[wv[st[i].len]--] = i;
    }
    // 建立后缀自动机
    void build(char str[]) {
        len = strlen(str);
        for (rint i = 0; i < len; i++) {
            extend(str[i] - 'a', i);
            pos[i] = last;
        }
        // 预处理倍增表
        for (rint i = 0; i < sz; i++)
            f[i][0] = st[i].link;
        for (rint j = 1; j < MAXLOG; j++)
            for (rint i = 0; i < sz; i++)
                f[i][j] = f[f[i][j-1]][j-1];
        toposort();
        // 如果需要维护 right 集合:从子节点开始,合并 endpos 集合
        for (rint i = sz; i > 0; i--) {
            int tmp = ws[i];
            if (st[tmp].link != -1)
                endpos[st[tmp].link] = merge(endpos[st[tmp].link], endpos[tmp]);
        }
    }
    
    int solve(int l, int r, int k) {
        // 首先从 endpos 为 r 的节点开始,倍增找到与目标串一样长的点
        int p = pos[r], curlen = r - l + 1;
        for (rint i = MAXLOG - 1; i >= 0; i--)
            if (st[f[p][i]].len >= curlen)
                p = f[p][i];
        int ans = query(endpos[p], 1, len, k);
        
        if (ans != -1)
            ans -= (curlen - 1);
        return ans;
    }

    void init() {
        cnt = 0, tot = 0, sz = 0;
        st[sz] = State(0, -1), last = sz++;
        memset(a, 0, sizeof a);     // 清空权值线段树
    }
    SuffixAutomaton() { init(); }
} SAM;
\end{minted}

\clearpage