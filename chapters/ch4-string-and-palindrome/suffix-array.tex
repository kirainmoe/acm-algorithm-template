\section{后缀数组}

\paragraph{定义}

\begin{itemize}
    \item $sa[i]$:排名为 $i$ 的后缀的起始下标(排第 $i$ 的是哪个后缀)
    \item $rank[i]$:起始下标为 $i$ 的后缀 $suffix(i ... len-1)$ 的排名(第 $i$ 个后缀排第几),满足 $sa[rank[i]] = i, rank[sa[i]] = i$.
    \item $height[i]$: $sa[i]$ 与 $sa[i-1]$(排名相邻的两个后缀) 的最长公共前缀。
    \item $h[i] = height[rank[i]]$: $suffix(i)$ 和在他前一名的后缀的最长公共前缀。满足 $h[i] \ge h[i-1] - 1$.
\end{itemize}

\begin{minted}{c++}
namespace SuffixArray {
    const int MAXN = 100050, MAXM = 50, MAXLOG = 30;

    char s[MAXN];
    int n, m;
    int sa[MAXN], rank[MAXN], height[MAXN], h[MAXN];
    int t1[MAXN], t2[MAXN], ws[MAXN], wv[MAXN];
    int log2[MAXN], minh[MAXN][MAXLOG];

    // 计算后缀数组和排名数组
    void build() {
        register int i, k, p, *x = t1, *y = t2;
        n = strlen(s) + 1;      // n 要多出 1
        for (i = 0; i < n - 1; i++) 
            s[i] -= '0', m = max(m, (int)s[i] + 1);
        s[n] = 0;               // important!
        for (i = 0; i < m; i++)
            ws[i] = 0;
        for (i = 0; i < n; i++)
            ws[x[i] = s[i]]++;
        for (i = 1; i < m; i++)
            ws[i] += ws[i-1];
        for (i = n - 1; i >= 0; i--)
            sa[--ws[x[i]]] = i;
        for (k = 1, p = 1; p < n; k <<= 1, m = p) {
            for (p = 0, i = n - k; i < n; i++)
                y[p++] = i;
            for (i = 0; i < n; i++)
                if (sa[i] >= k)
                    y[p++] = sa[i] - k;
            for (i = 0; i < n; i++)
                wv[i] = x[y[i]];
            for (i = 0; i < m; i++)
                ws[i] = 0;
            for (i = 0; i < n; i++)
                ws[wv[i]]++;
            for (i = 1; i < m; i++)
                ws[i] += ws[i-1];
            for (i = n - 1; i >= 0; i--)
                sa[--ws[wv[i]]] = y[i];
            swap(x, y);
            for (p = 1, x[sa[0]] = 0, i = 1; i < n; i++)
                x[sa[i]] = (y[sa[i-1]] == y[sa[i]] && y[sa[i-1] + k] == y[sa[i] + k]) ? p-1: p++;
            if (p >= n)
                break;
        }
        // 最终结果 sa[i] 和 rank[i] 分别都从 1 开始,需要将值或下标 -1
        for (i = n; i > 0; i--)
            sa[i] = sa[i-1] + 1;
        for (i = 1; i <= n; i++)
            rank[sa[i]] = i;
    }
    // 计算 height[] 数组
    void calcHeight() {
        register int i, j, k = 0;
        for (i = 1; i <= n; i++) {
            k = k ? k - 1 : 0;
            j = sa[rank[i] - 1];
            while (s[i+k-1] == s[j+k-1])
                k++;
            height[rank[i]] = k;
        }
    }
    // 预处理 log2 的值
    void preprocess() {
        lg2[0] = -1;
        for (int i = 1; i < MAXN; i++)
            lg2[i] = (i & (i - 1)) ? lg2[i-1] : lg2[i-1] + 1;
    }
    // 预处理 ST 表
    void initST() {
        for (int i = 1; i <= n; i++)
            minh[0][i] = height[i];
        for (int i = 1; i <= lg2[n]; i++) {
            int lim = n - (1 << i) + 1;
            for (int j = 1; j <= lim; j++)
                minh[i][j] = min(minh[i-1][j], minh[i-1][j + (1 << i >> 1)]);
        }
    }
    // 询问 suffix(a) 和 suffix(b) 的最长公共前缀
    int lcp(int a, int b) {
        a = rank[a + 1], b = rank[b + 1];
        if (a > b)
            swap(a, b);
        a++;
        int t = lg2[b - a + 1];
        return min(minh[t][a], minh[t][b - (1 << t) + 1]);
    }  
}
\end{minted}

\paragraph{应用}
\begin{itemize}
    \item 求某两个后缀的最长公共前缀:$suffix(j)$ 和 $suffix(k)$ 的最长公共前缀为:$min\{height[rank[j] + 1], height[rank[j] + 2], ... height[rank[k]]\}$;这是一个 RMQ 问题,可以用 ST 表$O(nlogn)$ 预处理,$O(1)$ 查询。
    \item 可重叠最长重复子串:子串一定是某个后缀的前缀;可重叠的子串则等价于求两个后缀的最长公共前缀,所以求 $height[]$ 数组的最大值即可。
    \item 不可重叠最长重复子串:首先在 $[1...max(height)]$ 区间二分答案 $k$,判断是否存在两个长度为 $k$ 的子串是相同且不重叠的。利用 $height$ 值对后缀分组,使得每组的 $height$ 值都不小于 $k$(如果不存在就单独分组)。有希望成为最长公共前缀不小于 k 的两个后缀一定在同一组。然
    后对于每组后缀,只须判断每个后缀的 $sa$ 值的最大值和最小值之差是否不小于 $k$。
    \item 可重叠并出现 $k$ 次的最长重复子串:同样先二分长度答案 $x$ 分成若干组,判断的是有没有一个组的后缀个数不小于 $k$. 如果存在则满足条件。
    \item 不相同的子串个数:因为子串一定是某个后缀的前缀,问题等价于求所有后缀之间不相同的前缀的个数,如果所有的后缀按照 $suffix(sa[i])$ (排名)的顺序计算,对于新加入的后缀 $suffix(sa[k])$,将贡献 $n - sa[k] + 1 - height[k]$ 个不同的子串,累加即可。
    \item 最长回文子串:将整个字符串反过来接在原字符的后面,中间用一个特殊的字符隔开,求新字符串某两个后缀的最长公共前缀即可。
    \item 连续重复子串:(定义:字符串 $L$ 由某个字符串 $S$ 重复 $r$ 次得到,则 $L$ 称为连续重复子串)
    \item 求 $r$ 的最大值:枚举字符串 $S$ 的长度 $k$,然后判断是否满足。判断时先看 $L$ 的长度能否被 $k$ 整除,再看 $suffix(1)$ 和 $suffix(k+1)$ 的最长公共前缀是否等于 $n-k$,如果是则合法。
    \item 重复次数最多的连续重复子串:首先预处理 LCP;枚举重复部分的长度 $l$,然后枚举每一个起始位置$i$。如果重复部分出现大于等于 2 次,那么一定会有 $s[0], s[l], s[2*l]...s[k*l]$ 其中两个连续出现在重复组成的串中。所以对于确定的长度,$O(1)$ 查询 $s[i]$ 和 $s[i+l]$ 的公共前缀,然后向前向后匹配,重复串长度即为 $lcp(i, i + l) + (l - k \% l)$,然后再检查一下起点 $t=i-l+k\%l$ 是否溢出,以及 $lcp(t, t+l)$ 的长度是否大于 $k$ 即可更新答案。时间复杂度为 $O(nlogn)$.
    \item $A,B$ 的最长公共子串:首先把 $B$ 连接到 $A$ 的末尾,两者中间用一个没出现过的字符(如\$隔开),然后求新串的后缀数组、height 数组等。然后遍历 height 数组,当 $suffix(sa[i])$ 和 $suffix(sa[i-1])$ 不是同一个字符串中的两个后缀时,最大的$height[i]$就满足条件。判断 $suffix(sa[i])$ 和 $suffix(sa[i-1])$ 是否为同一个字符串中的两个后缀,只需判断下标的位置即可。
    \item $A,B$ 的长度不小于 $k$ 的公共子串个数(可以相同):思路是计算 A, B 的所有后缀之间的 lcp 的长度,统计 $lcp \ge k$ 的答案。首先把 $B$ 连接到 $A$ 的末尾,两者中间用一个没出现过的字符(如\$隔开),计算 height 后按 height 值大于等于 $k$ 分组。然后将后缀扫描一遍,遇到一个 B 的后缀就统计与前面的 A 的后缀能产生多少个长度不小于 $k$ 的公共子串。A 的后缀可以用单调栈维护;然后将 A 用相同的办法统计一次。
    \item 给定 $n$ 个字符串,求出现在不小于 $k$ 个字符串中的最长子串:将 $n$ 个字符串连接,中间用没出现过的字符隔开,求后缀数组。然后二分长度 $l$ 并利用 $height$ 数组分组,判断每组的后缀是否出现在不小于 $k$ 个原串中。
    \item 给定 $n$ 个字符串,求在每个字符串中至少出现两次且不重叠的最长子串:和上面一样,判断的时候,要看是否有一组后缀在每个原来的字符串中至少出现两次,并且在每个原来的字符串中,后缀的起始位置的最大值与最小值之差是否不小于当前答案(判断能否做到不重叠,如果题目中没有不重叠的要求,那么不用做此判断).
    \item 给定 $n$ 个字符串,求出现或反转后出现在每个字符串中的最长子串:将每个字符串反转后的结果也拼到总串中,求后缀数组。判断的时候,要看是否有一组后缀在每个原来的字符串或反转后的字符串中出现.
\end{itemize}

\clearpage