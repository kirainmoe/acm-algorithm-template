\section{回文树/回文自动机}
\paragraph{功能}
\begin{itemize}
\item 求前缀字符串 $s[0] \sim s[i]$ 中的本质不同的回文串种类
\item 求每个本质不同的回文串出现的次数,求回文串的个数
\item 求以下标 $i$ 的字符结尾的回文串个数/种类
\item 每个本质不同的回文串,包含的本质不同回文串种类
\end{itemize}
\begin{minted}{c++}
class PalindromicTree {
public:
    struct Node {
        // len: 当前节点回文串长度;fail: 失配指针
        // cnt: 当前节点表示的回文串在 s 中出现的次数
        // num: 以节点 i 表示的回文串末尾字符结尾,但不包含本条路径上的回文串的数目(fail 指针路径深度) 
        int len, fail, cnt, num, next[26];
        Node(int l = 0, int c = 0, int f = 0, int n = 0): len(l), fail(f), cnt(c), num(n) {
            memset(next, 0, sizeof next);
        }
    } t[MAXN];
    int p, n, cur, last, s[MAXN];	// p: 节点数,n: 字符数
    
    int alloc(int len, int fail) {
        t[p] = Node(len, 0, fail);
        return p++;
    }
    
    void init() {
        p = n = cur = last = 0;
        alloc(0, 1), alloc(-1, 0);	// len, fail
        s[0] = -1;
    }
    
    int getFail(int x) {
        while (s[n - t[x].len - 1] != s[n])
            x = t[x].fail;
        return x;
    }
    
    void add(int c) {
        c = c >= 'a' ? c - 'a' : c;
        s[++n] = c;
        
        int cur = getFail(last);
        if (!t[cur].next[c])
            t[cur].next[c] =  alloc(t[cur].len + 2, t[getFail(t[cur].fail)].next[c]);
        last = t[cur].next[c];
        t[last].cnt++;
    }
    
    void count() {
        for (rint i = p - 1; i >= 0; i--)
            t[t[i].fail].cnt += t[i].cnt;
    }
} PT;
\end{minted}

