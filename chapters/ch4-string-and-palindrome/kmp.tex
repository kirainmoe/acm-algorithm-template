\section{KMP 算法}
\begin{minted}{c++}
namespace KMP {
    int next[MAXN];
    void calcNextArr() {
      next[0] = -1;
      int j = 0, k = -1, len = strlen(pattern) - 1;
      while (j < len) {
        if (k == -1 || pattern[j] == pattern[k]) {
          next[++j] = ++k;
          if (pattern[j] == pattern[k])
            next[j] = next[k];
        } else
          k = next[k];
      }
    }

    int KMP() {
      int i = 0, j = 0;
      int tlen = strlen(target), plen = strlen(pattern);
      while (i < tlen && j < plen)
        if (j == -1 || target[i] == pattern[j])
          i++, j++;
        else
          j = next[j];  
      if (j == plen)
        return i - j;
      else
        return -1;
    }
}
\end{minted}

\paragraph{KMP 求最小循环节} 假设 $S$ 的长度为 $len$,则 $S$ 存在最小循环节,循环节的长度 $L$ 为 $len-next[len]$,子串为 $S[0…len-next[len]-1]$。
\begin{itemize}
  \item 如果 $len$ 可以被 $len - next[len]$ 整除,则表明字符串 $S$ 可以完全由循环节循环组成,循环周期 $T=len/L$。
  \item 如果不能,说明还需要再添加几个字母才能补全。需要补的个数是循环个数 $L-len \mod L=L-(len-L) \mod L=L-next[len] \mod L,L=len-next[len]$。
\end{itemize}