\section{多项式求导 \& 指对数运算}
\begin{minted}{c++}
namespace Polynomial {  
    const ll P = 998244353, g = 3, gi = 332748118;  
    static int rev[N];  
    int lim, bit;  
    int add(int a, int b) {  
        return (a += b) >= P ? a - P : a;  
    }  
    int qpow(int a, int b) {  
        int prod = 1;  
        while(b) {  
            if(b & 1) prod = (ll)prod * a % P;  
            a = (ll)a * a % P, b >>= 1;  
        }
        return (prod + P) % P;  
    }  
    void process_inverse() {  
        for(int i = 1; i < lim; ++i)  
            rev[i] = (rev[i >> 1] >> 1) | ((i & 1) << (bit - 1));  
    }  
    void NTT(int *A, int inv) {  
        for(int i = 0; i < lim; ++i)  
            if(i < rev[i]) swap(A[i], A[rev[i]]);  
        for(int mid = 1; mid < lim; mid <<= 1) {  
            int tmp = qpow(inv == 1 ? g : gi, (P - 1) / (mid << 1));  
            for(int j = 0; j < lim; j += (mid << 1)) {  
                int w = 1;  
                for(int k = 0; k < mid; ++k, w = (ll)w * tmp % P) {  
                    int x = A[j + k], y = (ll)w * A[j + k + mid] % P;  
                    A[j + k] = (x + y) % P;  
                    A[j + k + mid] = (ll)(x - y + P) % P;  
                }  
            }  
        }  
        if(inv == 1) return;  
        int invn = qpow(lim, P - 2);  
        for(int i = 0; i < lim; ++i)  
            A[i] = (ll)A[i] * invn % P;  
    }
    static int x[N], y[N];  
    void mul(int *a, int *b) {  
        memset(x, 0, sizeof x);  
        memset(y, 0, sizeof y);  
        for(int i = 0; i < (lim >> 1); ++i)  
            x[i] = a[i], y[i] = b[i];  
        NTT(x, 1), NTT(y, 1);  
        for(int i = 0; i < lim; ++i)  
            x[i] = (ll)x[i] * y[i] % P;  
        NTT(x, -1);  
        for(int i = 0; i < lim; ++i)  
            a[i] = x[i];  
    }  
    static int c[2][N];  
    void inv(int *a, int n) {  
        int p = 0;  
        memset(c, 0, sizeof c);  
        c[0][0] = qpow(a[0], P - 2);  
        lim = 2, bit = 1;  
        while(lim <= (n << 1))  
        {  
            lim <<= 1, bit++;  
            process_inverse();  
            p ^= 1;  
            memset(c[p], 0, sizeof c[p]);  
            for(int i = 0; i <= lim; ++i)  
                c[p][i] = add(c[p^1][i], c[p^1][i]);  
            mul(c[p^1], c[p^1]);  
            mul(c[p^1], a);  
            for(int i = 0; i <= lim; ++i)  
                c[p][i] = add(c[p][i], P - c[p^1][i]);  
        }  
        for(int i = 0; i < lim; ++i)  
            a[i] = c[p][i];  
    }  
    void derivative(int *a, int n) {  
        for(int i = 1; i <= n; ++i)  
            a[i - 1] = (ll)a[i] * i % P;  
        a[n] = 0;  
    }  
    void inter(int *a, int n) {  
        for(int i = n; i >= 1; --i)  
            a[i] = (ll)a[i - 1] * qpow(i, P - 2) % P;  
        a[0] = 0;  
    }  
    static int b[N], T[N], K[N];  
    void ln(int a[], int n) {  
        memcpy(b, a, sizeof b);  
        inv(b, n), derivative(a, n);  
        while(lim <= (n << 2)) lim <<= 1, bit++;  
        process_inverse();  
        mul(a, b);  
        inter(a, n);  
        for(int i = n + 1; i <= lim; ++i)  
            a[i] = 0;  
    }  
    void exp(int a[], int n) {  
        int z, d;  
        z = lim = 2, d = bit = 1;  
        memset(K, 0, sizeof K),  K[0] = 1;  
        while(z <= (n << 1)) {  
            z <<= 1, d++;  
            for(int i = 0; i < (z>>1); ++i)   
                T[i] = K[i];  
            ln(T, (z >> 1) - 1);  
            for(int i = 0; i < (z >> 1); ++i)  
                T[i] = add(a[i] + (i == 0), P - T[i]);  
            lim = z, bit = d, process_inverse(), mul(K, T);  
            for(int i = z; i <= (z << 1); ++i)  
                K[i] = T[i] = 0;  
        }  
        for(int i = 0; i <= n; ++i)  
            a[i] = K[i];
    }  
}  

\end{minted}