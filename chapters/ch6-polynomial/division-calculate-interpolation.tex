\section{多项式除法、多点求值、快速插值}

\par 来自 \href{https://brooksj.com/2019/07/24/%E5%A4%9A%E9%A1%B9%E5%BC%8F%E7%9B%B8%E5%85%B3%E7%AE%97%E6%B3%95%E9%9B%86%E6%88%90/}{《多项式相关算法集成》}。

\begin{minted}{c++}
const ll mod_v = 17ll * (1 << 27) + 1;      // 998244353
const int MAXL = 18, N = 1 << MAXL, M = 1 << MAXL;  // MAXL最大取 mov_v 的2次幂
struct polynomial {
​    ll coef[N]; // size设为多项式系数个数的两倍
​    ll a[N], b[N], d[N], r[N];
​    ll xcoor[N], ycoor[N], v[N], mcoef[N];  // size设为点个数的两倍
​    vector<vector<ll> > poly_divisor;
​     
​    static ll mod_pow(ll x,ll n,ll m) {
​        ll ans = 1;
​        while(n > 0){
​            if(n & 1)
​                ans = ans * x % m;
​            x = x * x % m, n >>= 1;
​        }
​        return ans;
​    }
​     
​    struct FastNumberTheoreticTransform  {
​        ll omega[N], omegaInverse[N];
​        int range;
​        void init (const int& n) {
​            range = n;
​            ll base = mod_pow(3, (mod_v - 1) / n, mod_v);
​            ll inv_base = mod_pow(base, mod_v - 2, mod_v);
​            omega[0] = omegaInverse[0] = 1;
​            for (int i = 1; i < n; ++i) {
​                omega[i] = omega[i - 1] * base % mod_v;
​                omegaInverse[i] = omegaInverse[i - 1] * inv_base % mod_v;
​            }
​        }
​        void transform (ll *a, const ll *omega, const int &n) {
​            for (int i = 0, j = 0; i < n; ++i) {
​                    if (i > j) std::swap (a[i], a[j]);
​                    for(int l = n >> 1; ( j ^= l ) < l; l >>= 1);
​            }
​            for (int l = 2; l <= n; l <<= 1) {
​                int m = l / 2;
​                for (ll *p = a; p != a + n; p += l) {
​                    for (int i = 0; i < m; ++i) {
​                        ll t = omega[range / l * i] * p[m + i] % mod_v;
​                        p[m + i] = (p[i] - t + mod_v) % mod_v;
​                        p[i] = (p[i] + t) % mod_v;
​                    }
​                }
​            }
​        }
        void dft (ll *a, const int& n) {
            transform(a, omega, n);
        }
        void idft (ll *a, const int& n) {
            transform(a, omegaInverse, n);
            for (int i = 0; i < n; ++i) a[i] = a[i] * mod_pow(n, mod_v - 2, mod_v) % mod_v;
        }
    } fnt ;
    
    ll mod_trans(ll v) {
        return abs(v) <= mod_v / 2 ? v : (v < 0 ? v + mod_v : v - mod_v);
    }
    
    // 二分求 $\prod{x-xi}$ 的所有二分子多项式的系数
    void binary_subpoly(int l, int r, int idx) {
        if (l == r - 1) {
            poly_divisor[idx].push_back(-xcoor[l]);
            poly_divisor[idx].push_back(1);        
        } else {
            int lidx = (idx << 1) + 1, ridx = lidx + 1;
            binary_subpoly(l, (l + r) / 2, lidx);
            binary_subpoly((l + r) / 2, r, ridx);
            int t = poly_divisor[lidx].size() + poly_divisor[ridx].size() - 1;
            int p = 1;
            while(p < t) p <<= 1;
            copy(poly_divisor[lidx].begin(), poly_divisor[lidx].end(), a);
            fill(a + poly_divisor[lidx].size(), a + p, 0);
            copy(poly_divisor[ridx].begin(), poly_divisor[ridx].end(), b);
            fill(b + poly_divisor[ridx].size(), b + p, 0);
            fnt.dft(a, p);
            fnt.dft(b, p);
            for (int i = 0; i < p; i++)  a[i] *= b[i]; 
            fnt.idft(a, p);
            for (int i = 0; i < t; i++) poly_divisor[idx].push_back(mod_trans(a[i]));
        }
    }
    
    // 模 $x^deg$ ,a为要求逆元的多项式系数,结果存放在b[0~deg]中
    // T(deg) = T(deg/2) + deg*log(deg),复杂度O(deg*log(deg))
    void polynomial_inverse(int deg, ll* a, ll* b, ll* tmp) {
        if(deg == 1) {
            b[0] = mod_pow(a[0], mod_v - 2, mod_v);
        } else {
            polynomial_inverse((deg + 1) >> 1, a, b, tmp);
            int p = 1;
            while(p < (deg << 1) - 1) p <<= 1;
            copy(a, a + deg, tmp);
            fill(tmp + deg, tmp + p, 0);
            fill(b + ((deg + 1) >> 1), b + p, 0);
            //fnt.init(p);
            fnt.dft(tmp, p);
            fnt.dft(b, p);
            for(int i = 0; i != p; ++i) {
                b[i] = (2 - tmp[i] * b[i] % mod_v) * b[i] % mod_v;
                if(b[i] < 0) b[i] += mod_v;
            }
            fnt.idft(b, p);
            fill(b + deg, b + p, 0);
        }
    }
    
    // A = D*B + R,A为n项n-1次幂,B为m项m-1次幂,D为n-m+1项n-m次幂,R为m-1项m-2次幂
    // 要求a,b中系数以低次到多次顺序排列
    // n >= m,复杂度O(n*logn);n < m,复杂度O(n)
    int polynomial_division(int n, int m, ll *A, ll *B, ll *D, ll *R) {
        if (n < m) {
            copy(A, A + n, R);
            return n;
        } else {
            static ll A0[N], B0[N], tmp[N]; //数组太大会爆栈,添加到全局区
    
            int p = 1, t = n - m + 1;
            while(p < (t << 1) - 1) p <<= 1;
    
            fill(A0, A0 + p, 0);
            reverse_copy(B, B + m, A0);
            polynomial_inverse(t, A0, B0, tmp);
            fill(B0 + t, B0 + p, 0);
            fnt.dft(B0, p);
    
            reverse_copy(A, A + n, A0);
            fill(A0 + t, A0 + p, 0);
            fnt.dft(A0, p);
    
            for(int i = 0; i != p; ++i)
                A0[i] = A0[i] * B0[i] % mod_v;
            fnt.idft(A0, p);
            reverse(A0, A0 + t);
            copy(A0, A0 + t, D);
    
            for(p = 1; p < n; p <<= 1);
            fill(A0 + t, A0 + p, 0);
            fnt.dft(A0, p);
            copy(B, B + m, B0);
            fill(B0 + m, B0 + p, 0);
            fnt.dft(B0, p);
            for(int i = 0; i != p; ++i)
                A0[i] = A0[i] * B0[i] % mod_v;
            fnt.idft(A0, p);
            for(int i = 0; i != m - 1; ++i)
                R[i] = ((A[i] - A0[i]) % mod_v + mod_v) % mod_v;
            //fill(R + m - 1, R + p, 0);
            return m - 1;
        }
    }
    
    // 多项式的点值计算
    // l和r为存储要求的点数组的左右边界(左闭右开), idx为除数多项式索引(初始0)
    // polycoef为用于计算点的多项式系数,num为其系数个数
    // 设多项式项数x=num,点数y=r-l,n=max(x,y),复杂度O(n(logn)^2)
    void polynomial_calculator(int l, int r, int idx, int num, ll *polycoef) {
        int mid = (l + r) / 2;
        int lidx = (idx << 1) + 1, ridx = lidx + 1;
        int lsize = poly_divisor[lidx].size(), rsize = poly_divisor[ridx].size();
        ll *lmod_poly = new ll[lsize - 1], *rmod_poly = new ll[rsize - 1];
        copy(poly_divisor[lidx].begin(), poly_divisor[lidx].end(), a);
        copy(poly_divisor[ridx].begin(), poly_divisor[ridx].end(), b);
        int lplen = polynomial_division(num, lsize, polycoef, a, d, lmod_poly);
        int rplen = polynomial_division(num, rsize, polycoef, b, d, rmod_poly);
        if (l == mid - 1) {
            v[l] = mod_trans(lmod_poly[0]);
        } else {
            polynomial_calculator(l, (l + r) / 2, lidx, lplen, lmod_poly);
        }
        if (r == mid + 1) {
            v[(l + r) / 2] = mod_trans(rmod_poly[0]);
        } else {
            polynomial_calculator((l + r) / 2, r, ridx, rplen, rmod_poly);
        }
        delete []lmod_poly;
        delete []rmod_poly;
    }
    
    // 拉格朗日插值:二分+快速数论变换
    // l和r为存储要求的点数组的左右边界(左闭右开), idx为由点二分构造出多项式的索引(初始0)
    // polycoef为插值得到的多项式结果
    // 设点个数为n,复杂度O(n*(logn)^2),结果polycoef为n-1次多项式
    void polynomial_interpolate(int l, int r, int idx, ll *polycoef) {
        if (l == r - 1) {
            polycoef[0] = ycoor[l] * mod_pow(v[l], mod_v - 2, mod_v) % mod_v;
        } else {
            int mid = (l + r) >> 1;
            int lidx = (idx << 1) + 1, ridx = lidx + 1;
            int sz = poly_divisor[idx].size() - 1;
            int lsize = poly_divisor[lidx].size(), rsize = poly_divisor[ridx].size();
            int p = 1;
            while (p < sz) p <<= 1;
            ll *leftpoly = new ll[p], *rightpoly = new ll[p];
            polynomial_interpolate(l, mid, lidx, leftpoly);
            polynomial_interpolate(mid, r, ridx, rightpoly);
            copy(poly_divisor[lidx].begin(), poly_divisor[lidx].end(), a);
            copy(poly_divisor[ridx].begin(), poly_divisor[ridx].end(), b);
            fill(leftpoly + lsize - 1, leftpoly + p, 0);
            fill(rightpoly + rsize - 1, rightpoly + p, 0);
            fill(a + lsize, a + p, 0);
            fill(b + rsize, b + p, 0);
            fnt.dft(leftpoly, p);
            fnt.dft(b, p);
            for (int i = 0; i < p; i++) leftpoly[i] = leftpoly[i] * b[i] % mod_v;
            fnt.idft(leftpoly, p);
            fnt.dft(rightpoly, p);
            fnt.dft(a, p);
            for (int i = 0; i < p; i++) rightpoly[i] = rightpoly[i] * a[i] % mod_v;
            fnt.idft(rightpoly, p);
            for (int i = 0; i < sz; i++) polycoef[i] = mod_trans((leftpoly[i] + rightpoly[i]) % mod_v); 
            delete []leftpoly;
            delete []rightpoly; 
        }
    }
    
    // 初始化点个数到二分子多项式个数
    // 调用polynomial_calculator和polynomial_interpolate前调用
    void init(int vnum) {
        int vnum2 = 1;
        while (vnum2 < vnum) vnum2 <<= 1;
        for (int i = 0; i < 2 * vnum2 - 1; i++)  poly_divisor.push_back(vector<ll>());
    }
} poly;
\end{minted}

用法:

\begin{minted}{c++}
// 多项式除法 & 取模
poly.fnt.init(1 << MAXL);
for(int i = 0; i < n; ++i)
    cin >> poly.a[i]; // 0次幂系数开始输入,缺失的幂系数输入0
for (int i = 0; i < m; i++)
    cin >> poly.b[i]; // 0次幂系数开始输入
int rlen = poly.polynomial_division(n, m, poly.a, poly.b, poly.d, poly.r);
for (int i = 0; i < n - m + 1; i++) 
    cout << poly.d[i] << " ";       // 商
for (int i = 0; i < rlen; i++)
    cout << poly.r[i] << " ";       // 余数

// 多点求值
poly.fnt.init(1 << MAXL);
int n, vnum; 
cin >> n;
for (int i = 0; i < n; i++)
    cin >> poly.coef[i];        // 多项式系数
cin >> vnum;
for (int i = 0; i < vnum; i++)
    cin >> poly.xcoor[i];       // 要求的点 $x_i$                                                                              
poly.init(vnum);
poly.binary_subpoly(0, vnum / 2, 1), poly.binary_subpoly(vnum / 2, vnum, 2); 
poly.polynomial_calculator(0, vnum, 0, n, poly.coef);
for (int i = 0; i < vnum; i++)
    cout << poly.v[i] << " ";

// 快速插值
poly.fnt.init(1 << MAXL);   
int vnum; 
cin >> vnum;            // 点的个数
for (int i = 0; i < vnum; i++)
    cin >> poly.xcoor[i] >> poly.ycoor[i];
poly.init(vnum);
poly.binary_subpoly(0, vnum, 0); 
for (unsigned int i = 1; i < poly.poly_divisor[0].size(); i++) {
    poly.mcoef[i - 1] = poly.poly_divisor[0][i] * i % mod_v;
}
// 遍历i计算所有 $\sum_{j!=i}{xi-xj}$
poly.polynomial_calculator(0, vnum, 0, poly.poly_divisor[0].size() - 1, poly.mcoef);
// 拉格朗日插值计算多项式
poly.polynomial_interpolate(0, vnum, 0, poly.coef);
// 输出插值得到的多项式系数 
for (int i = 0; i < vnum; i++)
    cout << poly.coef[i] << " ";
cout << endl;
// 将输入点代入插值得到的多项式中进行验证,输出计算得到的结果
poly.polynomial_calculator(0, vnum, 0, vnum, poly.coef);
for (int i = 0; i < vnum; i++)
    cout << poly.v[i] << " ";
\end{minted}